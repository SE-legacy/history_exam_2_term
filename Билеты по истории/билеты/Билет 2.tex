\section{Билет 2. Развитие России в условиях двоевластия (март – сентябрь 1917 г.).}

\subsection{Июльские события — первая попытка захвата власти большевиками}

\subsubsection{\textbf{Июль 1917 г. – первый всероссийский съезд советов.}}

\textbf{Состав:} большинство – эсеры, потом меньшевики, большевики.

Эсеры и меньшевики заявляют: на данный момент не существует настолько стабильной партии, которая смогла бы удержать власть в своих руках.

Ленин в ответ на это заявляет, что его партия (большевиков) способна это сделать.

Большевики продолжают активную пропаганду и в середине июля в крупных городах (Петроград, Рига, Иваново-Вознесенск) устраивают серию демонстраций.

Лозунги демонстрантов: «Долой войну», «Долой временное правительство» и «Вся власть советам».

Демонстрации разгоняются и это расценивается как попытка захвата власти большевиками, партия объявляется вне закона.

Ленин сначала маскируется под фальшивыми документами, а позже уезжает из Петрограда в Разлив. Там они продолжают свою деятельность посредством писем и записок. Эти действия спровоцировали второй кризис временного правительства.

\subsection{Выступление Л. Г. Корнилова и его последствия.}

Сторонники диктатуры организовывают попытку эту диктатуру ввести. Главнокомандующим сделали Л. Г. Корнилова. По их мнению, единственный способ предотвратить гражданскую войну и хаос – установить военную диктатуру.

Корнилов отдаёт приказ отозвать некоторые дивизии с фронта и отправляет их в Петроград. На что издаётся указ об аресте Корнилова. Рабочие Петрограда вооружаются и готовятся к обороне. Но мятеж произошёл изнутри из-за отсутствия поддержки Корнилова в самой армии и последний был арестован.

\subsubsection{\textbf{Последствия:}}

В связи с попыткой Корниловского мятежа, вырос авторитет большевиков и они стали занимать всё больше мест в выборных органах власти. Лидером Петроградского совета избрали Л. Троцкого.

Во временном правительстве (по инициативе Керенского) создаётся директория (над временным правительством).

\subsubsection{\textbf{Состав директории ВП:}}

Керенский, Терещенко, Никитин.

\subsubsection{\textbf{Призвание директории ВП:}}

Установить диктатуру во временном правительстве.

(Кадеты с этим не согласны и уходят из временного правительства, отказываясь от сотрудничества с Керенским.)

В середине сентября устраивается Петроградское демократическое совещание социалистических партий (с 14 сентября). Поднимался вопрос: «Кому должна принадлежать власть в стране?». Ведутся споры.

Из состава этого совещания выделяется еще один орган – предпарламент.

\subsubsection{\textbf{Призвание предпарламента:}}
 
Поддержка временного правительства.

Однако идея провалилась, так как большинство его членов не готовы были поддерживать Керенского, в частности из-за вопроса войны.

К концу сентября из 16 министров временного правительства 10 было социалистами, остальные беспартийные.

Однако это не предотвратило скатывание страны в беспорядки.