\section{Билет 2. Развитие России в условиях двоевластия (март – сентябрь 1917 г.).}

\subsection{Общественно"= политическая обстановка в стране весной 1917 г.}

В стране установлены два органа власти – «Совет рабочих и солдатских депутатов» и «Временное правительство», сформированное «Временным комитетом».

Во временном правительстве министром иностранных дел становится Милюков. Военным и морским министром – Гучков. Министром торговли и промышленности – Коновалов. Министром финансов – Терещенко. Министр Юстиции – Керенский. Всего из 12 членов временного правительства: 5 кадетов, 2 беспартийных и т.д.

Февральская революция открыла возможность демократического преобразования страны, но предстояло выбрать и сформировать конституцию, решить национальные противоречия, аграрный вопрос и определить отношение к войне. Кроме того, требовалось снабдить население продовольствием и топливом и стабилизировать ситуацию в стране.

Временное правительство и петроградский совет приходят к соглашению по 2 пунктам:

\begin{enumerate}
    \item Нужно избрать учредительное собрание по «четырёхвостке» (всеобщие, тайные, равные, прямые).
    \item Войну нужно продолжать.
\end{enumerate}
Однако на практике отношение между советом и правительством было острым, так как фактическую власть имел именно совет и армия прислушивалась к совету.



\subsection{Организация власти в центре и на местах.}

В начале марта на местах власть переходит к общественным исполнительным комитетам. Их участниками зачастую были члены земств и городских дум. 
Параллельно формируется петроградский совет рабочих и солдатских депутатов. А также формируются советы по всей стране. Они ориентируются на петроградский совет, как на главный орган.


\subsection{Внутренняя и внешняя политика Временного правительства}

Деятельность временного правительства носила дух широких демократических преобразований: провозглашение прав и свобод, отмена смертной казни, упраздняется цензура, реорганизуется структура и руководящий состав политических органов. В то же время санкционируется арест Николая 2 и его семьи. По расследованию их прежних дел создаётся чрезвычайная комиссия.

 апреле 1917 года происходит первый кризис временного правительства по вопросу войны. Оно считало, что войну нужно продолжать. Союзникам рассылается нота Милюкова, в которой говорится, что Россия будет продолжать войну до победного конца. Это приводит к бурному недовольству внутри страны и слышны голоса о требовании мира.
 
В апреле 1917 года в страну возвращаются из эмиграции многие лидеры политических партий, в частности, возвращается Владимир Ульянов. Перед толпой встречающих он произносит речь, также известную как апрельские тезисы:

\begin{enumerate}
    \item Нужно немедленно закончить войну любым способом
    \item Провести национализацию земли и банков (переход к социализму)
    \item Никакой поддержки временному правительству, вся власть советам, а большинство мест в советах должно принадлежать большевикам.
\end{enumerate}

\subsection{Июльские события — первая попытка захвата власти большевиками}

\subsubsection{июль 1917 г. – первый всероссийский съезд советов. Раскол революционной демократии}

\textbf{Состав:} большинство – эсеры, потом меньшевики, большевики.

Эсеры и меньшевики заявляют: на данный момент не существует настолько стабильной партии, которая смогла бы удержать власть в своих руках.

Ленин в ответ на это заявляет, что его партия (большевиков) способна это сделать.

Большевики продолжают активную пропаганду и в середине июля в крупных городах (Петроград, Рига, Иваново-Вознесенск) устраивают серию демонстраций.

Лозунги демонстрантов: «Долой войну», «Долой временное правительство» и «Вся власть советам».

Демонстрации разгоняются и это расценивается как попытка захвата власти большевиками, партия объявляется вне закона.

Ленин сначала маскируется под фальшивыми документами, а позже уезжает из Петрограда в Разлив. Там они продолжают свою деятельность посредством писем и записок. Эти действия спровоцировали второй кризис временного правительства.

Лидер временного правительства Львов понимает, что правительство не пользуется популярностью у народа и уходит в отставку. Новой главой становится Керинский. Также уходят несколько других министров и теперь большинство во временном правительстве занимают министры-социалисты.

В печати появляются сообщения, дискредитирующие большевиков, где их обвиняют в союзе с немцами. Факт в том, что немецкое правительство действительно финансировало действия большевиков (естественно не на прямую). Это привело к временному падению популярности большевиков.

\subsection{Выступление Л. Г. Корнилова и его последствия.}

Сторонники диктатуры организовывают попытку эту диктатуру ввести. Главнокомандующим сделали Л. Г. Корнилова. По их мнению, единственный способ предотвратить гражданскую войну и хаос – установить военную диктатуру.

Корнилов отдаёт приказ отозвать некоторые дивизии с фронта и отправляет их в Петроград. На что издаётся указ об аресте Корнилова. Рабочие Петрограда вооружаются и готовятся к обороне. Но мятеж произошёл изнутри из-за отсутствия поддержки Корнилова в самой армии и последний был арестован.

\subsubsection{\textbf{Последствия:}}

В связи с попыткой Корниловского мятежа, вырос авторитет большевиков, которые также выступили против него, и они стали занимать всё больше мест в выборных органах власти. Лидером Петроградского совета избрали Л. Троцкого.

\subsection{Последний виток кризиса власти: Директория, Всеросийское демократическое совещание, Предпарламент}

Во временном правительстве (по инициативе Керенского) создаётся директория (орган над временным правительством).

\subsubsection{\textbf{Состав:}}

Керенский, Терещенко, Никитин.

\subsubsection{\textbf{Призвание:}}

Установить диктатуру во временном правительстве.

(Кадеты с этим не согласны и уходят из временного правительства, отказываясь от сотрудничества с Керенским.)

В середине сентября устраивается Петроградское демократическое совещание социалистических партий (с 14 сентября). Поднимался вопрос: «Кому должна принадлежать власть в стране?». Ведутся споры.

Из состава этого совещания выделяется еще один орган – предпарламент, цель которого -- поддержка временного правительства. Однако к тому моменту Керенский не пользуется никаким авторитетом.

К концу сентября из 16 министров временного правительства 10 было социалистами, остальные беспартийные.

Однако это не предотвратило скатывание страны в беспорядки.