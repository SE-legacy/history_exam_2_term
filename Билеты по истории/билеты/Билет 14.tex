\section{Билет 14. Внешняя политика Советского Союза в 1922--1939 гг.}

В последние годы гражданской войны нормализация отношений началась с торговли. Иностранная интервенция прекратилась к концу 1919 года. Великобритания, Франция выводят свои войска из европейской части России, а в 1920 году подписывается договор о возобновлении торговых отношений. Обе страны в этом нуждались. Когда в 1921 году разразился голод, правительство было вынуждено обратиться к международным организациям за помощью (к Красному Кресту и американской ARA в частности). 

\subsection{Версалько"=Вашингтонская система мироустройства.}

Версальский мирный договор положил начало Версальско"=Вашингтонской системе международных отношений (еще была конференция в вашингтоне). 

Её смысл в следующем:

\begin{enumerate}
    \item Дискриминация побеждённых государств и советской России (Сокращались территории побежденных государств, теряли колонии и выплачивали репарации).
    \item Закрепление лидерства в международных отношениях трёх стран: Великобритания, Франция и США.
    \item Создание Лиги Наций (Регулировала отношения между странами и поддерживала стабильность международных отношений)
\end{enumerate}

\subsection{Отношение с западными странами (нет признания, требуют выплату долгов).}

\subsubsection{\textbf{Не признавание РСФСР другими странами.}}

Сложившаяся ситуация оказалась весьма угнетающей для Германии и РСФСР. Для последней еще было важно международное признание. Но большинство стран до конца гражданской войны этого не делали. В 1922 году проходит Генуэсская конференция из 19 стран, в том числе РСФСР под руководством Чичерина. Но западные державы не хотят признавать РСФСР и требуют выплатить долги царского и временного правительства. Кроме того, требовали возвращение собственности, принадлежавшей иностранцам (или компенсации ущерба). 

Западные страны заинтересованы в торговой деятельности, но при этом без выплаты долгов такое невозможно. Советское правительство выставляет контр-требования, требуя репараций за интервенцию в годы гражданской войны. 

\subsubsection{\textbf{Рапальское соглашение с Германией.}}

Германию также не признают, несмотря на все попытки поменять свое положение. Рядом с Генуэ в Рапауле(в методичке Рапалло) проходит частная встреча России и Германии. Заключается Рапальское соглашение, по которому обе страны признают друг друга, восстанавливают дипломатические отношения и отказываются от требований друг к другу. Заключается торговый договор, а затем в 1926 году договор о ненападении. Кроме того, РСФСР получает крупный заём. Этот разрыв дипломатической блокады облегчил процесс признания РСФСР другими странами. 

\subsubsection{\textbf{Восстановление дипломатических отношений}}

Франция и Великобритания после выхода из войны имели большие экономические проблемы, требовался выход и установление торговых отношений. В 1924 году Великобритания, Франция, Италия и другие восстанавливают дипломатические отношения. В 1925 году – Япония. Лишь США отказались восстанавливать дипломатические отношения.

\subsubsection{\textbf{Помощь Китайской партии Гомийдан(Чан"=Ганьши). Отказ Китая от отношений с СССР}}

Советский союз начал проводить дипломатическую политику в адрес восточных стран (Монголия, Китай Афганистан). Особое значение для Советского Союза приобрели отношения с Китаем. Там царствовала партия Гомийдан (Чан-Ганьши). Советский союз помогает прийти к власти Чан-Ганьши. При поддержке Советского Союза в китае создаётся коммунистическая партия. Чан-Ганьши понимает, что коммунистическая партия – его конкурент. В 1927 году Китай отказывается от отношений с Советским союзом.

Более успешно развивались контакты с другими восточными странами (Иран, Афганистан, Турция). 

\subsection{СССР и Коминтерн.}

В 20"=е годы Советский Союз всё еще пытался приблизить коммунистическую революцию. Создаётся коминтерн в 1919 году. Советский союз начинает поддерживать коммунистические партии в других странах. 

\subsection{Военная тревога 1927 года и ее роль в определении советского внешнеполитического.}

К концу 20"=х годов эта идея угасает, надежд всё меньше. Сталин провозглашает курс на построение социализма в отдельно взятой стране. С одной стороны в официальных выступлениях руководителей Советского государства звучат мысли о том, что нужно избежать вовлечения в военный конфликт с другими странами, к которому мы еще не готовы. А с другой стороны, в тех же ситуациях лидеры компартии намекают на возможность применения силы в отношении различных государств.

\textbf{Показательна ситуация 1927 года, так называемая «военная тревога».} Охлаждение отношений с Великобританией достигло пика из-за обнаружение сетей коминтерна. Еще и китай разрывает отношения. Возникает угроза войны с Великобританией, может быть, с Китаем и западными соседями (Эстония, Латвия, Литва, Румыния, Финляндия). 

Внутри советского союза начинается нагнетание военной истерии (вооружаться, готовиться к войне). Разворачивается борьба против бывших врагов народа, против тех, кто являлся выходцами из дворян и буржуазии. Происходит внутрипартийная борьба.

Партия ставит цель усиления боевой мощи армии. Начинается подготовка населения к потенциальной войне. Но постепенно тревога угасает, войны не случилось. 

\textbf{Внешняя политика разделилась на три этапа:}

\begin{enumerate}
    \item 1928--1933 гг. Советский Союз наиболее тесно сотрудничает с Германией. Это важно для советской политики, как противостояние другим западным странам. А на востоке активизируется советская внешняя политика в афганистане и Иране
    \item 1933--1939 гг. Советский союз переходит от сотрудничества с Германией	 к соперничеству. Сближение происходит с Британией, Францией, США.Начинается активная конфронтация с Японией.
    \item 1939--1941 гг. Происходит Сближение с Германией и Японией.
\end{enumerate}

\subsection{Действия на востоке}

В 1931 году Китай и Советский Союз улучшают отношения и-за общего врага – Японии. Она провозгласила курс на создание великой Японской империи, включавшую территории других стран. В 1931 году она начинает агрессию в Китае, захватывая Манчьжурию. Это нарушает интересы и Китая, и Советского союза, так как Манчжурия – плацдарм для вторжения в СССР. 

В 1930-х годах и Китай, и СССР пытались подтолкнуть друг друга к прямому столкновению. В 1939 году, так как Япония к тому моменту захватила большую часть Китая. СССР начинает поставки вооружения в Китай. В 1938 году около озера Хасан происходит Советско-Японское столкновение. В 1939 году на границе манчжурии и Монголии – на реке Халхин-гол. Отражал столкновение в последнем случае Жуков.

До 1939 года РСФСР оказывает поддержку Китаю в борьбе с Японией. А в апреле 1941 года СССР заключает договор с Японией для урегулирования взаимных отношений, а после этого прекращает поддержку Китая окончательно.

\subsection{Действия на западе}

На западе до военных столкновений не доходило, но отношения с Великобританией и Францией были не всегда стабильными. В 1930 году франция обнаруживает поддержку СССР коммунистическим партиям внутри страны. Но уже тогда понимают, что главный противник – не СССР, а усиливающаяся германия. В 1932 году СССР заключает договоры о ненападании с Францией, Финляндией, Латвией, Эстонией.

США долго не признавали СССР. Несмотря на это, торговые отношения ведутся.

\subsubsection{\textbf{Приход Гитлера к власти, смена направления отношений СССР.}}

До 1933 года в Германии стабильные отношения с СССР. Но потом, к власти приходит партия национал-социалистов под руководством А.Гитлера. Они отстаивали философию возрождения Германии. «У немцев отобрали заслуженную победу,» "--- говорил Гитлер. Это сделали коммунисты и, в первую очередь, евреи. С этим нужно было покончить, чтобы возродить Германию. К этому стали добавляться расизм и национализм в радикальных проявлениях. С этого момента Гитлер активно предпринимает уничтожение коммунистов внутри страны. 

Это вызвало смену направления отношений СССР. Коминтерн отмечал, что Германия "--- главный разжигатель войны в Европе. Теперь внешняя политика СССР носит антигерманский (и антиитальянский, с приходом к власти Муссолини). 

\subsubsection{\textbf{Попытка создать коллективную оборону против Германии. Политика <<умиротворения>> Гитлера}}

СССР стараются сделать коллективную оборону против германии. В 1933 году СССР признает США, а в 1934 году СССР принимается в Лигу Наций. В 1935 договор о взаимопомощи заключается с Францией и Чехословакией.

Но полного понимания между СССР и западными странами нет. На защиту от Германии страны отвечают уклончиво. Великобритания предлагала политику «умиротворения» Гитлера, т.е. давать его требования мелкими кусочками. Впоследствие стало понятно, что такая политика неудачная. 

\subsubsection{\textbf{Немного про гражданскую войну в Испании(в методичке написанно что-то про помощь СССР Испании, но инфы нет)}}

Разворачивается гражданская война в Испании, где борятся правительство из коммунистов и право-монархическое направление. В этой гражданской войне страны пытаются помочь выгодным себе сторонам. Эта война продолжается до 1939 года, республиканское правительство пало.

\subsubsection{\textbf{Присоединение Австрии Гитлером, Мюнхенский сговор (еще одна попытка <<умиротворения>>)}}

Гитлер уже в 1938 году осуществил аншлюз (присоединение) Австрии. Это акт нарушения международных отношений, но западные страны продолжают политику «умиротворения» и пытаются направить гитлера на восток. В 1938 году происходит встреча в Мюнхене, где присутствуют Италия, Англия, Франция. Они дают право Гитлеру присоединить Судетскую область (часть Чехии). Захватив Судеты, он надавил на Чехословакию и в 1938 году она переходит под «протекторат» Германии, однако фактически она была присоединена.

\subsubsection{\textbf{Пакт о ненападении 1939 года между Германией и СССР(пакт Молотова"=Риббентропа)}}

Этот Мюнхенский сговор произвел на Сталина пренеприятнейшие влияние. Расширение Германии угрожало СССР. Сталин действует на опережение и зондирует почву на восстановление дипломатических отношений с Германией. В результате секретных переговоров 23 августа 1939 года в Москве подписывается пакт о ненападении между Германией и СССР (известный также как пакт Молотова-Риббентропа). Не все пункты договора были опубликованы. К нему был составлен секретный договор, по которому две страны разделили сферы влияния в Европе. По нему Литва и западная часть Польши признавались в сфере интересов Германии, а восточная часть Польши, Эстония, Латвия и Финляндия признавались в интересах СССР. Когда этот пакт был подписан, Великобритания и Франция поняли, что происходит сближение СССР и Германии и разорвали свои дипломатические отношения с СССР.

