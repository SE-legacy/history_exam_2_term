\section{Билет 28. Политическое развитие России в 1990-е гг. Б.Н. Ельцин, Чеченская война.}

\subsection{Президент Ельцин и его окружение}
\subsection{Формирование и развитие новой политической системы}
Время с конца августа 1991 до декабря 1993 г. было особым периодом политической истории России, который называют «августовской республикой». 1992–1993 гг. стали началом: в экономике — смены форм и способов перераспределения собственности, уходом государства из сферы прямого управления экономикой; в социальной — демонтажа социальных институтов и гарантий, созданных за годы Советской власти; в идеологической — агрессивной либеральной экспансии; в политической - окончательный раскол правящей группы и борьба за власть между её осколками. Внешне он принял форму столкновения двух моделей её организации: президентской и парламентской республики.

\subsection{Политический кризис в верхах}
Собравшийся 10 марта 1993 г. VIII съезд народных депутатов отказался от ряда компромиссных соглашений с президентом. В ответ на это президент выступил 20 марта с Обращением к гражданам России, в котором сообщил, что подписал Указ об особом порядке управления страной до преодоления кризиса.

Указом презиодента 12 мая 1993 г. создавалось Конституционное совещание, которое 5 июня начало свою работу. На нём были представлены все уровни российской власти, все её ветви, Церковь, Академия наук и различные общественные организации. К концу июня 1993 г. существовали три различных проекта конституции: «президентский», «парламентский» и «советский». 12 июля 1993 г. в руках президента находился доработанный совещанием проект.

На совещании выявилось, что на конфликт между исполнительной и представительной властями накладывались и другие — между Центром и провинцией, а также между субъектами разного статуса.

После получения от Конституционного совещания 12 июля 1993 г. «нужного» Президенту проекта Конституции стало неободимо продавить его принятие. В августе напряжение в политической жизни нарастало с каждым днём. 10 августа Ельцин, прервав отпуск, вернулся в Москву и в тот же день на заседании Президентского совета в узком составе прямо заявил, что «нерешенность вопроса о Конституции выводит нас на силовые методы».

\subsection{Политический кризис 1993 года}
Наконец, 21 сентября был оглашён указ Ельцина №1400 «О поэтапной конституционной реформе в Российской Федерации». Президент вактически распустил Верховный Совет и съезд народных депутатов, а Конституционному суду запретил созывать заседания до начала работы нового парламента. Выборы в Государственную Думу назначались на 11–12 декабря 1993 г. К президенту переходило право назначения Генерального прокурора. Всё это означало, что фактически произошла узурпация власти. Установлена была информационная блокада Белого дома.

Собравшийся на экстренное заседание Конституционный суд уже к полуночи признал Указ №1400 незаконным. Собравшийся на экстренное заседание Верховный Совет, опираясь на решение Конституционного суда, оценил поступок президента как государственный переворот. В связи с этим полномочия президента Ельцина с 20.00 часов 21 сентября, а президентские обязанности надлежало исполнять вице-президенту А. В. Руцкому.

Вечером 23 сентября начал работу X (чрезвычайный) съезд народных депутатов.

Дважды — 22 и 29 сентября — к конфликтующим сторонам с призывом не допустить кровопролития обращался Патриарх Алексий II. 30 сентября, по инициативе и посредничестве Патриарха, в Свято-Даниловом монастыре в Москве была предпринята попытка примирить конфликтующие стороны. Переговоры провалились. 4 октября начался штурм Белого дома, был обстрелян из танков. К вечеру здание было захвачено, а руководство парламента и его защитники препровождены в тюрьму «Лефортово».

Случившееся в сентябре — октябре 1993 г. историки расценивают как ограниченный во времени и пространстве эпизод Гражданской войны, который, к счастью, ограничился верхушечной борьбой за власть в столице. Он не распространился на другие регионы страны и не привел к масштабному столкновению тех социальных сил, интересы которых объективно представляли Президент и Верховный Совет.

\subsection{Принятие новой конституции}

До середины декабря 1993 г. Россия оставалась без высшей представительной и законодательной властей. Многие важные стороны ее жизни регулировались президентскими указами. Б. Н. Ельцин воспользовался сложившейся ситуацией, чтобы в правовом плане закрепить достигнутую в начале октября победу. Распускались советы, учреждались новые органы местного самоуправления.

В октябре 1993 г. скорректированы условия проведения выборов в парламент и принятия новой Конституции. Указ от 11 октября «О выборах в Совет Федерации Федерального Собрания Российской Федерации» также вносил существенные изменения в Указ № 1400. В «ноябрьском» проекте Конституции права Президента были расширены даже в сравнении с его же предложениями весны–лета 1993 г.

\subsection{Основные партии и движения 90х, их лидеры и платформы}
Центральная избирательная комиссия зарегистрировала списки 13 партий и движений, собравших необходимое для участия в выборах количество подписей избирателей. Блок «Выбор России» объединил сторонников продолжения радикальных либеральных реформ.

На роль основной оппозиционной политической силы выдвинулась Коммунистическая партия Российской Федерации, воссозданная в феврале 1993 г. во главе с Г. А. Зюгановым. На роль третьей силы, выступавшей с государственнических, национально-патриотических позиций, претендовала Либерально-демократическая партия России (ЛДПР) во главе с В. В. Жириновским. Она выступала за возрождение Российского государства в границах СССР, сильную президентскую республику с регулируемой и социально ориентированной рыночной экономикой. В избирательной кампании остро ставила проблемы армии, защиты прав русскоязычного населения в республиках бывшего Союза ССР, положения беженцев из зон межэтнических конфликтов. Известность ЛДПР была во многом связана с личностью ее лидера В. В. Жириновского, который проявил себя как яркий оратор, способный своимиадресными обращениями привлекать симпатии достаточно широкого числа, в первую очередь «обездоленных», избирателей.

\subsection{Чеченская война. Хасавюртские соглашения}
В 1992 г. все дальше от правового поля Федерации отходила Чеченская республика, превращаясь в особую зону России. На ее территории осуществлялись беспошлинный ввоз и вывоз товаров, нелегальная торговля оружием, финансовые спекуляции. Регион стал крупным производителем и перевалочным пунктом торговли наркотиками; вступил в полосу острого социально4экономического кризиса. Стремительно шла криминализация чеченского общества. С конца 1991 г. начался захват военных объектов и складов с вооружением. К маю 1992 г. в распоряжении дудаевцев оказалось 80% боевой техники и 75% стрелкового оружия, ранее принадлежавших Советской Армии. К июню численность регулярных войск республики достигла 15 тыс. человек. Камнем преткновения на переговорах между Москвой и Грозным оставался вопрос о статусе Чечни: чеченская сторона настаивала на признании независимости республики

29 ноября 1994 г. Россия предъявила ультиматум Грозному с требованием немедленно разоружить все «нелегальные вооружённые формирования», который сепаратисты отказались удовлетворить. 10 декабря 1994 г. российские войска вошли на территорию Чечни. На момент начала полномасштабной вооружённой операции группировка федеральных сил насчитывала 23,8 тыс. человек, а в составе незаконных вооружённых формирований, с учётом добровольцев и наёмников, было до 13 тыс. человек.

В результате первой чеченской кампании (11 декабря 1994 — 31 августа 1996 г.) погибло десятки тысяч человек с обеих сторон (особенно кровопролитным оказался штурм Грозного в канун нового 1995 года), что стало следствием неподготовленности армии к ведению широкомасштабной войны, ошибок в управлении и излишней самоуверенности высшего военного руководства во главе с тогдашним министром обороны П. Грачёвым. 

Подписание Хасавюртовских мирных соглашений (31 августа 1996 г.) временно прекратило активные боевые действия, но не решило вопроса о статусе Чечни, который откладывался на 5 лет. До этого времени контакты с самопровозглашённой властью Чечни регулировали «Принципы определения основ взаимоотношений между Российской Федерацией и Чеченской Республикой». Последствием чеченской войны стала волна терактов, прокатившаяся по городам России в 1995–1999 гг. (Будённовск, Кизляр, Владикавказ, Буйнакск, Волгодонск и др.). Особо резонансными были взрывы жилых домов в Москве на ул. Гурьянова и на Каширском шоссе в сентябре 1999 г. Ответом на действия террористов стало повторное введение федеральных сил в Чечню 30 сентября 1999 г.

Хасавьюртовские соглашения вызвали неоднозначную оценку. Чеченцы трактовали их как закрепление своей победы в войне и суверенитета, поскольку «враг» покидал территорию республики. Генерал А. И. Лебедь усматривал успех в том, что удалось остановить кровопролитие и открывалась возможность вновь перевести урегулирование конфликта в политическое русло.

\subsection{Второе президентство Ельцина и его отставка}
В 1996–1999 гг. существенным фактором российской политики стало состояние здоровья Президента. Известие о своем повторном избрании Ельцин встретил на больничной койке. Осенью 1996 г. он перенес тяжелую операцию на сердце, а зимой 1996–1997 гг. боролся с простудным заболеванием.

Финансовый обвал 17 августа 1998 г. привел не только к падению правительства. 21 августа на внеочередном заседании Думы 248 депутатов призвали Президента добровольно уйти в отставку — в его поддержку выступили лишь 32 законодателя.

Так же был проведён импичмент, в ходе которого проводилось голосование в поддержку 5-ти пунктов признания Ельцина виновным: 

\begin{enumerate}[H]
  \item в подписании Беловежских соглашений и развале СССР (240 депутатов)
  \item в трагических событиях осени 1993 г. (263 депутатов)
  \item в развязывании войны в Чечне (283 депутатов)
  \item в развале армии (241 депутатов)
  \item в проведение политики геноцида против российского населения (238 депутатов)
\end{enumerate}

Импичмент требовал хотя бы 300 голосов по одному из пунктов. Идея не удалась но показала, что практически 3/4 депутатов выразили политическое недоверие президенту

Уже тогда многие политологи считали, что с весны 1999 г. в центре внимания Б. Н. Ельцина была именно проблема поиска преемника. Его имя названо Ельциным 9 августа 1999 г. после подписания ука4 за о назначении В. В. Путина и.о. премьер4министра, чье вступление в должность совпало по времени с началом крупномасштабной операции против чеченских боевиков

{\it спился дед...}
