\section{Билет 28. Политическое развитие России в 1990-е гг. Б.Н. Ельцин, Чеченская война.}
Время с конца августа 1991 до декабря 1993 г. было особым периодом политической истории России, который называют «августовской республикой». 1992–1993 гг. стали началом: в экономике — смены форм и способов перераспределения собственности, уходом государства из сферы прямого управления экономикой; в социальной — демонтажа социальных институтов и гарантий, созданных за годы Советской власти; в идеологической — агрессивной либеральной экспансии; в политической - окончательный раскол правящей группы и борьба за власть между её осколками. Внешне он принял форму столкновения двух моделей её организации: президентской и парламентской республики.

Собравшийся 10 марта 1993 г. VIII съезд народных депутатов отказался от ряда компромиссных соглашений с президентом. В ответ на это президент выступил 20 марта с Обращением к гражданам России, в котором сообщил, что подписал Указ об особом порядке управления страной до преодоления кризиса.

Указом презиодента 12 мая 1993 г. создавалось Конституционное совещание, которое 5 июня начало свою работу. На нём были представлены все уровни российской власти, все её ветви, Церковь, Академия наук и различные общественные организации. К концу июня 1993 г. существовали три различных проекта конституции: «президентский», «парламентский» и «советский». 12 июля 1993 г. в руках президента находился доработанный совещанием проект.

На совещании выявилось, что на конфликт между исполнительной и представительной властями накладывались и другие — между Центром и провинцией, а также между субъектами разного статуса.

После получения от Конституционного совещания 12 июля 1993 г. «нужного» Президенту проекта Конституции стало неободимо продавить его принятие. В августе напряжение в политической жизни нарастало с каждым днём. 10 августа Ельцин, прервав отпуск, вернулся в Москву и в тот же день на заседании Президентского совета в узком составе прямо заявил, что «нерешенность вопроса о Конституции выводит нас на силовые методы».

Наконец, 21 сентября был оглашён указ Ельцина №1400 «О поэтапной конституционной реформе в Российской Федерации». Президент вактически распустил Верховный Совет и съезд народных депутатов, а Конституционному суду запретил созывать заседания до начала работы нового парламента. Выборы в Государственную Думу назначались на 11–12 декабря 1993 г. К президенту переходило право назначения Генерального прокурора. Всё это означало, что фактически произошла узурпация власти. Установлена была информационная блокада Белого дома.

Собравшийся на экстренное заседание Конституционный суд уже к полуночи признал Указ №1400 незаконным. Собравшийся на экстренное заседание Верховный Совет, опираясь на решение Конституционного суда, оценил поступок президента как государственный переворот. В связи с этим полномочия президента Ельцина с 20.00 часов 21 сентября, а президентские обязанности надлежало исполнять вице-президенту А. В. Руцкому.

Вечером 23 сентября начал работу X (чрезвычайный) съезд народных депутатов.

Дважды — 22 и 29 сентября — к конфликтующим сторонам с призывом не допустить кровопролития обращался Патриарх Алексий II. 30 сентября, по инициативе и посредничестве Патриарха, в Свято-Даниловом монастыре в Москве была предпринята попытка примирить конфликтующие стороны. Переговоры провалились. 4 октября начался штурм Белого дома, был обстрелян из танков. К вечеру здание было захвачено, а руководство парламента и его защитники препровождены в тюрьму «Лефортово».

Случившееся в сентябре — октябре 1993 г. историки расценивают как ограниченный во времени и пространстве эпизод Гражданской войны, который, к счастью, ограничился верхушечной борьбой за власть в столице. Он не распространился на другие регионы страны и не привел к масштабному столкновению тех социальных сил, интересы которых объективно представляли Президент и Верховный Совет.

До середины декабря 1993 г. Россия оставалась без высшей представительной и законодательной властей. Многие важные стороны ее жизни регулировались президентскими указами. Б. Н. Ельцин воспользовался сложившейся ситуацией, чтобы в правовом плане закрепить достигнутую в начале октября победу. Распускались советы, учреждались новые органы местного самоуправления.

В октябре 1993 г. скорректированы условия проведения выборов в парламент и принятия новой Конституции. Указ от 11 октября «О выборах в Совет Федерации Федерального Собрания Российской Федерации» также вносил существенные изменения в Указ № 1400. В «ноябрьском» проекте Конституции права Президента были расширены даже в сравнении с его же предложениями весны–лета 1993 г.
\subsection{Основные политические партии и движения}
Центральная избирательная комиссия зарегистрировала списки 13 партий и движений, собравших необходимое для участия в выборах количество подписей избирателей. Блок «Выбор России» объединил сторонников продолжения радикальных либеральных реформ.

На роль основной оппозиционной политической силы выдвинулась Коммунистическая партия Российской Федерации, воссозданная в феврале 1993 г. во главе с Г. А. Зюгановым. На роль третьей силы, выступавшей с государственнических, национально-патриотических позиций, претендовала Либерально-демократическая партия России (ЛДПР). Она выступала за возрождение Российского государства в границах СССР, сильную президентскую республику с регулируемой и социально ориентированной рыночной экономикой.
