\section{\textbf{Проблемный вопрос} Билет 30. Внешняя политика Российской Федерации в 1990-е гг.}
\subsection{Российско-американские отношения после окончания Холодной войны, СНВ-2}
В 90-е годы распадающийся Союз Советских Социалистических Республик (СССР), а впоследствии и Российская Федерация (РФ), чрезвычайно активно предпринимали попытки наладить мирные и доброжелательные отношения с Соединёнными Штатами Америки (США), приветствуя стремительное распространение политического и экономического влияния Запада на пространстве Восточного Блока и идя на взаимные уступки. В этот период было подписано множество международных договоров, среди которых:
\begin{itemize}
    \item «Договор об обычных вооружённых силах в Европе», устанавливающий ограничения на типы и количество вооружений, размещённых восточными и западными странами между Атлантическим океаном и Уральскими горами;
    \item «Договор о сокращении стратегических наступательных вооружений», ограничивающий ядерные, орбитальные и баллистические арсеналы СССР и США;
    \item Одноимённый договор между Российской Федерацией и США.
\end{itemize}
В 1994 году Россия наряду с другими постсоветскими государствами начала участвовать в программе Организации североатлантического договора (НАТО) «Партнёрство ради мира». 22 июня 1994 года её подписал министр иностранных дел Андрей Козырев. Это событие стало началом процесса составления соглашения между НАТО и непосредственно Россией, ставшего впоследствии известным как «Основополагающий акт о взаимных отношениях, сотрудничестве и безопасности между Российской Федерацией и Организацией североатлантического договора». Церемония его подписания состоялась в Париже 27 мая 1997 года.
  В соответствии с текстом акта:
\begin{enumerate}
  \item Россия и НАТО не рассматривают друг друга в качестве противников
  \item Стороны согласны применять силу только с санкции Организации Объединённых Наций (ООН)
  \item Участники соглашения намерены развивать Организацию по безопасности и сотрудничеству в Европе (ОБСЕ), призванную укрепить общеевропейскую безопасность[5].
  \item Стороны согласны обмениваться друг с другом информацией по военным доктринам, бюджетам и военным стратегиям
  \item Россия и Организация североатлантического договора готовы сотрудничать в вопросах нераспространения оружия массового поражения и тактической противоракетной обороны (ПРО), обеспечения безопасности воздушного движения и борьбы с наркотрафиком.
  \item НАТО не разместит ядерное оружие на территории новых членов и не будет строить места его хранения
\end{enumerate}
  Представленная информация позволяет отметить, что Россия и НАТО обязывались вести активное сотрудничество по ряду вопросов, касающихся, в первую очередь, безопасности, а также шли на значительные взаимные уступки, ограничивая свои вооружения и их присутствие в европейском регионе.
\subsection{Россия и Европейский союз}
24 июня 1994 года было подписано двустороннее соглашение о партнёрстве и сотрудничестве между Европейским союзом и Россией (вступило в силу 1 декабря 1997 года). Первое заседание Совета сотрудничества ЕС—Россия состоялось в Лондоне 27 января 1998 года.
\subsection{Заключение вывода российских войск из Европы. Начало расширения НАТО на восток}
12 февраля 1990 года было положено начало переговорам между Организацией североатлантического договора и Организацией Варшавского договора по вопросу объединения Германии в формате "2+4", когда в Оттаве состоялась конференция минисиров иностранных дел государств, членствующих в данных военных блоках. На этой конференции Советский Союз получил гарантии нерасширения НАТО восточнее реки Одер. Данный факт подтверждается в протоколе встречи политических директоров министерств иностранных дел США, Великобритании, Франции и Германии, состоявшейся в Бонне 6 марта 1991 года[14].

В ноябре 1993 года была провозглашена программа "Партнёрство во имя мира", которая, со слов государственного секретаря США Уоррена Кристофера "открыла двери для постепенного расширения числа членов НАТО". Тем не менее, президенту России Б. Н. Ельцину было дано заверение, что НАТО не станет включать в свой состав всю Восточную Европу, которое У. Кристофер позднее назвал неверной трактовкой своих слов. 

Однако в ноябре 1997 года был подписан и ратифицирован Протокол о вступлении Польши, Чехии и Венгрии в Организацию североатлантического договора. 12 марта 1999 года эти наиболее экономически развитые страны-члены бывшей ОВД стали полноправными членами НАТО.

Учитывая то, что в восточноевропейском регионе (не считая гражданскую войну в Югославии) царила абсолютно безопасная обстановка, единственным объяснением расширения НАТО российская сторона увидела угрозу, связанную с целенаправленным движением к границам России. Позднее, 29 марта 2004 года, Россия окончательно убедится в этом, когда в рамках шестого расширения в состав НАТО войдут Болгария, Румыния, Словения и Словакия, бывшие участницы ОВД, а также Латвия, Литва и Эстония, входившие в состав СССР.

На основании данных событий можно сделать вывод, что, поскольку приближение границ НАТО к российским границам представляло потенциальную угрозу для России, расширения 1999 и 2004 гг. стали одной из болезненных тем в отношениях между РФ и странами Организации североатлантического договора.
\subsection{Вступление России в G8 и Совет Европы}
С 1996 года, после встречи в Москве, Россия начала всё активнее принимать участие в работе объединения G7, а с 1997 года участвовала в его работе на равных с другими участниками объединения, ставшего после этого группой восьми («Большой восьмёркой»).

Российская Федерация вступила в Совет Европы 28 февраля 1996 года, а 30 марта 1998 года ей была ратифицирована Европейская Конвенция о защите прав человека и основных свобод. Тем самым Россия подтвердила свою приверженность идеалам и принципам гуманизма и демократии, а также готовность скорректировать целый ряд законодательных актов, противоречащих положениям Конвенции. Присоединяясь к СЕ, Россия заверила организацию в том, что она готова привести своё законодательство и политическую систему в соответствие с европейскими порядками.
\subsection{Распад Югославии. Бомбардировки Югославии силами НАТО в 1999 году}
В ночь с 23 на 24 марта 1999 года генеральный секретарь НАТО Франсиско Хавьер Солана де Мадариага отдал приказ командующему силами НАТО в Европе американскому генералу Уэсли Кларку начать военную операцию против Югославии под кодовым названием «Союзная сила».

Распространившаяся об отдаче данного приказа информация вызвала значительный резонанс и вызвала возмущение Российской Федерации, дружественно настроенной к Югославии. Кроме того, это нападение не было одобрено Советом Безопасности Организации Объединённых Наций, что нарушало не только Устав ООН, но и подписанный раннее «Основополагающий акт о взаимных отношениях, сотрудничестве и безопасности между Российской Федерацией и Организацией североатлантического договора». 

В это время премьер-министр России Евгений Максимович Примаков на самолёте направлялся в Вашингтон с крупной делегацией, включавшей в себя большую группу губернаторов и бизнесменов, для встречи с 45-м вице-премьером США Альбертом Гором-младшм. Целью данного визита было заключение кредитных договоров на 15 миллионов долларов США, необходимых для восстановления российской экономики. Пролетая над Атлантическим океаном, Е. М. Примаков узнал о решении начать бомбардировки Югославии во время телефонного разговора с А. Гором. В связи с этим, глава российского правительства принял решение отменить визит и экстренно вернуться в Москву. По прибытии в аэропорт Внуково, Евгений Примаков сделал ряд жёстких заявлений, осуждающих готовящееся нападение НАТО на Югославию.

В этот же день президент Российской Федерации Б. Н. Ельцин провёл переговоры с главой Франции Жаком Шираком и с президентом Соединённых Штатов Америки Биллом Клинтоном[7]. Готовящиеся бомбардировки он назвал «ударом по всему международному сообществу».

Вторжение НАТО на Балканы было воспринято как акт агрессии в том числе против России и привело к постепенному прекращению сотрудничества с Организацией североатлантического договора. Уже 25 марта 1999 года МИД РФ принял решение о заморозке отношений между Российской Федерацией и НАТО[13].

По итогам бомбардировок Югославии, 9 июня 1999 года в Куманове было подписано Кумановское военно-техническое соглашение, в соответствии с которым Организация североатлантического договора прекращала бомбардировки на условиях вывода югославских сил из Косова и размещения там миротворческого контингента НАТО «Kosovo Force» (KFOR)[10]. России было предложено принять участие в миссии KFOR, но Российская Федерация требовала для своих миротворцев независимый сектор в северной части Косово.

В связи с тем, что НАТО отклонила данное требование, верховный главнокомандующий Б. Н. Ельцин отдал приказ 22-му отдельному парашютно-десантному полку воздушно-десантных войск Российской Федерации совершить марш-бросок на Приштину, столицу данного края, под командованием Виктора Михайловича Заварзина. Данная военная операция стала первой в истории российско-американских отношений демонстрацией решимости и силы России[11].

Исходя из вышеизложенного, становится очевидным, почему вторжение НАТО в Югославию вопреки принципам международного права и договорённостям между Россией и США принято считать переломным моментом в истории российско-американских отношений.
\subsection{Россия и страны СНГ}
Понимая исключительную важность коллективной обороны, главы государств Содружества еще 15 мая 1992 г. приняли решение о создании военно-политического союза, подписав Договор о коллективной безопасности. Стратегический курс взаимодействия России со странами СНГ был утверждён Указом президента РФ от 14 сентября 1995 г.

Приоритетность отношений со странами СНГ в политике России определяется прежде всего тем, что на территории СНГ сосредоточены наши главные жизненные интересы в области экономики, обороны, безопасности, защиты прав россиян.

Отношения в СНГ осложнялись ситуаций вокруг непризнанных («самопровозглашенных») республик: Абхазии, Южной Осетии, Приднестровья и Нагорного Карабаха.

Гражданская война в Приднестровье требовала вмешательства теперь уже бывших центральных властей в лице России. 7 июля 1992 года Россией и Молдавией был подписан план мирного разрешения конфликта; затем, 21 июля было заключено соглашение о мирном урегулировании, с которым согласилась приднестровская сторона; 29 июля в Бендеры и Дубоссары были введены миротворческие силы России, а 1 августа 1992 года было завершено разведение вооружённых формирований конфликтующих сторон. Таким образом при посреднечестве России в Приднестровье настал мир, гарантом сохранности которого являются размещённые там российские войска.

14 июля 1992 года в зону конфликта в Южной Осетии вошли миротворческие силы — три батальона. Российский, грузинский и осетинский. Грузия утратила контроль над Южной Осетией. Южная Осетия, не желающая входить в состав Грузии, провозгласила курс на сближение с Россией, многие жители непризнанной республики получили российское гражданство.
