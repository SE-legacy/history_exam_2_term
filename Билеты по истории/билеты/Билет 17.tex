\section{Билет 17. Ключевой этап Великой Отечественной войны (июль 1942 — декабрь 1943 г.)}

Данный период войны считается самым тяжёлым и для страны, и для народа.
Противник оккупировал территорию, где проживало 13\% населения, производилось до 30\% ВВП, выращивалось до 50\% продовольствия.

Немецкие войска в этот период впервые за годы Второй Мировой претерпели поражения.
В СССР было организовано плановое военное хозяйство, выросло боевое мастерство советских вооружённых сил. Это создало условия для коренного перелома в ходе войны с Германией.

\subsection{Оборона Сталинграда и Кавказа}

Крупнейшие битвы - Сталинградская битва (17 июля 1942 года - 2 февраля 1943 года). На Сталинград ведут наступление 4-я танковая армия вермахта и армия под командованием Паулюса. Оборону ведут войска Донского фронта и Сталинградского фронта.
На флангах у Паулюса - венгерские, итальянские и румынские части. Это окажется одним из важных факторов Сталинградской битвы.

До середины ноября 1942 года войска Сталинградского и Донского войска не позволили войти врагу в Сталинград с ходу. Развёрнуты беспрецедентные по масштабам и кровопролитности сражения. Город был практически полностью уничтожен.

Параллельно ведётся наступление и на северный Кавказ, открывающий путь к месторождениям нефти - важнейшему ресурсу современной войны. Планы Германии по захвату Кавказа срываются советскими войсками, вынудившими врага перейти к обороне.

Было перехвачено превосходство над врагом в силах и средствах. Ставка определяет целью захват стратегической инициативы.

19 ноября 1942 года, перейдя в наступление у Сталинграда, войска Сталинградского и Донского фронтов охватывают вытянувшиеся к Волге части вермахта. Понимая, что на флангах у Паулюса менее боеспособные части, было решено прорвать их там. Так и получилось: 8-я и 4-я танковые армии общей численностью до 300 тысяч оказались в окружени. Гитлер выдвинул группу армий "Дон" под командованием Монштейна, но прорыв блокады был предотвращён. Взято в плен более 90 тысяч человек.

Разгром армии Паулюса положил начало коренному перелому в войне. Тем временем большая часть Кавказского региона была освобождена, а на севере был создан сухопутный коридор в Ленинград. До января 1943 года Ленинград был полностью блокирован, единственные два маршрута - воздух (всегда) и Ладожское озеро (зимой). Но это не снятие блокады.

\subsection{Советский тыл}

Весной 1943 года наступила стратегическая пауза и стороны наращивали ресурсы для решающего сражения. СССР наращивал вооружение и принимал новые образцы. В частности, тяжёлый танк КВ и средний танк Т-34. Появились истребители Ларг-3, Як-1 и Ил-2.

В составе советских войск образуются иностранные части. Например, польская дивизия имени Тадеуша Костюшко, французский авиаполк "Нормандия-Неман". Были и югославские части.

Германия и её сателлиты провели тотальную мобилизацию. Соотношение сил в июле 1943 года уже складывалось в пользу советских войск. Превосходство по числу техники было кратным.

В августе 1943 г. вышло постановление «О неотложных мерах по восстановлению хозяйства в районах, освобождённых от немецкой оккупации». Прежде всего необходимо было обеспечить освобождённые территории профессиональными кадрами. Для этого отправляли из эвакуации работников и демобилизовали из армии специалистов с высокой квалификацией, особенно в добывающей и металлургической промышленности. Затем вернули материальную часть: станки и оборудование, транспорт и сельскохозяйственную технику. 

\subsection{Советское правительство и Русская православная церковь.}

22 июня 1941 года митрополит Сергий выступил с обращением к верующим, призвав защитить Отечество от врага. Стали собираться средства на военные нужды, была создана танковая колонна имени Дмитрия Донского. Коллаборационисты отлучались от Церкви. Сталин разрешил избрать нового Патриарха, им стал митрополит Сергий. Был образован Священный синод, открыто несколько семинарий по подготовке священников. Веротерпимость не отменяла контроля над Церковью.

\subsection{Немецкий оккупационный режим}

Возникает оккупационная система управления. Было создано министерство оккупированных восточных территорий, ему подчинялось несколько рейхскомиссариатов --- Остланд (Прибалтика), Украина. Молдавия оказалась под управлением Румынии. Часть Карело-Финской ССР была оккупирована Финляндией.

Нацисты развивали коллаборационистское движение. Советские органы власти закрывались, общественные организации запрещались. В селе создавались комендатуры, в которых приглашали пронемецки настроенных селян. На селе функционировала полиция, в городах --- отряды СС. Они контролировали захваченные территории и осуществляли карательные акции против противников режима.

Вёлся строгий учёт местного населения, подлежавшего регистрации в полиции. Для местного режима вводилась трудовая повинность, невыполнение которой жестоко каралось. Местное население обеспечивало немецкую армию пищей. В Германию и некоторые другие страны было принудительно перевезено около 5 млн человек.

Коммунисты, евреи и цыгане отправлялись в концентрационные лагеря, где осуществлялись массовые убийства людей всех возрастов и любого пола.

\subsection{Партизанское движение}

На это советский народ отвечал борьбой в тылу врага. В течение 1943 года на оккупированных территориях сражалось до 250 тысяч человек. Партизаны проводили диверсионные операции. Наиболее известные --- рельсовая война и "Концерн". Партизаны устроили до 250 тысяч крушений поездов, взорвали множество мостов и часть военной техники противника.

\subsection{Курская битва (5 июля -- 23 августа 1943 года) Коренной перелом в войне}

Образовался Курский выступ --- дуга, про которой проходил фронт. Немецкое командование решило провести стратегическую операцию "Цитадель". Планировалось, согласно ей, разгромить советские войска, прорвав оборону с юга и севера. Добившись здесь успеха, развернуть наступление на Москву с юга. Для этого было сосредоточено около 50 дивизий численностью до 900 тысяч человек. Здесь было большое скопление техники противника.

Им противостояли войска Центрального и Воронежского фронтов, имеющие превосходство в танках, самоходных орудиях и даже самолётах. В тылу стоял Степной фронт, готовый поддержать оборону в тяжёлые моменты.

Ставка решила сначала обороняться в первую очередь до разгрома танковых и механизированных сил врага, а затем перейти в контрнаступление. Советская армия подготовила эшелонированную линию обороны глубиной до 300 километров.

Около недели советские войска упорно оборонялись. Немцы наступали с юга и с севера от Курска. Продолжались ожесточённые и кровавые бои с огромными потерями с обеих сторон. К 12 июля наступление выдохлось и советские войска начали наступать на Орловском и Белгородско-Харьковском направлениях.

Победа на Огненной дуге стала коренным переломом в ходе Великой Отечественной Войны. После победы под Курском советская армия удерживала инициативу до конца войны. Это способствовало и успеху союзников.

\subsection{Форсирование Днепра}

После поражения под Курском немецкое командование пыталось перевести войну на позиционные формы, но  войска Юго-западного и Южного фронтов освобождают Донбасс и к концу сентября выходят к Днепру.

Здесь 4 советских фронта переименовывают в 4 украинских фронта. С тяжёлыми боями они успешно форсируют Днепр. 6 ноября советские войска вступили в Киев. Освобождаются Запорожье и Днепропетровск, противник блокируется на Крымском полуострове.

В это же время на севере действуют Калининский, Западный и Брянский фронты, которые отбрасывают врага на  300 километров от Москвы. Начинается освобождение Белоруссии.

\subsection{Проблема открытия второго фронта в Европе. Тегеранская конференция}

Победы советских войск позволяют в корне переменить обстановку в северной Африке. Африка постепенно очищается от немецко-итальянских войск. Начинается вторжение в Италию. Американцы и Британцы осуществляют десант сначала на Сицилии, а потом и на Аппенинском полуострове. Это помогло свергнуть режим Муссолини. Новое итальянское правительство объявляет войну Германии и заключает мир с союзниками.

Активизировалось дипломатическое и военно-политическое сотрудничество США и Великобритании. Состоялась Тегеранская конференция (28 ноября -- 1 декабря 1943 г.). Здесь заседала большая тройка в лице Сталина, Черчилля и Рузвельта. Операция "Оверлоад" (высадка союзников в Нормандии) была намечена на май 1944 года. Обсуждались контуры послевоенного устройства мира. Черчилль и Рузвельт высказали мысль о разделе Германии, Сталин был скептически настроен к этой идее. Он боялся, что резкое ослабление Германии оставит СССР один-на-один с капиталистическими державами. Иосиф Виссарионович надеялся создать дружественную силу в Европе.
