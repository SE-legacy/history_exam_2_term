\section{Билет 1. Февральская революция 1917 г.: причины, основные события, результаты.}

\subsection{Причины:}

\begin{enumerate}
    \item Нерешённые социальные и экономические проблемы (аграрный вопрос, рабочий вопрос)
    \item Национальный вопрос (борьба разных наций, проживавших в РИ за национальную автономию, развитие национального языка. Например, Польша, страны Прибалтики, ближний восток и другие.)
    \item Тяготы войны. Война вела к снижению уровню жизни, дефициту, росту цен, к мобилизации, изымающей из экономических отношений взрослых здоровых мужчин – основную рабочую силу.
    \item Кризис власти. Николай II за годы своего правления потерял авторитет. Многие люди стали выражать волнения о влиянии Распутина на власть. Распутин – крестьянин, который был очень близко знаком с императорской семьёй. Многие считали, что он оказывал непосредственное влияние на политику страны
\end{enumerate}

\subsection{Стачечное движение в столице (посмотреть в доп. источниках).}

В январе-феврале 1917 года число бастовавших рабочих составило 670 тысяч человек, причём 90\% забастовок носили политический характер – о расширении избирательных прав, об учредительном собрании. В Петрограде тогда же происходит дефицит продовольствия, связанный с трудностями на железных дорогах из-за забастовок рабочих.

Середина февраля: нехватка хлеба и рост цен привели к забастовкам и введению карточной системы.

18 февраля: Начинаются протесты рабочих Путиловского завода.

20 февраля: Закрытие, из-за протестов и перебоев в снабжении, Путиловского завода.

Итог: Тысячи рабочих оказались на улице.

\subsection{События в Петрограде 23-26 февраля.}

\subsubsection{\textbf{23 февраля (тогда международный женский день):}}

На улицы стали выходить рабочие и работницы, мелкие служащие и студенты с лозунгами: «Хлеба!», «Долой войну!», «Долой самодержавие!».

Действия правительства: т.к. было решение, что протесты вызваны страхом голода, было решено вывесить объявления о наличие у города запасов зерна.

\subsubsection{\textbf{24 февраля:}}

Бастовали почти все заводы. Начали выходить рабочие с красными флагами и стали выдвигаться к центру Петрограда. Происходили столкновения с конной полицией.

\subsubsection{\textbf{25 февраля:}}

Начали активизироваться большевики.
Действия правительства:

\begin{enumerate}
    \item Генерал Хабалов отдал инструкции полиции не допускать прохода демонстрантов через Невские мосты, но шествия всё равно прошли.
    \item Вечером Николай II отправил Хабалову телеграмму с требованием остановить беспорядки. Это была вся реакция от монарха.
\end{enumerate}

\subsubsection{\textbf{Ночь на 25-26 февраля:}}

Охранка произвела многочисленные аресты, нелегальные оппозиционеры такого не ожидали и решили занять выжидательную позицию.

\subsubsection{\textbf{26 февраля:}}

С окраин к центру города снова двинулись толпы рабочих.

Солдаты, которых поставило в заслоны правительство, отказались стрелять по рабочим. В это время на пулемёты были поставлены офицеры. В итоге офицеры и полиция убили в тот день свыше 150 человек.

Гвардейцы Павловского полка решили поддержать рабочих и открыли огонь по полиции.

Председатель Думы Родзянко предупредил Николая II, что правительство парализовано и «в столице анархия». Родзянко предложил Николаю II пойти на уступки и создать «правительство общественного доверия».

Действия правительства: Император и совет министров ввели чрезвычайное положение и Николай II подписал указ о роспуске Думы. Но Дума проигнорировала это решение. Николай II послал из Ставки войска для подавления революции, но отряд генерала Иванова был задержан под Гатчиной восставшими железнодорожниками.

\subsection{Вооруженное восстание.}

\subsubsection{\textbf{Ночь на 26-27 февраля:}}

Солдаты некоторых лейб-гвардейских полков (Павловского, Волынского, Преображенского) взбунтовались против своих офицеров, которым они не могли простить приказа стрелять в рабочих.

\subsubsection{\textbf{27 февраля:}}

Революционный натиск рабочих совпал с движением солдат. Демонстранты начали братание с солдатами, после восставшие захватили арсенал (были сразу розданы 40 тыс. винтовок), общественные здания (Петропавловская крепость). Это ознаменовало победу революции. Но восставшие на этом не остановились и направились к Таврическому дворцу. Начались аресты царских министров и образование новых органов власти.

\subsection{Создание временного комитета Государственной думы и Петроградского совета рабочих и солдатских депутатов}

\subsubsection{\textbf{Образование совета:}}

27 февраля были проведены выборы в «Петроградский совет рабочих депутатов». Для руководства его деятельностью был избран Исполнительный комитет. Большинство мест заняли меньшевики.

Председатель: меньшевик Чхеидзе

Заместитель: Эсер А.Ф. Керенский

Исполнительный комитет взял на себя поддержание общественного порядка и снабжение населения продовольствием.

1 марта он пополняется солдатами и совет становится «Петроградским советом рабочих и солдатских депутатов».

\subsubsection{\textbf{Образование временного комитета:}}

27 февраля встревоженная образованием совета Государственная дума создаёт «Временный комитет государственной думы» для предотвращения беспорядков в Петрограде под управлением право-центристких партий (преобладали кадеты). 

\subsection{Отречение Николая 2.}

Николай II едет из Могилёва в Петроград, но состав останавливается в Пскове – дальше не пропускают. Это происходит 1 марта. 2 марта в Псков приезжает делегация от государственной думы во главе с представителем октябристов Гучковым и представителем монархической партии Шульгиным (монархисты многие также были недовольны Николаем II). Делегация считает, что лучшим решением будет отречься от престола. Николай II пытается выяснить, на чьей стороне армия. Он посылает телеграммы командующим фронтами, где спрашивает их отношение к ситуации. Все, кроме командующего черноморским флотом, отвечают, что лучшим решением будет отречение. В результате Николай II подписывает манифест об отречении (в том числе отрекается за своего сына) и передает власть своему брату. Тем же манифестом назначает Г. Львова новым главой правительства. Так как он относился к тому спектру взглядов, который был во временном комитете.

3 марта Михаил Александрович (Брат Николая 2) узнаёт о своём назначении на пост великого самодержца. Увидев манифест об отречении он понимает, что это не его выбор и сам отрекается от престола и возлагает выбор власть на учредительное собрание. Таким образом 3 марта 1917 года прекращает существование российское самодержавие (монархия).

\subsection{Временное правительство: состав}

Во временном правительстве первый министр"=предедатель "--- Львов, министром иностранных дел становится Милюков. Военным и морским министром – Гучков. Министром торговли и промышленности – Коновалов. Министром финансов – Терещенко. Министр Юстиции – Керенский. Всего из 12 членов временного правительства: 5 кадетов, 2 беспартийных и т.д.

\subsection{Приказ №1 1 марта 1917г}

1 марта 1917 года Петроградский совет издал приказ №1, обязавший низшие чины избрать себе командование из своих чисел и отстранивший офицеров и высшее командование от управления войсками.

\subsection{Историческая оценка февраля}

\begin{enumerate}
    \item Крах монархии, существовавшей тысячу лет. \\
    \item Формирование новых органов власти на местах \\
    \item Начало процесса переустройства страны
\end{enumerate}
