\section{Билет 24. Внешняя политика СССР в 1964—1985 гг.}
    \subsection{Внешнеполитические условия развития СССР после 1964 г.}
    
    Происходили существенные сдвиги. Угроза глобальной ядерной катастрофы требовала снижения международной напряженности. Начинается сближение СССР и США. СССР считали, что «завоевание» мира начнётся с колоний стран 3-го мира (колоний западных стран). Исходя из этой доктрины, советская внешняя политика развивалась по 3 направлениям:
    
    \begin{itemize}
        \item Социалистические страны. Максимально дружеские отношения и поддержка.
        \item Капиталистические страны. Противоборство
        \item Страны 3-го мира. СССР выстраивает отношения исходя из конкретной страны и её близости к социализму.
    \end{itemize}

    \subsection{Отношения с социалистическими странами}
    
    Социалистическим странам позволялись большие экономические свободы. Из-за этого к власти приходили всё более либеральные лидеры, которые пока еще не отказывались от социализма, но пытались его подстроить. СССР влиял на такие страны через совет экономической взаимопомощи. Страны стараются наладить экономические отношения с капиталистическими, так как это более выгодно. Однако отношения с советским союзом у них (социалистических стран) были не равноправные. Так, в 1968 году, в Чехословакии предотвращения либерализации Чехословакии СССР вводят не только советские войска, но и войска стран, входящих в ОВД. 
    
    С 1971 года принята программа углубления сотрудничества, которая обеспечивала страны дешевыми энергоносителями. Крупными проектами стало строительство нефтепроводов. Так, например, нефтепровод «Дружба» (СССР -> Польша -> ГДР, СССР -> Австрия, Венгрия, Чехословакия). Были сотрудничества и в космической программе (просто экипаж был из разных стран). СССР ЗАПУСТИЛ ПЕРВОГО МОНГОЛА В КОСМОС (надеюсь, он был согласен).
    
    Активно проводились действия в рамках Варшавского договора. Аналогичные чехословацким событиям, произошли события в Польше в 1980-х. Из-за кризиса, начавшегося в 1970-х годов, начались протест власти. Появился независимый профсоюз «Солидарность» и началось массовое забастовочное движение против социалистического режима. Это приводит к тому, что в 1981 году в Польше вводится военное положение.

    Наряду со странами, входившими в состав ОВД и СЭВ, были социалистические государства, не зависевшие от СССР. Например, Югославия.

    Советский Союз проводил по отношению к Югославии сдержанно"=доброжелательную политику. Похожие отношения сложились с Румынией. У руководства Румынии находился Чаушеску. Чаушеску был самостоятельный, но в Румынии действовала советская система и СССР её не трогал.
    
    Плохие отношения были с Албанией, которая негативно восприняла десталинизацию в Советском Союзе и увидела угрозу в потеплении между СССР и СФРЮ.
    
    Ещё более жёсткая конфронтация сложилась с Китайской Народной Республикой в результате оттепели. Мао Цзедун посчитал предательством отказ от сталинской модели социализма. Культурная революция в Китае, связанная с политикой жёсткого ограничения и чистки китайской культуры и интеллигенции, также отрицательно сказалась на советско-китайских отношениях. В 1965-1966 гг. из Китая стали уезжать советские граждане, прекратились экономические, культурные и политические контакты. Противостояние дошло до военных столкновений на границе. Такие столкновения произошли на острове Даманский на реке Уссури, а также в средней Азии (Семипалатинская область). Эти стычки не привели к масштабной войне, но вносили напряжённость в отношения двух стран. КНР рассматривалась как потенциальный противник до смерти Мао Цзедуна и Брежнева. Только после этого отношения стали улучшаться. В СССР началась Перестройка, в Китае "--- свои изменения.

    \subsection{Отношения с капиталистическими странами.}
    
    Отношения с капиталистическими странами носили противоречивый, но конструктивный характер. Началось потепление с Францией, которую возглавлять Шарль де Голль. Франция вышла из НАТО, это давало Советскому Союзу установить с Францией более тесные экономические связи. Улучшились отношения с ФРГ, поскольку на выборах в Бундестаг победила социал-демократическая партия Германии. Это послужило началом более развёрнутых двухсторонних контактов.
    
    В 1972 году ФРГ и СССР официально заключили мирный договор. Западная Германия признала конфигурацию границ СССР, Польши и ГДР. В 70-е и 80-е советско-западногерманские отношения развивались весьма позитивно.
    
    Наиболее натянутыми были отношения с Великобританией и с Японией. В случае с Великобританией многое зависило от правящей там партии. С консерваторами отношения были натянутые, с лейбористами "--- более тёплые. При лейбористах проблемы были только в тех ситуациях, когда ловился разведчик одной из двух стран.
    
    Советский Союз не заключил с Японией после Второй Мировой из-за того, что Япония требует возвращения четырёх южных Курильских островов.
    
    \subsection{Помощь северному Вьетнаму в войне с США}

    Из всех капиталистических стран важнейшими являлись отношения с США. Были периоды как конструктивного сотрудничества, так и его крушения. С 1964 года всё открывалось новым обострением: в том же 1964 году, когда Хрущёв был отстранён от власти и началась эпоха застоя, США начали полномасштабные военные действия во Вьетнаме.
  
    В период 1965-1973 гг. американцы пытались поддержать режим южного Вьетнама, а на севере у власти были социалистические силы, поддерживаемые Советским Союзом. В ходе этой войны США потеряли около 60 тысяч человек, а Вьетнам (как северный, так и южный) "--- от одного до трёх миллионов человек. Советские военные специалисты готовили армию северного Вьетнама, советские лётчики непосредственно участвовали в столкновениях и боевых вылетах. Впоследствии американские войска были выведены из Вьетнама на фоне мощной антивоенной кампании в США и других факторов.
    
    Победа вьетнамских коммунистов способствовала поступательному движению социалистов в Лаосе, Кампучии (Камбодже) и других странах региона. В 70-е годы возрастает количество стран Азии и Африки, где приступают к строительству социализма.
    
    В юго-восточной Азии советское руководство поддерживало повстанческие движения, но влияние СССР постепенно падало под давлением Китая. Под влиянием большого количества этнических китайцев в регионе, культурной близости местного населения с Китаем местные движения становились всё более лояльными Китаю. Это обостряло советско-китайские отношения.

    \subsection{Разрядка международной напряженности}
    
    Советский Союз пошёл на сближение с США. В 1972 году Никсон приехал в Москву и подписал соглашения об ограничении стратегических вооружений (ОСВ-1). Сокращались межконтинентальные ракеты наземного и подводного базирования. Для обеих стран это было важно, так как экономикам обеих стран было тяжело поддерживать этот потенциал, а последствия конфликта с использованием такого вооружения будут плачевны для обеих стран. Установился режим разрядки.
    
    После визита Никсона часто руководители ездили друг к другу в гости, обсуждали политические и экономические вопросы.

    \subsection{Совещание по сотрудничеству и безопасности в Европе}
    
    Совещание по сотрудничеству и безопасности в Европе состоялось в 1975 году в г. Хельсинки. На этом совещании были представлены 33 европейских государства, США и Канада. Был подписан заключительный акт совещания, который узаконил послевоенное положение в Европе и в мире. Выражалось намерение, что никто не будет навязывать возможное изменение этого строя. Признавались установившиеся в Европе и в мире границы.
    
    Советский Союз и США продолжали пытаться остановить гонку вооружений. В 1979 году был подписан договор ОСВ-2, в котором предполагалось ещё установить некие ограничения на размещение и производство ядерного оружия. Этот договор силы не получил, так как в декабре 1979 года Советский Союз ввёл войска в Афганистан. Сенат США не ратифицировал договор ОСВ-2. Снова вспыхнула гонка вооружений. В западной Европе размещаются американские ракеты средней дальности, оснащённые ядерной боеголовкой.

    \subsection{Афганская война}

    \subsubsection{Причины}
    Причины Афганской войны (ввода советских войск на территорию Афганистана) в следующем:

    В 1978 году в Афганистане произошла революция. К власти пришла народно-демократическая партия Афганистана. Она вводила Афганистан в русло советской внешней политики. Против народно-демократической партии выступили внутренние силы, началась гражданская война. Афганские коммунисты обратились за помощью к СССР, тот ответил финансовой поддержке.
    
    В Иране тем временем происходит исламская революция. Новый аятолла Хоммейни запретил марксистские партии. А Китай ввёл войска во Вьетнам. Это показало, что Советский Союз теряет позиции в крупных азиатских странах. Именно под влиянием этих причин руководство страны во главе с Брежневым принимает решение отправить советское войска. Решение было принято Брежневым и несколькими первыми фигурами. Против выступил только Косыгин, который знал, что у государства есть серьёзные экономические проблемы, которые не позволят успешно вести войну.

    \subsubsection{Основные действия}
    С декабря 1979 года вводится ограниченный контингент советских войск в Афганистан для выполнения интернационального долга, помощи братскому народу Афганистана. Помощь моджахедам "--- противникам народно-демократической партии Афганистана "--- оказывали США.
    
    Почти в полном составе Генеральной ассамблеей ООН принимается резолюция, осуждающая ввод войск в Афганистан. Эти события отрицательно сказались на внешнеполитической репутации СССР.
    
    На первом этапе войны С декабря 1979 года по февраль 1980 года советские войска вводятся в Афганистан и размещаются погарнизонно в разных частях страны.
    
    Второй этап. Март 1980 --- апрель 1985 года. Ведутся активные боевые действия. Советский Союз проводит рекомплектацию и укрепление союзных сил.
    
    Третий этап. Апрель 1985 года --- янваль 1987 года. Наступает перелом в войне. Новое советское руководство предпочитает снизить активность в Афганистане. Местные войска поддерживаются авиацией и сапёрами. Некоторые соединения выводятся из страны.
    
    Заключительный этап. Январь 1987 года --- февраль 1989 года. Подготовка советских войск к выводу и их полный вывод из Афганистана.

    \subsubsection{Последствия войны}
    
    Потери Советского Союза подсчитать сложно. Обычно они колеблются от 15 до 26 тысяч человек. Среди афганцев потери гораздо больше: от 600 тысяч до 2 миллионов человек.
    
    Афганская война привела к подрыву и без того проблемной экономики Советского Союза. Обществу не была понятна идеологическая и конкретная цель этой войны.

    \subsection{Обострение советско"=американских отношений}
    
    В начале 1980х отношения со странами Запада и США обострились настолько, что эти годы можно называть пиком Холодной войны. В эти годы конструктивные контакты практически прекратились. Вводится эмбарго на поставку зерна в СССР.
    
    В средствах массовой пропаганды стран обоих блоков развиваются мысли о подготовке к войне. Под прикрытием разрядки ведётся противостояние в странах Третьего мира.
    
    С 1985 года опять начинается потепление и разрядка.
    