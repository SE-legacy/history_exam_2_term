\section{Билет 10. Политический курс советского государства и внутрипартийная борьба.}

\subsection{Уголовные и гражданские кодексы. И всякое связанное с судами (реформы и тд)}

В 1920 году были утверждены уголовные и гражданские кодексы. Упразднены революционные трибуналы и вместо них были введены народные суды. Судопроизводство оформлено по"=новому. Суда присяжных не было, так как он считался буржуазным. Заново утверждены прокуратура и адвокатура. Законодательно закреплена цензура. 

ОГПУ (Объединенное государственное политическое управление) существовала при совнаркоме. Выполняла функции ВЧК.

\subsection{Эмиграция, репрессии, ликвидация небольших партий.}

Победа большевиков в гражданской войне вынудила несогласных покинуть страну. Началась русская эмиграция первой волны, её численность оценивается в количестве 2--3 млн. чел. В них входили такие знатные деятели, как Леонтьев, Бунин, Шаляпин, Рахманинов, Кандинский, Сикорский, Прокудин-Горский.

В рамках борьбы с инакомыслием по инициативе Ленина, в 1922 году из страны было выслано более 80 представителей интеллигенции. Эту высылку назвали «Философский пароход».

Всех активно выражавших несогласие с политикой репрессировали. В 1923 году на Соловецких островах начал действовать СЛОН (Соловецкий лагерь особого назначения). Тут отбывали наказание не только преступники, но и представители различных партий. Он стал одним из первых в системе исправительных лагерей.

В начале 1920"=х последние меньшевистские и эсеровские организации были распущены или пресечены государством. В 1923--1924 году устранены последние меньшевистские организации в Грузии и Украине. Летом 1922 года в Москве прошел открытый судебный процесс над эсерами. 

\subsection{ЦК РКП(б) и Политбюро (состав, роль).}

Наиболее значительные решения в стране принимались центральным коммитетом РКП(б). Он состоял из 20 членов и 19 кандидатов. Но подлинным руководящим центром было Политбюро, состоявшее из 5 человек: Сталин, Троцкий, Ленин, Зиновьев, Каменев. Бухарин, Калинин, Молотов были кандидатами в политбюро. Оно обсуждало все крупные политические и экономические вопросы, решения по которым сначала закреплялись в партийных органах, а потом выдвигались в советских органах. Такая модель действовала в том числе на местном уровне.

С 1922 года РКП(б) переименовалась в ВКП(б) – Всесоюзная коммунистическая партия большевиков.

\subsection{Нарастание внутрипартийной борьбы.} 

Регулярно устраивались выборы в советские органы власти. Поскольку постоянно на этих выборах побеждала одна и та же партия, то люди перестали ходить на выборы. Это вынуждало партийное руководство посылать агитаторов в сёла, которые побуждали к действию.

Всё больше нарастала внутрипартийная борьба. Партия была похожа на замкнутую милитаристскую организацию в годы войны. Проводились попытки демократизации, но эти разговоры не приводили ни к чему.

На 10 съезде РКП(б) попытались решить этот вопрос. Была запрещена фракционная деятельность, узкий круг вождей (политбюро) могли называть любую попытку инакомыслия антипартийной.

\subsection{Борьба за власть в последние годы жизни Ленина. Троцкий против триумвирата.}

В последние годы Ленина (1922-1923) сложилась группа вокруг Льва Троцкого. Они противостояли триумирату из Зиновьева, Каменева и Сталина. После победы триумирата, вокруг Зиновьева организовалась «Ленинградская» оппозиция с 1925 года. Затем возникла «Объединённая» оппозиция в 1926--1927 годах во главе с Троцким, Зиновьевым и Каменевым против Сталина. Во время противостояния всегда начинались дискуссии по поводу противоречий.

\subsection{Сталин "--- генеральный секретарь ЦК ВКП(б). Смерть Ленина.}

Весной 1922 года генеральным секретарём ЦК ВКП(б) стал Сталин. Это ускорило процесс централизации и бюрократизации партии. Во главе главкомов стали находиться освобожденные секретари, которые не занимались никакой другой работой. Они должны были иметь определенный стаж и подтверждение сверху (т.е. нельзя было продвинуться вверх без утверждения Сталина). Такой аппаратный курс вызывал протест в рядах старой партийной гвардии, что еще больше обостряло борьбу. На протяжении 1923--1924 года партия вступила на полосу испытаний и раскол между старой  и новой гвардией стал очевиден. В начале 1923 года тяжелобольной Ленин пишет письмо к съезду, в котором дал характеристики всем влиятельным в партии фигурам. Сталин: «Товарищ Сталин, сделавшись генсеком, сосредоточил в своих руках серьезную власть и я сомневаюсь, что он сможет правильно ей пользоваться».

В январе 1924 года умирает Ленин. Теперь политбюро состояло из 7 человек: Сталин, Троцкий, Зиновьев, Каменев, Бухарин, Рыков (следующий глава совнаркома после ленина), Томский. 

\subsection{Гос. структура}

Что касается советской государственной структуры, то все они стали во главе органов. Рыков стал главой совнаркома, Томский возглавлял ВЦСПС (профсоюзное движение), Калинин был председателем ВЦИК, Зиновьев возглавлял исполком коминтерна, Троцкий был главой армии и флота, Дзержинский возглавлял высший совет народного хозяйства и был главой ОГПУ. Бухарин возглавлял газету «Правда», которая к тому времени уже являлась основной газетой.

\subsection{(мб нахер не нужная часть) Борьба за власть подробно(пытался сократить)}

Первый этап борьбы за власть начался в 1923 году, после того, как Троцкий в нескольких письмах подверг критике экономический курс, проводимый триумвиратом.

Появляется понятие троцкизм. Одна из теорий "--- теория перманентной революции заключалась в том, что революция должна происходить всегда. Также он считал, что невозможно построить социализм в отдельно взятой стране, пока не произойдет всеобщая революция. 

Эта дискуссия вылилась в обсуждение на 13 съезде, на котором большинство пошло в сторону триумвирата и Троцкого обвинили в мелкобуржуазном уклоне. В течение 1923--1924 годах проводились кадровые перестановки, постепенно ослаблявшие Троцкого, а впоследствие лишившие его предстедательства в Реввоенсовете. 

Еще одним решением было набрать членов партии из среды рабочих. (Среднестатистический член партии, который ранее был интеллигентом, превратился в рабочего, что делало массу более управляемой.)

Рост влияния Сталина заставил уйти в оппозицию Зиновьева и Каменева. Объединённую оппозицию поддерживала Крупская, Сокольников.

На 14 съезде партии в декабре 1925 года сторонники объединенной партии критикуют политику Сталина (А еще Бухарина, Рыкова, Томского). Оппозиция требовала широкой внутрипартийной демократии. Логика этой борьбы привела к тому, что Троцкий присоединился к Зиновьеву и Каменеву. Споры ведутся до 1927 года. В конечном итоге верх одерживает группа Сталина, а его противники выводятся из партийных органов. Наиболее опасный конкурент Сталина – Троцкий, был выслан в Казахстан в 1928 году.

Внутрипартийная борьба продолжалась и кризис хлебозаготовок в 1927 году поставило вопрос о продолжении НЭПа. Сформировалась «Правая» оппозиция в лице Бухарина, Рыкова и Томского. Они считали возможным продолжать развитие сельского хозяйства и промышленности на основе идей НЭПа. По мнению Бухарина, идеи НЭПа должны были завершить строительство социализма лишь в ближайшие 30--40 лет. Но против него выступили Каменев, Каганович и др, считавшие, что западные страны не позволят советскому союзу развиваться в мирном ключе. Нужно было догнать и перегнать западные страны, для чего, по мнению Сталина, требовалось свернуть НЭП. 

\subsection{Результат политической борьбы.}

В 1929 году правая оппозиция потерпела поражение.Таким образом, к 1930 году фактическим лидером партии сделался единолично Сталин. Политбюро состояло из Калинина, Сталина, Молотова, Ворошилов, Киров, Куйбышев, Каганович.