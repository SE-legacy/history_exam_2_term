\section{Билет 9. Личность и деятельность Ленина.}

\subsection{Рождение, детство, семья.}

Владимир Ильич Ульянов (Ленин) – 22 апреля 1870 – 1924 гг. Родился в городе Симбирск (Ульяновск). Родился в семье чиновника. 

Отец был по происхождению из астраханских мещан. Дослужился до 4 уровня по табелю о рангах. Работал сначала инспектором, а потом директором народных училищ Самарской губернии. Илья Николаевич благодаря своему статусу стал потомственным дворянином.

Детей в семье было пятеро – Александр, Владимир и Дмитрий, Ольга и Мария. Старший сын Александр поступил в московский университет, но потом с другими студентами создал организацию «Террористическую фракцию», исповедую народнические идеи, готовил покушение на Александра 3. Организация была раскрыта в 1887 году и Александр был приговорён к смертной казни повешением. 

\subsection{Учеба}

В этом же году Владимир Ильич заканчивал гимназию. Проявлял выдающиеся достижения в учёбе. По всем предметам кроме логики были пятёрки. Несмотря на террористическую деятельность брата, Владимир поступил на юридический факультет Казанского университета, но с самых первых дней Владимир занялся революционной борьбой. Читал Маркса, Энгельса, Плеханова и вместе с другими студентами обсуждал эту литературу. Через 3 месяца после поступления случились студенческие беспорядки, в которых Владимир был одним из активистов. Из-за этого был отчислен. Доступ в Вузы был закрыт и Владимир стал учиться сам. Это позволило в 1891 году сдать экзамены экстерном. Так он сдал экзамены юридического факультета в петербургском университете.

\subsection{Становление революционером.}

Юристкая карьера Владимира не привлекала. Он решил стать профессиональным революционером и посвятить себя борьбе с царизмом. В начале 1890-х годов Ульянов являлся убеждённым марксистом. Этим он заражал других людей.

В 1895 году создаёт «союз борьбы рабочего класса». Союз ведет пропагандистскую работу среди пролетариата Петербурга. 

Фамилия-псевдоним, возможно, взята как название реки Лена. В антитезе Плеханову, который подписывался как Волгин.

\subsection{хз как назвать мб и не нужна}

После раскрытия Союза, Ленин оказывается в тюрьме на 2 года, а затем в ссылке 3 года в селе Тушино (?). Он вступает в брак.

После освобождения из ссылки уезжает в Швейцарию, где при его участии создаётся газета «Искра». Редакция этой газеты стала объединением социал-демократов. Там публикуются работы Ленина. Проходит 2 съезд РСДРП, на котором полностью оформляется политическая партия, создаётся программа партии (Ленин один из разработчиков). Ленин становится вожаком лево-радикального крыла РСДРП.

\subsection{Времена революции.}

Во время революции 1905-1907 находился в россии, где распространял идеи противодействия царизма. Он верил в всеобщее вооружённое восстание. После окончания революции уезжает в эмиграцию до 1917 года.

Летом 1917 года распространяется информация о связях большевиков и Ленина с германским и австрийским командованием.

Ленин становится одним из главных организаторов Октябрьской революции 1917 года, а затем возглавляет Совнарком до своей смерти.

\subsection{Годы гражданской войны.}

В годы гражданской войны проявляет себя как энергичный талантливый организатор, проявляющий безжалостное отношение к противникам большевиков. Он становится организатором красного террора. Ленин был атеистом. Это повлияло на отделение церкви от государства.

\subsection{Последние годы жизни и культ Ленина}

Поддержка Ленина была велика, возник культ личности.
Последние годы жизни сильно болел. Умер в январе 1924 года от кровоизлияния в мозг.

Для советской идеологии Ленин был фигурой номер 1. Получил известность за рубежом. Его заслуги в создании советского государства всячески превозносились (литература, искусство, кино), но были и противники Ленина. Его считали фанатиком, который вел страну к разрушению.

5 памятников Ленину в Саратове