\section{Билет 23. СССР в 1964-1985гг внутренняя политика и экономика. Л. И. Брежнев.}

\subsection{Данный период "--- период "застоя".}

Этот период именуется по-разному, либо как период «наиболее стабильного развития СССР», либо как период «застоя». Если задуматься, то оба определения – синонимы. 

В этот период назревал кризис государства и общества. 

Одной из отличительных черт был демографический рост. В 1970-м году население СССР составляло 240 млн. чел. А через 15 лет уже 280 млн. Также этот период характеризуется урбанизацией – увеличением городского населения. В начале периода – 57\%, к концу – 64\%.

\subsection{Утверждение Л.И. Брежнева.}

В 1931 году вступил в партию. В конце 1930-х перешел на партийные должности и строил очень быструю карьеру. Он участвовал в ВОВ на политических должностях. Он служил в политических управлениях фронтов и армий. Так он поднимался все выше и выше. Непосредственно участвовал в освобождении Новороссийска. Окончил войну в звании генерал-майора. После войны находился на руководящих партийных должностях разных республик, а в 1956 году вошел в секретариат ЦК, а в 1960 стал председателем президиума верховного совета. Участвовал в заговоре против Хрущёва. После смещения стал первым секретарём ЦК партии, а в 1966 году эту должность вновь начали именовать «генеральный секретарь ЦК». Он был четырежды героем советского союза. Пережил клиническую смерть в 1976 году, после этого был глубоко больным, умер окончательно в 1982 году. 

\subsection{Конституция 1977 года.}

С точки зрения развития политической системы, до сих пор сохраняется партийная номенклатура (правящая элита общества). 

Принимается четвертая конституция в 1977 году. Совет депутатов трудящихся меняется на совет народных депутатов. Таким образом подчеркивали социальную однородность. В конституции появилась статья №6, которая юридически закрепила положение КПСС. Высшим законодательным органом власти между заседаниями становился президиум, а председателем был Н.В. Подгорный. Первым секретарём ЦК был Л.И. Брежнев, совет министров СССР возглавил Косыгин.

Конституция отменила административное деление. Восстанавливается вертикальная связь в экономике. Это вело к возрастанию роли бюрократии. 

