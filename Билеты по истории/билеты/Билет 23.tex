\section{Билет 23. СССР в 1964-1985гг внутренняя политика и экономика. Л. И. Брежнев.}

\subsection{Данный период "--- период "застоя".}

Этот период именуется по-разному, либо как период «наиболее стабильного развития СССР», либо как период «застоя». Если задуматься, то оба определения – синонимы. 

В этот период назревал кризис государства и общества. 

Одной из отличительных черт был демографический рост. В 1970-м году население СССР составляло 240 млн. чел. А через 15 лет уже 280 млн. Также этот период характеризуется урбанизацией – увеличением городского населения. В начале периода – 57\%, к концу – 64\%.

\subsection{Утверждение Л.И. Брежнева.}

В 1931 году вступил в партию. В конце 1930-х перешел на партийные должности и строил очень быструю карьеру. Он участвовал в ВОВ на политических должностях. Он служил в политических управлениях фронтов и армий. Так он поднимался все выше и выше. Непосредственно участвовал в освобождении Новороссийска. Окончил войну в звании генерал-майора. После войны находился на руководящих партийных должностях разных республик, а в 1956 году вошел в секретариат ЦК, а в 1960 стал председателем президиума верховного совета. Участвовал в заговоре против Хрущёва. После смещения стал первым секретарём ЦК партии, а в 1966 году эту должность вновь начали именовать «генеральный секретарь ЦК». Он был четырежды героем советского союза. Пережил клиническую смерть в 1976 году, после этого был глубоко больным, умер окончательно в 1982 году. 

\subsection{Переход к консервативному внутриполитическому курсу}

Противоборство мнений по вопросу о выборе путей дальнейшего развития общества завершилось поворотом от реформаторства периода хрущевской "оттепели" к умеренно-консервативному курсу в политике и идеологии. Идейно-теоретической основой деятельности нового руководства была разработанная в конце 60-х годов концепция "развитого социализма". Эта концепция концентрировала внимание на необходимости решения текущих задач одного из этапов построения коммунизма - этапа "развитого социализма". Существовавшие в обществе недочеты и кризисные явления рассматривались как результат неизбежных в процессе его развития противоречий. Устранению недостатков должна была способствовать политика "совершенствования" социализма. Активными проводниками концепции "развитого социализма" были Л.И. Брежнев, сменивший его на посту главы КПСС Ю.В. Андропов и преемник последнего - К.У. Черненко

\subsection{Реформы А.Н. Косыгина}

Косыгинские реформы (1965 – 1970 гг.) предполагали увеличение мотивации колхозников. Стало очевидно, что работать только за идею миллионы колхозников не хотят. Правительство Косыгина предпринимает:

\begin{itemize}
    \item Снижение плана обязательных закупок зерна и объявляется неизменным на 10 лет. 
    \item Производимые сверхплана товары государство покупает по повышенному тарифу.
    \item Снимаются ограничения на подсобное хозяйство.
    \item Увеличение хозрасчетной самостоятельности на предприятиях.
    \item Отказ от совнархозов и восстановления отраслевого принципа
    \item Сокращение числа плановых показателей (с 30 до 9)    
\end{itemize}

Такими мерами пытались стимулировать развитие экономики и промышленности. Но эта реформа не меняла устройство плановой экономики, лишь добавила «экономические рычаги». 


\subsection{Социально"=экономическое развитие страны. Падение темпов экономического роста.}

Когда началась реформа, аппарат стал сопротивляться, так как увидел в этом посягательство на власть. (При плановой экономике предприятию выгодно производить один тип товара по высокой цене). В начале 1970-х годов от этой реформы отказались. Для совхозов и колхозов лишь незначительно увеличилась самостоятельность. В 1974 году колхозники получают паспорта.

Экономика страны в этот период характеризовалась перманентным снижением темпов роста. Реформы косыгина только задержали темпы падения, но когда в 1970-х это крутое пике стала еще более ощутимым, то от реформ отказались. Даже официальная статистика не была способна скрыть показатели кризиса. Экспорт стал сырьевым (вывозился нефть, газ, лес). 

 конца 1960-х годов начинается разработка нефте-газовых месторождений в западной сибири. Одновременно с этим начинается рост цен на нефть. Советский союз начинает получать баснословные доходы от нефтедолларовой выручки. Это позволяло несколько сглаживать и компенсировать падение некоторых экономических показателей.

 Из-за социальной «уравниловки» инженер на производстве получал немногим меньше, чем работник сталелитейного цеха. Из-за этого падение стимула роста квалификации и производительности труда. В промышленности ручным трудом было занято более 40\% работников (не требующее специализированного образования). Негативные процессы в связи с этим стали отражаться на социальной жизни. Медленное строительство домов. Оно шло, но не успевало компенсировать потребность (к тому же учитывайте бурный рост населения в тот период). Всё чаще возникали проблемы с продовольствием, транспортом, медицинским обслуживанием. Заработная плата росла незначительно. Эти процессы привели к тому, что в начале 80-х годов реальный уровень жизни падал. Начал развиваться чёрный рынок. Органы пытались бороться с этим. 


\subsection{Конституция 1977 года.}

С точки зрения развития политической системы, до сих пор сохраняется партийная номенклатура (правящая элита общества). 

Принимается четвертая конституция в 1977 году. Совет депутатов трудящихся меняется на совет народных депутатов. Таким образом подчеркивали социальную однородность. В конституции появилась статья №6, которая юридически закрепила положение КПСС. Высшим законодательным органом власти между заседаниями становился президиум, а председателем был Н.В. Подгорный. Генеральным секретарём ЦК был Л.И. Брежнев, совет министров СССР возглавил Косыгин.

Конституция отменила административное деление. Восстанавливается вертикальная связь в экономике. Это вело к возрастанию роли бюрократии. 

Советская бюрократия стала самой большой в мире, уступая только Китаю. Советская бюрократия достигла 18 миллионов человек (1/7 часть трудоспособного населения). Общество испытывало дефицит средств для содержания бюрократического аппарата.

\subsection{Изменение в государственно"=политическом руководстве в последние годы брежневского периода}

Партийные собрания, конференции проводятся постоянно, но они все больше напоминают спектакли, на которых единогласно принимаются решения, спущенные свыше. Повсеместно принята практика продвижения только одного кандидата. Он мог не пройти на должность, только если против него проголосует большинство членов партии, что было невозможно, так как никто не голосовал против. Исчезновение хоть каких-то признаков демократии приводило к пассивности органов власти. В выборах по статистике участвовало 90\% населения, но как правило эти цифры намеренно завышались и мало что означали.

Тем не менее, социальные лифты всё еще работали. Но подниматься по ним можно было только став примерным чиновником и слушаясь всех спущенных свыше указаний. Это действовало во всех областях – экономике, культуре и т.д. Это было основной причиной того, почему партия в этот период активно росла. 

Исключение из рядов партии автоматически вело снятия со всех постов.
В этот период высшие слои партийной номенклатуры становились элитой и обладали множеством привилегий: частые поездки зарубеж (вывозя при этом валюту), персональные дачи, машины, квартиры, имели доступ к закрытым спец.распределителям, где можно было приобрести дефицитные товары. Это несоответствие идее партии становилось всё более разительным. (Как на это реагировало население?)

При Брежневе произошел небольшой возврат к сталинизму. 23 съезд КПСС (1966 г.) положил конец разоблачению культа личности Сталина. Стали появляться положительные черты Сталина, а отрицательные качества затушёвывались. Постепенно начал расти культ личности Брежнева, постоянно восхвалялись его качества, стали называть героем ВОВ, хотя вклад был небольшим. Культ личности достиг пика к концу жизни самого Брежнева. Но этот новый культ, в отличие от сталинского, все больше напоминал фарс. Брежнев под конец своей жизни нуждался в постоянном приёме лекарств, не мог работать больше 2-3 часов в день, дееспособность падала. Из-за этого появилось огромное количество анекдотов про него. Анекдоты подчеркивали его недееспособное положение.

\subsection{Противоречия общественной жизни и культуры}



\subsection{Диссидентские движения}

Всё это привело к тому, что в обществе начали возрастать негативное отношение к власти. Ответом народа стало диссидентское движение. Диссидент – несогласный. Иногда оно выражалось в том, что люди уезжали из страны. Естественно всё это каралось уголовными статьями, особенно за «Агитацию пропаганды». Людей вывозили из страны (а куда?). Применялось насильное лечение в психиатрических больницах. 
Движение диссидентов набирало обороты. 


Самым крупным стало преследование писателей Даниэля и Синявского. Они осмелились опубликовать свои произведения за границей. Их привлекли за публикацию, за то, что в своих произведениях они критикуют сталинизм и очернение советского строя. В 1968 году произошли события Чехословакии, которые также повлияли на рост числа диссидентов. Власть дала понять, что идеи 20-го съезда стали не совсем актуальны.

\subsection{Поиски путей упрочения социализма при Ю.В. Андропове и К.У. Черненко}

Попытку реформировать партию предпринял Ю.В. Андропов (который потом занял место Брежнева). 
Он пришел к власти в 68 лет после смерти Брежнева в 1982 году. Андропов до генерального секретаря возглавлял КГБ СССР. Он, как глава КГБ попытался придать аппарату эффективности и избавить от коррупционеров. Милиция усилилась, стали останавливать прохожих в рабочее время. По инерции курс Андропова продолжился и при следующем главе – Черненко. К 1985 году политбюро быстро и естественным образом уходило из жизни.
