\section{Билет 26. Внешняя политика Советского союза в годы <<Перестройки>>}

Стремлением Советского Союза во время перестройки было \textit{улучшение коммуникации с развитыми капиталистическими странами}. Это было провозглашено на съезде №27 в 1986 году. 

\subsection{<<Новое политическое мышление>>}
Впервые советская власть во главу угла поставила не построение социализма, а общечеловеческие ценности (жизнь, свобода, безопасность, независимость, здоровье, социальное равенство). Приоритетом становятся глобальные проблемы. Решить эти проблемы можно только балансом сил и взаимопонимания. Эта концепция получила название <<новое политическое мышление>>. Она подразумевала
\begin{itemize}
    \item отказ от деления мира на противостоящие системы,
    \item приоритет мирного решения международных проблем,
    \item решение этих проблем на основе баланса интересов, а не баланса сил.
\end{itemize} \vspace{0.5cm}

НПМ означало отказ от пролетариатского интернационализма. 

\subsection{Переговоры с лидерами западных стран}
Горбачев часто встречался с лидерами западных государств. На этих встречах обсуждалось 3 вопроса: 
\begin{itemize}
    \item прекращение гонки вооружений и его сокращение, 
        \begin{itemize}
            \item уменьшения количества ракет у обоих сторон
            \item стратегическая оборонная инициатива (СОИ)
        \end{itemize} 
    \item вывод войск из Афганистана, 
    \item расширение контактов между странами разных социальных систем.
\end{itemize} \vspace{0.5cm}

В результате переговоров и дипломатических контактов в 1987 году был подписан договор о ликвидации ракет средней и меньшей дальности. Смысл в том, что советские и американские ракеты ликвидировались в Европе. Кроме того, Советский союз ликвидировал часть ракет в дальневосточном регионе. Это было первое реальное сокращение целого класса вооружений, предпринятое с момента окончания второй мировой войны. Но подписав такой договор, Советский Союз пошёл на большие уступки, так как хотел, чтобы к договору присоединились Франция и Великобритания, однако этого не произошло.

\subsection{отношения СССР с другими странами}
Потепление в отношениях было уже с 1985 года. Связи между советскими организациями и частными лицами с Запада расширяются. Советское руководство было заинтересовано в развитии технических связей, а также надеялось получить кредиты. Западные государства шли на расширение торговых связей прежде всего из-за политической ситуации внутри СССР.

\underline{Приоткрывается «железный занавес»}. 

Во главе министерства иностранных дел был Шиворнадзе. Советская дипломатия пыталась поддерживать дружеские отношения с максимальным количеством стран, из-за чего увеличилась площадь соприкосновения СССР и КНР. Подобные же процессы происходили и в отношении стран Азии, налаживаются отношения с Чили, с африканскими странами (в частности с ЮАР). При этом СССР всё еще продолжает оказывать военно"=экономическую помощь коммунистическим странам (Алжир, Ангола, Эфиопия, Ирак, Сирия). Отношения советского союза со странами социалистического содружества строились на обычной почве (ОВД), но изменения происходили. Так как западные социалистические страны оказались в арьергарде перестройки: СССР подталкивает их к изменениям.

В 1989 году происходит вывод войск из Афиганистана. Но даже после вывода войск, СССР продолжает оказывать поддержку афганскому правительству. Несмотря на вывод советских войск, афганское правительство на некоторое время сохранило позиции и продолжало войну.

С этого же 1989 года во всех социалистических странах мира (за исключением двух: Северная Корея, Куба) приходят оппозиционные силы. Все страны форсированно вводят рыночные отношения, что приводит к падению жизненного уровня и темпам экономики. Из-за этого снижается объем торговли между социалистическими странами. Изменения цен на продукции, перебои с транспортом – всё это приводит к тому, что экономическое общение между социалистическими странами падает до минимума, Союз Экономической Взаимопомощи перестаёт существовать. ОВД перестаёт существовать в марте 1991 года. Советский Союз в конце 1991 года выводит войска из восточно-европейских стран, тем самым заканчивая своё военное присутствие. 

В октябре 1990 года происходит объединение Германии (сам процесс начался с приходом к власти Христианской демократической партии в 1989 году). Эта партия провозгласила курс на скорейшее объединение страны. Это осложнялось тем, что в ГДР находились Советские войска. Начались переговоры, в результате которых Германия остаётся в НАТО, а СССР получает выгодные кредиты от ФРГ. Кроме того, в процессе обсуждения объединения Германии, поднимался вопрос о нераспространении НАТО на восток. По устным заверениям США, НАТО не будет расширять своё присутствие на востоке. Однако письменную форму такие соглашения так и не обрели. 

С апреля 1991 года начинается потепление отношений с Японией (однако вопрос о Курилах остаётся открытым). 

В странах восточной Европы начинаются мирные демонстрации направленные на устранение просоветского правительства. Такие революции прошли в Польше, Венгрии, Чехословакии, Болгарии, Румынии. Получили название <<Бархатных революций>>, так как прошли без жертв.

\subsection{Вопрос о судьбе Советского ядерного оружия}

На момент 1991 года все ядерные боеголовки были сосредоточены в Белоруссии, Украине, Казахстане и РСФСР. После распада Советского Союза встал вопрос о судьбе договора СНВ-1. В 1992 году были подписан Лиссабонский протокол, регулировавший преемственность стран в вопросе договора СНВ-1. В дальнейшем Беларусь и Казахстан и передали всё ядерное вооружение России. 

\vspace{0.5cm}
\begin{itemize}
    \item[] В целом, период с 1985 по 1991 год характеризовался улучшением глобальной обстановки, преодолен «железный занавес», перестал существовать единый социалистический лагерь, в связи с чем можно говорить об окончании холодной войны. 
\end{itemize}
