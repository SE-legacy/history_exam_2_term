\section{Билет 22. Социально-экономическое и культурное развитие страны в период “Оттепели".}

\subsection{Совнархозы.}

С 1957 года в СССР начинается экономическая реформа для повышения производительности. Советский союз был разделен на районы, в которых развитием экономики должны были заниматься совнархозы. Каждая из республик делилась на множество таких районов. Совнархозы подчинялись правительству конкретной республики. Такая децентрализация требовалась для того, чтобы приболизить руководство к низам, однако суть экономики не менялась и особых экономических результатов реформа не дала.

\subsection{Реформы в промышленности.}

Поскольку к середине 50"=х годов государство выбралось из экономических трудностей, то СССР перешел к экономическому росту. В пятой пятилетке (1951--1955) рост промышленного роста составлял 70\%. 

Разворачивались промышленные предприятия в поволжье, закавказье. Увеличивались капиталовложения в промышленность. Но при этом акцент все еще делался на тяжелой промышленности. Для государства важнее были отрасли напрямую связаны с военной отраслью. 

Велись работы по созданию единой энергетической системы. Создаются множество новых электростанций. Целый каскад ГЭС (Волжско"=Камский) открывается в тот период времени. Потом Нижегородская, Жигулевская, Волжская, Саратовская. Эти энергостанции были очень важной вехой развития энергетической промышленности, так как до сих пор являются важными объектами. 

Появляются первые атомные электростанции. Первой стала Обнинская АЭС. Благодаря всему этому в 56--58 годах (6"=я пятилетка) темпы развития промышленности были очень велики. Высокие темпы роста побудили Хрущева объявить семилетний план, который был признан авантюрой после отставки Хрущева.

Изменился топливный баланс. В 53 году нефть и газ составляли 32\% топливного баланса, а в 65 уже 52\%. 

Развивается воздушный транспорт. В 1957 году был спущен первый атомный ледокол «Ленин».

\subsection{Развитие самолетов, космических ракет. Освоение космоса. Появление телевидения.}

Развитие экономики происходит под эгидой научно"=технического роста. К этому же периоду относят научно"=техническую революцию в СССР. Это обуславливается автоматизацией, контролем чего"=нибудь электроникой, созданием новых материалов. 

СССР впереди по освоению космоса, химизации аграрного сектора, развитии электроники. Всё это подкрепляется крупными государственными программами. Технический прогресс способствовал изменениям как в гражданской, так и в военной сфере. 

В 1956 году опробуется первый реактивный пассажирский самолет ТУ-104. В 1957 году стартовал первый в мире советский искусственный спутник земли. Советский союз становится лидером освоения космического проекта, который проводился под руководством Королёва. Благодаря космическому проекту СССР 12 апреля 1961 года произошёл первый запуск человека в космос.

Показатель научно-технического прогресса, это также развитие средств массовой информации. В 50-е годы появилось телевидение. 

\subsection{Развитие сельского хоз"=ва}

В то же время некоторые отрасли хозяйства не справлялись с планом. Это было сельское хозяйство и легкая промышленность. В 1953 году ЦК КПСС выработал новый курс по развитию сельского хозяйства. 

\textbf{Задумались о материальной заинтересованности колхозников:} 

\begin{itemize}
    \item начали повышать заготовительные и закупочные цены
    \item снижались нормы обязательных поставок государству
    \item увеличивалось вложение в машиностроение (но только для сельского хозяйства)
\end{itemize}

Кроме того, финансовые вливания не приводили к изменению структуры землепользования. Помимо финансовой помощи проблему сельского хозяйства пытались решить увеличением посевных площадей, незначительным увеличением экономической самостоятельности и других. 

\subsection{"Целинная эпопея".}

Создана программа освоения целинных земель. Это северная часть Казахстана, юг западной Сибири. Их начинают активно распахивать с 1954 года. Это дало лишь кратковременный положительный результат. В 1957 году Хрущёв выдвинул лозунг о том, что СССР должен перегнать Америку по производству всего. В частности, он предложил утроить производство мяса в стране за три года. 

\subsection{"Рязанский почин".}

Но и тут нашлись те, кто решил этот план перевыполнить. Секретарь рязанского обкома Ларионов предложил утроить производство за год. В историю это вошло под названием «рязанский почин». Стали забивать на мясо молочный скот, племенной скот, стали забирать личный скот у колхозников, стали покупать скот у других регионов. План был выполнен… но только по мясу. По молоку нет. А потом резко упало производство и мясо, и молока, и даже зерна, так как колхозники бунтовали из-за отобранного скота. Уже в 1960 году Ларионов умер «от сердечной недостаточности».

\subsection{Аграрный кризис "--- "Кукурузная кампания".}

«Кукурузная кампания». В 1959 году Хрущев посетил США на встречу с Эйзенхауром. Его возили по предприятиям народного хозяйства. И особенно Хрущев впечатлился кукурузным полем и по возвращении домой стал настаивать на увеличении производственных мощностей курузы. Это привело к катастрофе, так как кукуруза "--- это теплолюбивое и влаголюбивое растение.

\subsection{Реорганизация МТС.}

В 1958 году производится реорганизация МТС. Технику решили продать колхозам. Но в значительной степени эта техника нуждалась в ремонте, а на это нужны деньги, которых нет у колхозников.

\subsection{Толи еще про сельское хоз"=во, толи какая"=то другая хуйня.}

Правительство развернуло кампанию по борьбе с личным хозяйством. Но привело это к тому, что крестьяне стали неспособны обеспечивать себя и рынки. Рост продукции сельского хозяйства вместо 70\% вырос всего на 14\%.

Такие трудности заставили власти в начале 60-х годов начать повышение цен на продукции питания. И иногда население стало доводить до власти своё недовольство весьма радикальными способами. Самое известное событие – новочеркасское движение в 1962 году. Трудящиеся решили устроить демонстрацию: около 7000 человек вышли на улицы с призывом улучшить качество жизни. С помощью войск и милиции демонстрация была разогнана, несколько человек убито, зачинщиков приговорили к расстрелу.

Но уровень жизни действительно рос в период оттепели. Появились пенсии, отдельные труженники стали получать квартиры, для всех рабочих установлен 7-ми часовой рабочий день… при 6-ти дневной рабочей неделе, выросла среднемесячная зарплата с 76 до 98 рублей, возрастало потребление потребительских товаров.

\subsection{Жилищная программа.}

С 1957 года принято решение о расширенном строительстве жилых домов. Начали строиться целые микрорайоны из типовых панельных пятиэтажных домов низкой себестоимости.
