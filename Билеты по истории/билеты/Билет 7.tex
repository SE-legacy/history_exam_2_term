\section{Билет 7. НЭП(Новая Экономическая Политика) (1921--1928) в советской России}

\subsection{Причины:}

Причины экономические:

\begin{enumerate}
    \item Упадок, вызванный первой мировой и гражданскими войнами, интервенцией и политикой военного коммунизма.
    \item Продразвёрстка (обязательная сдача государству продукции сельского хозяйства).
    \item Уязвимое положение крестьянства в случае неурожая.
    \item Сокращение посевных площадей, поголовья скота.
    \item Национализация промышленности.
    \item Потеря доверия к деньгам.
    \item Возврат к натуральному хозяйству в деревне (выращивание товара только для собственных потребностей, не для продажи).
    \item Голод 1921-1923 и человеческие потери.
\end{enumerate}

\vspace{0.5cm}

Причины политические:

\begin{enumerate}
    \item Сопротивление политики военного коммунизма 
    \item Кронштадское востание (вся власть советам, а не .....)
\end{enumerate}

\vspace{0.5cm}

С 8 по 10 марта происходит 10 съезд РКПБ(Российская коммунистическая партия большевиков) в дни крондштатского восстания. Там закладываются меры НЭП(основы НЭПа):

\begin{enumerate}
    \item Замена продразвёрстки продналогом (система из 13 налогов в натуральной форме. Чтобы заработать деньги на уплату налогов и приобретение продукций, крестьяне были вынужденны торговать с гос"=вом)
    \item Свободная торговля
\end{enumerate}

\subsection{Основные меры. Сельское хозяйство:}

\begin{itemize}
    \item Вводится продналог, который составлял определенный процент от выработки, который объявляется заранее, до начала посевного сезона.
    \item Возврат возможности ведения единоличного хозяйства.
    \item В ограниченных размерах возвращены аренда земли и наёмный труд.
\end{itemize}

\subsection{Основные меры. Промышленность:}

\begin{itemize}
    \item Разрешено частным лицам создавать мелкие и средние предприятия (кол-во рабочих не больше 20).
    \item Основной формой управления производством стали тресты "--- объединения однородных или взаимосвязанных предприятий. Работая на условиях хоз расчета они самостаятельно решали: что производить, где реализовывать продукцию и тд. Законом предусматривалось, что "гос казна за долги трестов не отвечает"
    \item Возврат концессий "--- разрешение правительства брать иностранным предприятиям предприятия в аренду. 
    \item Также вернулись совместные предприятия.
    \item Поощерение коопераций (кооперативов).
    \item Для предприятий появилась возможность хозрасчёта.
    \item Гос предприятия возвращаются к добровольному найму вместо трудовой повинности.
    \item Стали приглашать старых и иностранных работников.
\end{itemize}

\subsection{Основные меры. Финансовая сфера:}

\begin{itemize}
    \item Восстановление государственного банка и системы кредитных учреждений.
    \item В 1922-1924 году проводится денежная реформа, восстановившая денежную систему в стране к системе до первой мировой войны, т.е. восстанавливается золотой стандарт. Вместо совзнаков вводятся медные и серебрянные монеты, а также бумажные знаки.
    \item Восстановлена система налогообложения, подоходный налог.
    \item Вводится налог для нэпманов (частных предпринимателей), вводятся косвенные налоги.
    \item Вводится плата за услуги: транспорта, связи, коммунального хозяйства.
\end{itemize}

\subsection{Основные меры. Торговля:}

\begin{itemize}
    \item Востановлена частная торговля.
    \item Востановлены крупные торговые сосредоточения: ярмарки, рынки. 
    \item Восстанавливаются товарные биржи.
\end{itemize}

\vspace{0.5cm}

Создаётся госплан как часть ВСНХ(высший совет народного хозяйства), который планирует экономическую деятельность страны на будущее. 

Ввиду нехватки золота у государства в 1926 году был фактически отменён золотой стандарт.

Из-за кризиса городов, многие люди переселяются в деревни, что сдерживает прогресс.

В 1920 году был утверждён план государственной комиссии по электрификации России (План ГОЭЛРО). Он подразумевал строительство нескольких электростанций. Но этот план мешают выполнять отсутствие ресурсов и специалистов. Поэтому была создана лишь одна электростанция "--- Воковская.

\subsection{Финал НЭПа:}

\beign{enumerate}
    \item Советскому государству удалось за короткие сроки выйти из кризиса, восстановить промышленность и сельское хозяйство.
    \item Макроэкономические показатели страны вернулись к довоенному уровню в 1928 году.
    \item В конечном итоге НЭП не смог решить основные экономические проблемы и власть начинает свёртывать НЭП. Официально НЭП был свернут 11 октября 1931 года. Начинается переход к индустриализации и коллективизации.
\end{enumerate}



