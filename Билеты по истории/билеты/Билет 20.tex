\section{Билет 20. Личность и деятельность И. В. Сталина в исторической науке.}

\subsection{Рождение, детство, семья.}

Иосиф Виссарионович Джугашвили (6-9 декабря 1878 года?) родился в городе Горик, в семье сапожника. Отец много пил, часто бил и детей и жену. Мать отдала Иосифа в духовное училище, а потом поступил в Тефлисскую духовную семинарию.

\subsection{Учеба, ознакомление с трудами Маркса.}

Учился хорошо, но уже в это время он осознал, что стать священником – не его судьба и во время учёбы познакомился с трудами Маркса. Это был конец 1890-х годов, когда активно распространялся марксизм. Вскоре стал собирать кружки среди рабочих и пропагандировать идеи марксизма. В 1899 году был исключен из семинарии «потому что не явился на экзамен», не явился из-за того, что просто не хотел становиться священником.

\subsection{Революцонная деятельность и партия.}

После этого Иосиф Джугашвили активно занимается революционной войной, устраивает демонстрации и забастовки вместе с другими марксистами. В Тефлисе создается местный комитет РСДРП. Партийная кличка Иосифа – Коба. На 3 съезде РСДРП, когда произошёл раскол между меньшевиками и большевиками. Джугашвили присоединился к большевикам, что было не характерно для грузинского движения.

\subsection{Личная жизнь(Первая жена).}

Женился на Екатерине Сванидзе, родился сын Яков. Екатерина умерла в 1907 году от Тифа, Якова воспитывал кто-то из двоюродных сестёр Сталина. 

\subsection{Член ЦК. Появление псевдонима <<Сталин>>. Ссылка.}

В 1909 году по предложению самого Ленина его выбирают членом центрального коммитета РСДРП. Примерно тогда он останавливается на псевдониме «Сталин». Отправляется в Туруханский край в ссылку в результате очередного ареста. Там Сталин находился с 1913 по 1917 год, когда февральская революция дала амнистию всем полит.заключенным. Он возвращается в Петроград и до возвращения Ленина из Швейцарии стал лидером большевиков. Помогал в проведении октябрьской революции 1917 года и возглавил нарком по делам национальностей. Во время гражданской войны входил в реввоенсовет. Временами руководил целыми фронтами армии. 

\subsection{Личная жизнь(Вторая жена).}

В 1918 году вторично женился на Надежде Аллилуеве. Появился сын Василий и дочь Светлана. Оба сына были участниками ВОВ. Яков был артиллеристом и попал в плен, был застрелен в 1943 году во время попытки бегства. Василий был лётчиком, после войны жизнь была тяжелой, умер скорее всего от алкоголизма.

Надежда Аллилуева в 1932 году покончила с собой. Возможно из-за той обстановки, которая была в семье. Сталин грубо обращался со своей женой.

\subsection{Член политбюро. Генеральный секретарь партии. Главный лидер партии.}

С 1922 года он также член политбюро и генеральный секретарь партии. Последняя должность вроде бы чисто техническая, однако она позволила Сталину расставить своих сторонников на ключевые должности. В это время он позиционирует себя, как потенциальный преемник Ленина. Тогда же Сталин обнаружил, что если он хочет играть роль вождя, то он должен быть не только практиком, но и идеологом и начал писать теоретические работы. Современники выделяли Эгоизм Сталина, пытливость, настойчивость, хорошо разбирался в слабостях людей. При этом манеры были достаточно простыми, выступал достаточно скучно, иностранных языков не знал. Для него были характерны однозначность суждений, упрямство, мстительность. В результате внутрипартийной борьбы в конце 1920-х годов превратился в главного лидера партии. Он всегда чутко реагировал на требования большинства партии. Умело использовал противников.

Вряд ли была сфера, в которой Сталин разбирался досконально. Троцкий назвал его «выдающейся посредственностью», так как Сталин разбирался во всём, но посредственно. Как практик, он иногда допускал серьезные ошибки, но они становились не его личными, а коллективными, так как все решения одобрялись членами политбюро, центральным комитетом.

Сталин никогда не служил в армии, хотя был командующим, генералиссимусом. Генерал Жуков, разрабатывавший крупнейшие операции ВОВ, отмечал, что со Сталиным было трудно работать в начале войны, но тем не менее, он командовал всем, его слово не подлежало обсуждению. Маршал Василевский отмечал, что попытки сталина решать возникшие вопросы в первые дни войны, приводили к еще большим проблемам. Но впоследствии он смог разобраться и учитывал советы других командующих.

\subsection{Культ личности Сталина.}

Еще до войны сложился невиданный по своему масштабу культ личности Сталина. Ещё в 1938 году вышел «Краткий курс истории ВКП(б)», в которой были заложены идеологические основы понимания истории. А потом вышла «Краткая биография Сталина». До выхода Сталин их редактировал. В обоих этих работах образ Сталина подчеркивался, как образ вождя и лидера, ведущего к страну к непременному процветанию. Эти материалы стали основой для понимания текущего положения всех граждан. Сталину приписывались все заслуги страны по построению коммунистической страны. Превознесение его личности после войны достигло апогея, так как теперь его личность связывали с победой над Гитлером. Начали публиковаться сочинения Сталина.

\subsection{Заключение.}

Ответственность Сталина за формирования в стране тоталитарно-диктаторской системы и различные репрессии непосредственна. Только с 1930 по 1953 гг. за антисоветскую деятельность было осуждено 3,2 млн. чел. Но бессмысленно возлагать вину за формирования систему только на Сталина, в этом были виновны в том числе и все лидеры партии правительства.