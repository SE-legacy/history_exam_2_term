\section{Билет 5. Гражданская война в России: причины и основные события на начальном этапе (1917-1918гг).}

\subsection{Причины Гражданской Войны в России.}
\begin{enumerate}
	\item Отказ значительной части общества признавать законным переворот в октябре 1917 года.\\
	\item Брестский мир. Многие восприняли его как предательство национальных интересов (потеря территорий, статуса). \\
	\item «Красногвардейская атака на капитал» - национализация всех предприятий. Монополия не внешнюю торговлю, захват помещичьих (землей владели также казаки и крестьяне) земель на селе. Всё это привело к противостоянию с советским правительством. \\
	\item Сепаратистские настроения в регионах, существовавшие ещё с дореволюционных времён.
\end{enumerate}

\subsection{Противостоящие силы.}
\begin{center}
	\begin{table}[H]
		\begin{tabular}{|ll|lll}
		\hline
		\multicolumn{2}{|l|}{\textbf{Причины недовольства}} &  &  &  \\ 
		\hline
		\multicolumn{1}{|l|}{\textbf{Большевики}}                          & \begin{tabular}[c]{@{}l@{}}Большевики подавили вооружённые \\ выступления в Петрограде и Москве\\ Казачьи территории Дона, Кубани, \\ южного Урала Подпольные \\ антибольшевистские организации \\ (Союз защиты Родины и свободы, \\ Союз возрождения России, \\ «Национальный центр»)\end{tabular} &  &  &  \\ 
		\cline{1-2}
		\multicolumn{1}{|l|}{\textbf{Сторонники временного правительства}} & \begin{tabular}[c]{@{}l@{}}Разгон Учредительного собра-\\ ния и заключение Брестского мира \\ 1918 г. Антибольшевистские, \\ преимущественно \\ эсеровские, правительства (Комитет \\ членов Учредительного собрания\\ (Комуч; июнь, Самара), Временное \\ Сибирское правительство (июнь, Омск), \\ Верховное управление Северной \\ области (август, Архангельск), \\ Уфимская директория (сентябрь, Уфа))\end{tabular} &  &  &  \\ 
		\cline{1-2}
		\multicolumn{1}{|l|}{\textbf{Интервенты}}                          & \begin{tabular}[c]{@{}l@{}}Бескомпромисность большевиков, \\ заключение Брестского \\ мира 1918 г.\end{tabular} &  &  &  \\ 
		\hline
		\end{tabular}
		\end{table}
\end{center}


\subsection{Создание Красной Армии.}

В январе 1918 года создаётся РККА, основу которой, поначалу, составляли добровольцы, а затем начинает восстанавливаться воинская повинность. Офицерами в РККА служили бывшие офицеры императорской армии. Вооружение и обмундирование также перешло из императорской армии. 
Основные командовавшие: Антонов-Авсеенко, Лев Троцкий, Тухачевский, Фрунзе, Егоров.
Снабжение армии осуществлялось за счет национализации производств и трудовой повинности, за счёт продовольственных отрядов, изымавшие излишки у кулаков, вводится карточная система. Действует система продразвёрстки (обязательства возлагаемые на крестьянство поставлять продукцию государству, фактически не ограничивался размер продразвёрстки).

В связи с национализацией предприятий, почти треть из них простаивала без дела. Уменьшился выпуск продукции по сравнению с 1914 годом.
Вводится обязательное военное обучение рабочих и крестьян неэксплуататоров. Вводится должность военных комиссаров и структура военного комиссариата. Возвращается система назначений. Вводится принудительная система комплектования армии.
ВЦИК принимает решение о преобразовании страны в военный лагерь, создается структура (РевВоенСовет). На должность главнокомандующего всеми вооруженными силами РСФСР Ватсетис. Для руководством государством был создан совет рабочей и крестьянской обороны, который проводил мобилизацию всех сил и средств.

\subsection{Л. Д. Троцкий.}

Вернувшись в Россию в 1917 году, Л. Д. Троцкий (Бронштейн) возглавил пост наркомвоенмора, на 2 всероссийском съезде советов рабочих и солдатских депутатов получает должность наркома иностранных дел. Однако, после провала  мирных переговоров с германией в феврале 1918 года, подаёт в отставку и вскоре получает должность председателя реввоенсовета. 

С формированием РККА, Троцкий фактически становится первым главнокомандующим. Он активно ездит по фронтам гражданской войны, лично появляется на передовой, активно и безжалостно наводит порядок в армии. Л. Д. Троцкий руководит красной армией на южном фронте, самом горячем на момент 1919 года.

Троцкий также руководил обороной Петрограда осенью 1919 года. Город был охвачен паникой перед белогвардейскими танками, однако Лев Давыдович сумел поднять боевой дух и разгромить белогвардейские войска Юденича.

\subsection{Идеология Белого движения и создание Добровольческой армии.}

Идеология: «Свержение диктатуры большевиков», «Установление власти выборным способом», «Недовольство переворотом большевиков». (нужна инфа)
«Белые» военные силы в основном состояли из офицерства императорской армии, не поддержавшей переворот, и казачества. В конце 1917 года первый крупный центр белого движения формируется в Новочеркасске. Там же формируется добровольческая армия, которую возглавил генерал Корнилов (который в августе 1917 года пытался провести переворот). Кроме того в донской области казаками командует Каледин. На востоке России также формируется несколько центров.
Снабжались белые армии к инквизиции продовольствия и припасов. Какое-то время им оказывали денежную помощь страны Антанты.
Программа белой армии не была единой, так как против большевистского правительства выступили самые разные слои населения, однако основными были «Свержение диктатуры большевиков» и «Установление власти выборным способом». Кроме того, большая часть белого движения выступала за «Единую и неделимую Россию на территориях бывшей Российской	 империи».

\subsection{Важнейшие антибольшевистские правительства: КомУч, Директория, правительственные структуры А. В. Колчака, А. И. Деникина и Н. Н. Юденича.}
После разгона учредительного собрания в январе 1918 года, многие депутаты съезжаются в Самару, где организуют комитет учредительного собрания (КомУч). Так как их избрало населения, они должны взять власть в свои руки, по их мнению. Там же формируется народная армия. Плюс в Поволжье к ним примкнули части полковника Капеля. Формируется Уральская армия, сибирская армия, в Забайкалье также формируется армия под командованием Семёнова. 

В Сибири трёхсоттысячную белую армию сформировал адмирал А. В. Колчак, провозгласивший себя «верховным правителем Российского государства».

В июле 1919 г. А. И. Деникин захватил Украину и провёл там мобилизацию. В связи с последующими действиями Деникина был образован Южный фронт под командованием А. И. Егорова.

В Эстонии белогвардейскую армию возглавлял генерал Н. Н. Юденич.

\subsection{Чехословацкий мятеж}

В мае 1918 г. восстали солдаты чехословацкого корпуса. В нем были собраны
военнопленные славяне из австро-венгерской армии, изъявившие желание участвовать в войне
против Германии на стороне Антанты. Корпус был отправлен Советским правительством по
Транссибирской магистрали на Дальний Восток. Предполагалось, что далее он будет доставлен
во Францию. Восстание привело к свержению советской власти в Поволжье и Сибири. В
Самаре, Уфе и Омске были созданы правительства из кадетов, эсеров и меньшевиков. Их
деятельность опиралась на идею возрождения Учредительного собрания, выражалась в
противостоянии как большевикам, так и крайне правым монархистам. Эти правительства
просуществовали недолго и были сметены в ходе Гражданской войны.

В июне 1918 г. против восставшего чехословацкого корпуса и антисоветских сил Урала и
Сибири был образован Восточный фронт под командованием И. И. Вацетиса (с июля 1919 г.—
С. С. Каменева). В начале сентября 1918 г. Красная Армия перешла в наступление и в течение
октября — ноября вытеснила противника за Урал. Восстановлением советской власти в
Приуралье и Поволжье завершился первый этап Гражданской войны.

\subsection{Зелёное движение "--- третья сила в Гражданской войне}

Существовало множество нерегулярных формирований, не связанных ни с белыми, ни с красными, что обуславливало затягивание войны.

«Зелёные», «зелёные партизаны», «Зелёное движение», название нерегулярных, преимущественно крестьянских и казачьих вооружённых формированпй, противостоявших большевикам и белогвардейцам в годы Гражданской войны в России.

Массовый характер это явление приобрело летом 1919, когда белогвардейцы развернули принудительную мобилизацию населения в занятых интервентами и белогвардейцами р-нах (особенно на Сев. Кавказе, в Дагестане, Крыму).
Оно стало одной из форм сопротивления населения антибольшевистским режимам.
«Зелёные» из числа рабочих и крестьян организовывались в «красно-зелёные» партизанские отряды и вели вооруж. борьбу за восстановление сов. власти.

«Красно-зелёные» партизанские отряды Крыма после объединения в августе 1920 в Повстанческую армию Крыма участвовали в боях против армии П.Н. Врангеля.
Белогвард. командование также создавало в тылу Кр. армии из остатков белых армий отряды «бело-зелёных». Так, в 1920 на Сев. Кавказе была создана т. н. «Армия возрождения России».

(Источник: Абациев Д.П. Революция и Гражданская война в России 1917-1923, Том 2)

\subsection{Основные фронты и военные действия на них в 1918 г.}

Началом войны можно считать октябрьский переворот, так как именно с того времени начались вооруженные столкновения. Совнарком, понимая, что возникают противостоящие силы, выпускает обращение ко всему населению, в котором призывает встать в ряды советской армии, а те, кто будут причастны к противостоящим образованиям, объявляются контрреволюционерами.
Общее командование красной армией на южном фронте было поручено Антонову-Авсеенко. Он действует против донской добровольческой армии и войск центральной Рады в Украине. Захватывают Таганрог и Новочеркасск, вытесняя оттуда донские части. Генерал Каледин после поражения застрелился, а остатки добровольческой армии уходят в Сальские степи. Они вынуждены отходить от Ростова на Кубань. Происходит «Ледяной поход» добровольческой армии. Корнилов погибает вследствие штурма Екатеринодара и командование переходит к Деникину. Используя поддержку казачества, он смог пробиться на Дон с остатками добровольческой армии и получить там дополнительную поддержку. 
В соответствии с перемирием с четверным союзом совнарком признает Украинскую народную республику и её право отделиться от России. Это продлилось недолго и в январе 1918 года правительство большевиков обвиняет Центральную Раду в поддержке германским войск, поддержке Каледина на дону и объявляет войну. 26 января 1918 года большевистская армия занимает Киев.
В закавказье действует перемирие с Турцией. В феврале 1918 года правительство Турции пользуется (чем?!) и начинает наступление.
В конце 1917 года в Бессарабии провозглашена Молдавская народная республика, которая не хочет устанавливать отношения с правительством Ленина. Там же проходит Румынский фронт, где доминируют большевики. Начинаются столкновения большевиков и молдавской армии. На стороне Молдавии присоединяется Румыния. Она занимает Кишинёв и он становится частью Румынского государства.
Зимой 1918 года ведутся крупные боевые действия на южном урале. Там действует оренбургское казачье войско под командованием Бутова. Сюда советское правительство присылает часть солдат и моряков. Советские войска под командованием Блюхера действуют успешно, занимают Оренбург, наносит поражение Бутову и отбрасывает их в Сибирь.
После заключения Брестского мира события развиваются ещё стремительнее.

\subsubsection{Боевые действия в конце 1918 г.}

Основные события происходят на восточном фронте. Васетис ведет войну с чехо-словацким корпусом, белой армией на поволжье.
Добровольческая армия Деникина набирает мощь на юге. 
Ведутся тяжелые сражения на царицинском и воронежском направлении. Красная армия сумела отбросить армию Деникина с Кубани.
Формируется северный фронт гражданской войны. При поддержке интервентов формируется добровольческая армия, которая действует у Петрограда.
Осенью 1918 года происходят изменения на международной арене. Турция подписывает со странами Антанты перемирие, по условиям которого Англичане занимают Баку. Тогда же в ноябре 1918 года вспыхивает восстание в Германии и Австро-Венгрии. Они решают выйти из войны. 11 ноября 1918 года подписывается пока ещё перемирие в городке Комп-Ем. По условиям этого перемирия Германские и Австро-Венгерские войска освободят оккупированные территории.
В ноябре-декабре 1918 года новые десанты союзников высаживаются в Причерноморье, Южной Осетии, Владивостоке.
На востоке страны главным центром становится Омск, где на главный план выходит фигура адмирала Колчака. Он захватывает власть, распускает Омское правительство, объявляет себя верховным правителем России и верховным главнокомандующим армиями, противостоящие большевикам. В декабре 1918 года он подписывает союзное соглашение с странами-интервентами.
13 ноября 1918 года выходит постановление об аннулировании Брестского мира.

\subsection{Иностранная интервенция} (нужна доработка)

Началось вторжение иностранных государств на территорию страны. Армия великобритании и Франции высаживаются в нынешнем Мурманске, Япония и США начинают интервенцию на востоке(, в связи с чем создаётся дальневосточная республика), Турция так же вводит свои войска.

Весной 1918 г. началась иностранная интервенция. Германские войска оккупировали
Украину, Крым и часть Северного Кавказа. Румыния захватила Бессарабию. Страны Антанты
подписали соглашение о непризнании Брестского мира и будущем разделе России на сферы
влияния. В марте в Мурманске был высажен английский экспедиционный корпус, к которому
позднее присоединились французские и американские войска. В апреле Владивосток был занят
японским десантом. Затем на Дальнем Востоке появились отряды англичан, французов и
американцев

\subsection{Красный и белый террор}

Применение террора обеими сторонами стало одной из характерных черт гражданской войны.
В июле 1918 года начинается выступление левых эсеров. Они осуществляют убийство посла Германии, считая, что это может вызвать недовольство со стороны Германии, однако это приводит лишь к вооруженным столкновениям с большевистской армией.
Уральский Ревком в связи с выступлениями по согласованию с Москвой инициирует расстрел царской семьи в ночь на 18 июля. 
В августе 1918 года левые эсеры предпринимают попытку устранить членов временного правительства.
Создаётся институт заложников. Людей, которых относят к эксплуататорским классам, берут в заложники, сажают в тюрьмы/помещения, вынуждая их заплатить контрибуцию. 
Точное число жертв красного и белого террора посчитать невозможно, считают, что со стороны красного террора погибло ~1,2 млн человек, со стороны белого ~300 тыс.
