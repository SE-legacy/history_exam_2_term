\section{(ВОЗМОЖНО СТОИТ ДОРАБОТАТЬ, МБ НУЖЕН НЕБОЛЬШОЙ РЕВОРК) Билет 13. Оформление тоталитарной системы и культа личности И. В. Сталина. Репрессии 1930"=х гг}

\subsection{Сталин и его оппозиция}

Сталину противостояла оппозиция из Бухарина, Рыкова и Томского. Они выступали за стабилизацию отношений рыночного хозяйства на основе НЭП. Оппозиция выступали за умеренные темпы индустриализации.

Но Сталинская группа мыслила по-другому. В эту группы выходили Куйбышев (председатель народного хозяйства), Ворошилов (нарком обороны), Орджоникидзе, Молотов, Калинин (возглавлял ВЦИК). Они считали, что требуется максимальная концентрация ресурсов в тяжелой промышленности. Они же настаивали на форсированной коллективизации. Эти споры в 1928-1929 году закончились в ноябре 1929 году, когда центральный комитет партии поддержал сталинскую группу. После этого из политбюро выводятся сначала Бухарин, а затем Рыков и Томский.

\subsection{Рост партии и исключение не лояльных.}

В это время партия численно растёт (в 1926 году в ВКПб было около 1 млн. чел., то в 1929 году было 2,4 млн. чел.) Всё реже созываются съезды партии. До 1925 года съезды происходили каждый год, а затем интервал увеличивается на год. 

Диктат партийной верхушки поддерживался чистками партий, которые означали исключение из партии тех, кто был не лоялен сталинскому курсу. Постепенно разворачиваются репрессии и против самих членов партии.

Расширяются партийные диктаты на производстве. В колхозах и совхозах создаются партийные ячейки. Под их ведомом находятся все предприятия, и в том числе советы.

\subsection{Новая конституция 5 декабря 1936 года.}

5 декабря 1936 года на 8 всесоюзном съезде принята новая конституция. Она запрещала эксплуатацию человека человеком. Конституция представила всем людям широкие права и свободы. Свобода слова, совести и прочее. Союзными республиками становятся Казахстан и Киргизия, а азербайджан, грузия входят в состав СССР.

Но эта конституция не соблюдалась. Все формальные нормы легко нарушались властями. Все, кто критиковали курс партии и лично Сталина, репрессировались.

\subsection{Закон об <<Охране социалистической собственности>> и указ об ....}

7 августа 1932 года принимается закон об «Охране социалистической собственности». Этот закон вводил за воровство социалистической собственности расстрел с конфискацией имущества, вне зависимости от размера украденного.

В 1940 году принят указ, по которому самовольный уход с предприятия карался заключением под стражу. А прогул"--- исправительные трудовые работы.

\subsection{ГУЛаг и прочее.}

Для противодействия антисовестким силам создаётся система управления лагерями при ОГПУ, а в 1931 году создается Главное управление лагерями – ГУЛаг. ОГПУ в 1934 году было преобразовано в НКВД (народный комиссариат внутренних дел) при СССР, первым комиссаром стал Генрих Ягода. НКВД занимался борьбой с уголовной и политической преступностью. Создано «Особое совещание» при НКВД, которое рассматривало дела по упрощенной схеме и выносило приговор.

1 декабря 1934 года обычный коммунист Николаев убил С.М. Кирова. Тогда же ВЦИК принимает постановление о новом порядке ведения террористических дел. Дела рассматриваются не более 10 дней, дела слушаются без присутствия сторон, помилования и обжалования не допустимы, приговор приводить в исполнение немедленно.

Репрессии проводились на фоне активного культа личности сталина. Образ вождя возвеличивался, повсеместно превозносились его качества и заслуги. Весь советский народ был обязан партии.

\subsection{Сфабриковывание дел(НКВД).}

С конца 1930"=х разворачиваются громкие процессы, сфабрикованные НКВД. Они не стеснялись пренебрегать никакими правилами и положениями. Применялись пытки, ложные доносы без проверки, обман. 

Цель у органов была одна – найти и обвинить. Процессы организовывались над советскими лидерами публично и освещались. Процесс антисовесткого объединенного троцкистского и зиновьевского движения, главными обвиняемыми были Зиновьев и Каменев. В 1937 году происходит процесс о параллельном антисоветском троцкистском центе, по которому обвиняются Педаков, Сокольников, Радек. В 1938 году антисоветский троцкистский блок "--- Бухарин, Ягода, Рыков.

\subsection{Репрессии 1930"=х годов}

мб и не нужно, хз, потом мб удалим (лучше иметь и не нуждаться)

\subsubsection{\textbf{Подробнее про НКВД.}}

Репрессии эти коснулись и партийного аппарата. В том числе это коснулось и красной армии.

Тройка НКВД – это чрезвычайное судебное образование, состоявшее из начальника управления НКВД в данном регионе, секретаря обкома ВКПб и прокурора.  

Возглавлял НКВД Ежов до 1939 года, а потом был смещен, арестован и расстрелян. Позднее НКВД возглавил Берия. Но репрессии не прекратились после 1939 года. С 1939 по 1953 было осуждено 1100 тысяч человек, из которых расстреляно 54 тысячи. 

\subsubsection{\textbf{Подробнее про репрессии.}}

Репрессировались не только конкретные люди, но и члены их семей. В личных делах начали появляться аббревиатуры ЧСВН (член семьи врага народа) или ЧСИР (член семьи изменника Родины). Был создан специальный лагерь «АЛЖИР» - Акмалинский лагерь жён изменников Родины, туда ссылали жён осуждённых.

От репрессий пострадали все слои населения, но больше всего интеллигенция, деятели культуры, технические инженеры, религиозные деятели. 

С 1930-х годов борьба с религией приобретает новые обороты. «Союз воинствующих безбожников» во главе с Е. Ярославским. Они пропагандировали идеи атеизма и препятствовали работе церкви.

\subsection{Итоги}

Репрессии серьезно затронули верхушку партии. На 17 съезде было 1200 делегатов, из которых 1100 было расстреляно. Из центрального коммитета из 139 претендентов 98 человек арестовано и расстреляно. 
В 1937-1938 году происходила чистка армии, когда погибло 45\% высшего руководства армии.
Первыми маршалами были Тухачевский, ворошилов, Будёный, Плюхер, Егоров. 