\section{ Билет 21. Политическое развитие страны в период «Оттепели» 1953—1964 гг. Н.С, Хрущев, Отношения СССР с другими странами.}

\subsection{Советское государство после смерти Сталина: столкноваение сил в политическом руководстве.}

Смерть Сталина наступила 5 марта 1953 года и поставила вопрос о лидерстве в партии и государстве.

Состоялось необычное совместное заседание всех трёх органов. Пристутсвовали ЦК партии, совет министров и президиум Верховного совета. Были потеряны генсек партии и председатель совета министров.

На этом заседании председателем совета министров стал Георгий Максимилианович Маленков, Лаврентий Павлович Берия сохраняет прежние должности и становится заместителем Маленкова. Министерство обороны возглавил Николай Александрович Булгарин. Булгарин и Каганович также стали заместителями Маленкова. Ворошилов возглавил президиум Верховного совета, Никита Сергеевич Хрущёв возглавил генеральный секритариат ЦК партии.

Кроме того, на этом заседании МГБ поглощается структурой МВД, во главе которого стоит Л. П. Берия.

\subsubsection{Поражение Берии в борьбе за власть}

Все боялись Лаврентия Павловича Берия, потому что он возглавлял репрессивный аппарат и обладал чувствительной информацией, которую он мог бы пустить вход против своих соперников. Поэтому произошёл заговор, на заседании ЦК партии в адрес Берия прозвучали различные обвинения. Жуков с группой вооружённых людей арестовали Берию на этом же заседании, позже были арестованы его ближайшие сторонники. Берия был приговорён к расстрелу, который состоялся в декабре 1953 года.

Устранение Берия укрепило положение Хрущёва. Чтобы избежать появления нового "Берии" было решено провести реформу. Вместо МВД был создан КГБ, подчиняющийся совету министров СССР.

\subsubsection{Падение Маленкова}

К концу 1954 года резко обострилась борьба между ближайшими соратниками Сталина. Столкнулись старые сталинисты и Хрущёв со своими сторонниками в партии и в армии. Хрущёв добился отставки Маленкова в 1955 году, его заменил Булгарин --- человек Хрущёва, --- а пост военного министра занял Жуков.

\subsubsection{Упрочение позиций Н. С. Хрущева}

Влияние Хрущёва в новом составе руководства страны становилось преобладающим. Заседания президиума ЦК становятся регулярными, там обсуждаются не только партийные, но и экономические проблемы. Берётся курс на реабилитацию жертв сталинизма. Именно за уменьшение контроля и строгости со стороны государства эта эпоха была прозвана оттепелью.

В марте 1953 года предложение прекратить пропаганду культа личности Сталина внёс Маленков, буквально через несколько дней после смерти Сталина. Он же снизил налоги с крестьянства, аннулировал многие долги колхозов. Развивалась торговая сеть страны. Эти меры были направлены на повышение популярности Маленкова и новой комбинации вождей.

Либерализация проявлялась не только в этом, она коснулась и репрессивных органов. Публикуется сообщение МВД СССР о ложности обвинений против кремлёвских врачей. Вскоре последвоало официальное осуждение органов госбезопасности, занимавшихся Мингрельским делом.

Поворотный этап формирования нового политического курса --- XX съезд КПСС.

\subsection{Хрущев "--- генеральный секретарь партии. 20 съезд КПСС. Критика культа личности сталина}

После смерти Сталина (Генеральный секретарь партии) Хрущев занимает эту должность и становится первым секретарем партии.
В 1953-1954 гг. начинается процесс либерализации. Но коренной перелом произошёл на 20"=м съезде КПСС в феврале 1956 года. 

На этом съезде Хрущев сделал доклад «О культе личности и его последствиях». В нём Хрущев разоблачил преступные действия Сталина. Вся вина за преступления и репрессии возлагалась на Сталина и лица его ближайшего окружения. В докладе всячески подчеркивалось, что доклад не мог изменить природу самого социализма. И поскольку наиболее идеозные фигуры сталинского режима либо ушли из жизни, либо были отстранены, то это снимало ответственность за преступления сталинизма от остального руководства партии. Оно оказалось вне критики. Тогда процесс либерализации еще сильнее ускорился.

\subsection{Реабилитация  ряда депортированных народов.}

Процесс реабилитации жертв репрессий начался еще до 20 съезда, но после него пошёл еще быстрее. С 53 по 56 год было реабилитировано более 700 тысяч человек. Многие реабилитировались посмертно. Восстанавливали доброе имя репрессированных еще в 20-30е годы. Процесс реабилитации не коснулся Бухарина, Рыкова, Томского, Троцкого. При этом Тухачевского реабилитировали. Первые предлагали изменить ход построения государства, из-за чего их реабилитация – это признание их правоты. 

\subsubsection{\textbf{Абзац про национальную политику.}}
Снимаются почти все обвинения с народов, переселенных в период репрессий, восстанавливается их автономия. Но автономия не коснулась поволжских немцев. Этот процесс продолжался до конца 60-х годов, а затем волна реабилитаций сходила на нет.

В это же время происходят изменения в уголовном законодательстве. Статья за антисоветскую деятельность все еще присутствует, но отменено понятие «враг народа», повышен возраст уголовной ответственности с 14 до 16 лет, запрещено насилие во время следствие, обязательное присутствие обвиняемого на процессе, адвокатская защита.

\subsection{22 съезд КПСС и программа <<развернутого строительства коммунизма>>.}

\subsubsection{\textbf{22 съезд КПСС.}}

Что касается строительства социализма, то эта цель не снимается. Более того на 22 съезде в 1961 году Хрущев объявил о построении коммунизма аж к 1980 году. 

\subsubsection{\textbf{Условия для построения коммунизма в социалистических странах.}}

\begin{enumerate}
    \item Первое для коммунизма требуется превысить уровень развития перед капиталистическими странами (первое место по продукции на душу населения).
    \item  Второе "--- ликвидация различий между классами, общество полностью однородное.
    \item Ликвидация существенных различий между городом и деревней.
    \item Ликвидация развития между физическим и умственным трудом.
    \item Возрастание общности наций.
    \item Развитие черт человека коммунистического общества:
        \begin{itemize}
            \item высокая идейность
            \item широкая образованность
            \item моральная чистота
            \item физическое совершенство
        \end{itemize}
    \item Последнее важное условие "--- в нём все граждане принимают участие в управлении общественными делами на основе широкой социалистической демократии

\end{enumerate}

\subsection{Попытка престановки кадров. Отстранение от должности Хрущева.}

Ради такого Хрущев планирует перестановку кадров, что вызывает противодействие аппарата, который не хочет отдавать свои места. Аппарат ополчился на Хрущёва. Были высказаны в упрек Хрущеву экономические трудности 1960-х годов. В 1964 году верхушка власти обвиняет его в волюнтаризме, субъективизме и путём демократического голосования был отстранен от должности первого секретаря цк и отправлен на пенсию. С этого момента во главе советского государства находится триумвират: Косыгин, Подворный, Брежнев.

\subsection{Внешняя политика.}

Этот период характеризуется увеличением связей с западом и мыслями о возможном сокращении вооружения. Нормализуются отношения с Югославией. В 1953 году решается Корейский вопрос заключением перемирия, который ознаменовал распад Кореи на две части: КНДР и Корейской республики (на юге).

В 1955 году решена судьба Австрии. В 1955 году Хрущевское руководство дает разрешение на восстановление Австрии и завершается её оккупация.

\subsubsection{\textbf{Венгерские восстания.}}

С другой стороны Советский союз пытается сохранить контроль над всеми странами. В 1953 и 1956 году происходят народные восстания против советского строя в ГДР и Венгрии соответственно. 

\subsubsection{\textbf{Варшавский договор.}}

В 1955 году заключается Варшавский договор – военно политический блок, который должен противостоять НАТО. В него входят: Советский Союз, Польша, Чехословакия, Венгрия, Румыния, Болгария, ГДР, Албания.

\subsubsection{\textbf{Карибский кризис.}}

Главным направлением внешней политики этой эпохи является отношения с США. В 1957 году Советский союз впервые испытывает межконтинентальные баллистические ракеты. Это серьезно меняет расклад сил и Американцы пытаются установить военные базы все ближе к границам СССР, например в Турции (входит в НАТО). В 1962 году разразился карибский кризис. В этом году СССР пытается переправить ракеты на Кубу. Американцы, узнав об этом, попытались этому воспрепятствовать. 

Мир оказался на пороге ядерной войны. Этого удалось избежать благодаря личным переговорам Хрущева и Кеннеди. Обе стороны пошли на компромисс: СССР убирает ракеты с Кубы, а США с Турции. 