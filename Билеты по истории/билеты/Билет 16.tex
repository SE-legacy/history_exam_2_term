\section{Билет 16. Великая Отечественная Война: причины, планы сторон, соотношения сил. Начальный этап войны: июнь 1941-июль 1942.}

\subsection{Причины войны}
\begin{enumerate}
    \item Стремление гитлеровского правительства обеспечить мировую гегемонию Германии и расширение жизненного пространства для немцев. С точки зрения Гитлера, СССР - препятствие для расширения мировой гегемонии, к тому же обладающее ценными территориями.
    \item Идеологическая причина. С момента прихода к власти Гитлера в Германии господствовали антикоммунистические идеи и теория расового превосходства. Славяне - неполноценный народ. Евреи, цыгане и коммунисты подлежат уничтожению.
    \item Оценка военного потенциала СССР как крайне низкого.
\end{enumerate}

\subsection{План <<Барбаросса>>}

План нападения на СССР получил название в честь средневекового императора Барбаросса. Германия должна использовать тактику блицкрига, нанести сокрушительное поражение, подчинить важнейшие центры СССР - Москву, Ленинград, Украину, Кавказ и нижнюю Волгу. 

\subsection{План <<Ост>>}

После плана <<Барбаросса>> должен был вступить в силу генеральный план Ост. По нему с территории Польши и запада СССР нужно выселить до 80\% местных жителей. Часть уничтожить (коммунистов, евреев и цыган), а славян оттеснить за Волгу. Оставшиеся 20\% должны обслуживать интересы немцев, колонизирующих освободившиеся территории.

\subsection{Начальный этап войны}

22 июня 1941 года на рассвете Германия атакует западные регионы Советского Союза. Начинаются бомбардировки всех военных объектов и крупнейших городов.

Группировка сил фашистского блока для удара по СССР насчитывала 5,5 млн человек. Большое количество советской техники не было готово к войне. Морские рубежи защищали Северный, Балтийский и Черноморский флоты. Части Красной армии не были полностью укомплектованы даже личным составом.

Сразу после вторжения для общего военного руководства создаётся Ставка верховного главнокомандования, её главой и верховным главнокомандующим стал лично Сталин. Он также стал председателем Совнаркома и наркома обороны. В Ставку вошли Молотов (до 1941 также председатель Совнаркома, теперь только председтаель совнаркома иностранных дел), Жуков, адмирал Кузнецов, Ворошилов, Тимошенко.

В стране объявляется мобилизация. Большое количество людей идут добровольцами. Нарком Молотов призвал дать отпор врагу. 3 июля 1941 года Сталин открыто заявил о смертельной угрозе и призвал всех граждан спасти Отечество.

В первые дни войны гитлеровские войска, создав на главных направлениях ударов подавляющее превосходство сил, внезапно атаковали и, прорвав оборону, завоевали господство в воздухе. Попытки советского командования оказать противодействие затормозили сопротивление врага. Пограничные сражения продолжались до 29 июня и закончились поражением войск Западного и Северо-западного округов. Самый известный подвиг этой стадии - оборона Брестской крепости, продолжавшаяся до 23 июля 1941 года. Этот случай и несколько подобных не могли оказать серьёзного влияния на ход военных действий.

К середине июля враг оккупировал Литовскую и Латвийскую ССР, частично Эстонскую, западные регионы Белорусской и Украинской ССР. Общее продвижение вглубь страны - на 500-600 километров.

СССР терпел большие потери личного состава и техники. У Германии - только в технике (например, до 40\% танков).


\subsection{Переход страны на военные рельсы}
30 июня 1941 года была создана структура для руководства экономикой - Государственный комитет обороны, возглавленный Сталиным. В него вошли Молотов, Ворошилов, Маленков (правая рука Сталина в партийных делах), народный комиссар внутренних дел Берия и ещё несколько человек.

Сталин сосредоточил в своих руках всю партийную и военную власть фактически и формально.

Начинается перевод всего народного хозяйства на военные рельсы. Важнейшие предприятия с запада СССР начинают в срочном порядке эвакуироваться в Поволжье, на Урал, в среднюю Азию и в Сибирь.

За вторую половину 1941 года было перенесено более 2600 предприятий вместе с запасами продовольствия, скотом, техникой, материально-культурными ценностями и сотрудниками предприятий. Это снижало темпы производства военной техники, поэтому в первые месяцы войны нехватка вооружения ощущалась особенно остро.

18 июля 1941 года ЦК партии принял постановление об организации борьбы в тылу германских войск. Все комитеты партий должны были развернуть в тылу врага широкую сеть подпольных партийных и комсомольских организаций.

В 1942 году институт политруков упраздняется для усиления единоначалия в войсках. На начальном этапе войны вышел приказ Ставки №270 от 16 августа 1941 года, который объявлял всех попавших в плен изменниками Родины. Люди, сдавшиеся в плен, рассматривались как предатели. Цель приказа - остановить массовую сдачу в плен (в возникавших котлах сдавались целыми соединениями, это нужно было остановить).

12 сентября вышла директива ГКО о создании заградительных отрядов, препятствующих отступлению советских войск.
\subsection{Бои за Киев, Смоленск, Блокада ленинграда}
Группа армий "Центр" была впервые остановлена под Смоленском. Оборонительные сражения разгорелись на подступах к Ленинграду. Лишь в августе войска "Севера" смогли подойти к южным окраинам Ленинграда. В это же время финские войска наступали с севера и заняли перешейки между Финским заливом, Онежским и Ладожским озером. 8 сентября 1941 года началась блокада Ленинграда, продолжавшаяся 872 дня.

Войска юго-западного и южного фронтов отступали вглубь СССР. Пока ещё Киев и Одесса удерживались советскими войсками. После того как основные усилия германское командование перенесло на Украину, Киев был взят, войска попали в окружение в сентябре 1941 года.

Этот успех позволил развернуть дальнейшее наступление в сторону Донбасса. В октябре-ноябре враг прорывается в Крым, где его сковала героическая оборона Севастополя.

\subsection{Битва за Москву, крах Блицкрига}

Основные усилия теперь направлены на захват столицы. Битва за Москву продолжалась с 30 сентября 1941 года по 20 апреля 1942 года. У германского командования существовал план "Тайфун" - охвата Москвы с двух сторон. Группа армий "Центр" прорывает оборону советских войск, окружает их значительную часть. На ближайших подступах к Москве силами Калининского фронта (Конев), Западного фронта (Жуков), Брянского фронта (Захаров) враг был остановлен. Москва объявлена на осадном положении, ведётся эвакуация предприятий, предполагается возможная эвакуация советского правительства, подготовлена база советского правительства в Куйбышеве.

Поздней осенью 1941 года Москва оказалась в критическом положении. Ударные группировки противника обходили Москву с севера и с юга, но к зиме группа армий "Центр" была обескровлена и вынуждена перейти к обороне. Советское командование пыталось усилить Калининский, Западный и Брянский фронты.

К концу 1941 года в результате тяжелейших боёв отступила местами до 1200 километров. В руках агрессора оказались важнейшие экономические районы СССР. Ценой колоссальных жертв советские войска заставили вермахт перейти к стратегической обороне. Потери СССР - до 5 миллионов погибшими, ранеными и пленными, до 20 тысяч танков, до 17 тысяч самолётов и другой техники. Потери Германии - до 750 тысяч личного состава, незначительно меньшее количество техники.

В начале декабря 1941 года советские войска, получив спешно пришедшие подкрепления с востока, развернули контрнаступление под Москвой. Враг отбрасывается на 100-150 километров на запад. Ставка принимает решение о наступлении советских войск по всему фронту. Главным итогом Московской битвы стало спасение столицы и провал блицкрига.

Руководство Германии оказалось перед неизбежной перспективой затяжной войны.

\subsection{Создание Антигитлеровской коалиции}
Страны Атнигитлеровской коалиции. Главные действующие страны - Советский союз, Великобритания и США. Уистон Черчилль, узнав о нападении на СССР, выразил поддержку Советскому Союзу. Франклин Рузвельт пообещал оказать всю возможную помощь Советскому Союзу. Осенью 1941 года в Москве состоялась конференция с участием представителей сран Анитигитлеровской коалиции о поставке вооружений.

\subsection{Военные действия весной -- летом 1942 г.}

Ставка переоценила наступательные возможности и силы советской армии. Германское командование с весны 1942 года основные усилия сосредоточило на южном направлении, делая ставку на захват нефтедобывающих районов Кавказа, Дона, Кубани, нижней Волги. Советское командование пытается снабжать войска новыми видами вооружения. Сталин и Ставка настаивают на скорейшем разгроме Вермахта и освобождении всех территорий страны.

В ходе наступления немецких войск в мае 1942 года в Крыму был разгромлен Крымский фронт. Потери только здесь составили 170 тысяч человек. Севастополь был потерян. Войска юго-западного и южного фронтов от Изюма к Харькову потерпели поражение.

Создаётся центральный штаб партизанского движения при Ставке верховного главнокомандующего. Поддерживается связь с большими партизанскими отрядами. Им всеми возможными способами и методами поставляется оружие и боеприпасы.

Под ударами вермахта войска советских фронтов оставляют Донбасс, немецкие войска выходят в большую излучину Дона.

Сталинград - важнейший транспортный узел, в котором к тому же находится сталинградский тракторный (теперь танковый) завод.

\subsection{Приказ №227}

17 июля 1942 года начинается оборонительный период Сталинградской битвы. Изадётся приказ НКО №227 от 27 июля 1942 года "Ни шагу назад!", запрещающий отступать без приказов командования. Трусы и паникёры подлежат уничтожению на месте.