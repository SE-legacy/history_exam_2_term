\section{Билет 19. СССР в послевоенный период (1945--1953): экономика, социум, внешняя политика.}

\subsection{Что-то типа небольшого вступление для понимания (хз как по другому назвать, но оно будто бы реално надо)}

После ВОВ почти вся европейская часть России оказалась разорена. => главной задачей внутренней политики СССР в 1945--1953 было восстановление народного хозяйства.

Оно началось ещё в 1943 и расширялось по мере изгнания оккупантов. После победы над Германией ГКО (гос комитет обороны) постановил перевести часть оборонных предприятий на выпуск товаров для населения; в июне 1945 принят закон о демобилизации армии.

4 сентября 1945 года упразднён ГКО. Функции управления страной сосредоточились в СНК. 10 февраля 1946 года состоялись выборы в ВС СССР 2"=го созыва. 15 марта принят закон о преобразовании СНК в Совет Министров СССР.

\subsection{Послевоенная (четвертая) пятилетка.}

Целью четвертой пятилетки 1946 — 1950 являлись форсированные темпы восстановления за 5 лет. (американские специалисты заявили, что для восстановления потребуется не менее 100 лет). 

Приоритетным направлением было восстановление и развитие тяжелой промышленности. Это решение диктовалось необходимостью укрепления обороноспособности страны в условиях холодной войны, к тому же тяжелая промышленность является базовой для развития всех отраслей. 

\subsubsection{\textbf{Возобновлению нормального режима труда на предприятиях:}}

\begin{itemize}
    \item Отмена обязательных сверхурочных работ 
    \item Восстановление 8"=часового рабочего дня
    \item Восстановление ежегодных оплачиваемых отпусков
\end{itemize}

\subsubsection{\textbf{План предусматривал:}}

\begin{enumerate}
    \item Восстановление разрушенного войной народного хозяйства.
    \item Увеличение выпуска промышленной продукции по сравнению с довоенным уровнем (на 48\%, сельскохозяйственной "--- на 23\%). 
\end{enumerate}

Конверсия промышленности, сопровождавшаяся сокращением объёмов производства на 17\%, завершилась в 1946.

\subsubsection{\textbf{Конец пятилетки:}}

\begin{enumerate}
    \item В октябре 1947 в экономике уже был превышен довоенный уровень.
    \item Через 4 года и 3 мес выполнены плановые задания.
    \item К концу пятилетки увеличился выпуск промышленной продукции на 73\% по сравнению с 1940.
    \item За пятилетку введено в строй 6200 восстановленных и вновь построенных предприятий.
\end{enumerate}

\subsection{Аграрный сектор.(как я понял восстановление народного хозяйства и деревня в послевоенные годы и про голод в 1946--1947 (сам в ахуе)(И да это все еще отноистся к той пятилетке)}

\subsubsection{\textbf{Деревня в послевоенные годы.}}

Аграрный сектор экономики вышел из войны ослабленным. В деревне почти на треть уменьшилось трудоспособное население. На протяжении нескольких лет на село не поставлялась новая техника. Поголовье лошадей в колхозах сократилось за годы войны более чем наполовину. 

Поскольку руководство страны решило начать восстановление хозяйства с тяжёлой промышленности, постольку интересы села были подчинены выполнению этой задачи. Помощь города селу была минимальной.

\subsubsection{\textbf{Голод 1946--1947 гг.}}

Положение осложнилось тем, что первый послевоенный год оказался неблагоприятным по погодным условиям. Валовой сбор зерновых в 1946 был в 2,2 раза меньше, чем в 1940. Начавшийся голод привёл к гибели 770,7 тыс. чел. и вызвал массовый отток сельского населения в города. Острота продовольственной проблемы была снята хорошими урожаями в 1947 и 1948.

\subsubsection{\textbf{Еще про деревню и народное хоз"=во.}}

Было проведено укрупнение мелких колхозов, при уменьшении общего числа колхозов. По официальным данным, в 1950 продукция земледелия всех категорий хозяйств практически достигла уровня 1940, а показатели животноводства оказались даже выше довоенных. Однако планы повышения уровня потребления продовольственных товаров не были выполнены.  На протяжении 1946–51 42–65\% (в некоторых районах до 90\%) совокупного дохода колхозникам приносило подсобное хозяйство, в то время как оплата по трудодням составляла 15–20\%.

\subsection{Отмена карточек и денежная реформа. Снижение розничных цен.}

В 1947 отменена карточная система, введённая в годы войны, проведена денежная реформа, носившая в значительной степени конфискационный характер. 

Одновременно с денежной реформой объявлено о снижении розничных цен на продукты питания и промышленные потребительские товары. Розничные цены на товары массового потребления к 1950 были на 43\% ниже уровня 1947. Жизненный уровень населения в целом оставался довольно низким, но имел тенденцию к повышению.

\subsection{Разработка атомной бомбы.}

Особое место отводилось оборонной промышленности, в первую очередь решению создания атомной проблемы. С этой целью был создан Технический совет под председательством Бориса Львовича Ванникова. Руководство организациями и промышленными предприятиями, решавшими атомную проблему, осуществляло Первое главное управление (ПГУ) при СНК СССР, подчинённое Специальному комитету.  

Испытание бомбы проведено 29 августа 1949 на Семипалатинском полигоне. 20 августа 1953 официально сообщено об испытании в СССР и водородной бомбы. Ведущую роль в её создании сыграли академики И. Е. Тамм, А. Д. Сахаров, В. Л. Гинзбург, Я. Б. Зельдович.

\subsection{Политический режим в последние годы жизни сталина.}

Политический режим в эти годы с формальной точки зрения почти не изменился с довоенного времени. Если в военное время основные функции выполнял Государственный коммитет обороны, то в мирное время им стал Верховный совет СССР. Но он на деле лишь оформлял решения, принимаемые верхним руководством страны. Депутаты верховного совета почти всегда единогласно утверждали все указы и законопроекты. В период между сессиями выполнял функции Президиум Верховного совета СССР. 

В состав президиума входили, как правило, высшее руководство партии, члены бюро ВКПБ. Ни один вопрос не мог быть поставлен на рассмотрение без утверждения Сталина.

Но заседания президиума становились все реже и реже, что не соответствовало конституции. В 1946 году совет народных комиссаров реформировали в совет министров. Этим пытались подчеркнуть увеличение власти. Совет министров и высший комитет партии принимали решения совместно. Председателем совета министров оставался Сталин.

Наиболее доверенными лицами сталина в это время были Г.М. Маленков, Л.П. Берия, Н.С. Хрущёв, В.М. Молотов, Н.А. Булганин.
В 1952 году изменилось название партии: С Всесоюзной коммунистической партии (большевиков) на Коммунистическую партию Советского Союза, сокращенно, КПСС. 

\subsection{Поиск внутренних врагов, система тотального шпионажа (Ленинградское дело)}

В сферу компетенции министерств, возглавляемые Берией, занимались вопросом поиска внутренних врагов. Существовала налаженная система тотального шпионажа, подавления малейшего проявления инакомыслия. Методы репрессивного аппарата совершенствовались и он становился главным регулятором общественной жизни. 

Авторитарная система нуждалась в постоянном поддерживании внутреннего недоверия и органы организовали \textbf{несколько крупных политических дел:} 

\begin{enumerate}
    \item Мингрельское дело "--- дело против администрации и партии западной Грузии. Тогда лидеров местной партийной организации обвиняли в шпиионаже, троцкизме и пр.
    \item Дело еврейского антифашистского коммитета (был создан в годы отечественной войны и занималась общественной деятельностью)
    \item \textbf{Ленинградское дело} "--- самое крупное дело. В ходе него обвинялись лидеры партийной организации в Ленинграде. Обвинялись они в создании антипартийной группы, подрывной деятельности, превращении партийного аппарата в инструмент антигосударственной деятельности. Самых «опасных» расстреляли. Было осуждено более 2 тысяч коммунистов.(насчет 2 тыс комунистов возможно это про все дела вместе)
\end{enumerate}

\subsection{Реальные враги: бандеровцы, лесная группа.(возможно можно скипнуть, но лучше почитать хотя бы)}

Эти дела больше были инсценировкой, борьбой с врагами совесткой власти. Но были и реальные враги, как правило на западе. Наиболее опасные группировки существовали на западе Украины, так называемые бандеровцы. В годы ВОВ эта организация сотрудничала с Германией. В послевоенные годы советские органы боролись с таким движением.

Еще одним очагом была Литва, и так называемая, лесная группа, которая партизанскими методами боролась с Советской властью. В ходе действий Советской армии на территории Китая, было захвачено немало деятелей белого движения. Многие из них были активными пособниками фашистов в годы ВОВ. Такими стали атаман Семенов, генерал Краснов и др.

Суд состоялся и над руководителями Российской освободительной армии под руководством генерала Власова. Как правило, фигуранты всех предыдущих дел приговаривались к смертной казни.

\subsection{Идеологические кампании.(да здесь 1.5 предложения которые нихрена не дают, но другой инфы нет)}

В идеологической жизни страны оставила след дискуссия, состоявшаяся в ноябре 1951. Она привела к написанию Сталиным книги «Экономические проблемы социализма в СССР» (1952), ставшую последней теоретической работой Сталина, которая лишь подкрепляла идеи социализма и  содержала положения о неизбежности войн между капиталистическими странами.

\subsection{Государственная политика в области культуры}

Развитие науки и культуры в послевоенное время сочеталось с контролем идей социалистической сферы. Расходы на образование и науку в это время были увеличены, так как в стране был дефицит квалифицированных кадров после ВОВ. 

Советская власть пытается стимулировать среднюю и высшую школу. Создаются крупные научные исследовательские институты: Институт прикладной физики, институт атомной энергии. Такие центры позволяли поднимать и развивать различные сферы науки. Но это были отрасли, так или иначе связанные с военными нуждами. Они должны были помочь в начинающейся холодной войне. А те отрасли, которые не были связаны с войной, либо отменялись, либо финансировались по остаточному принципу. 

Самыми крупными критикуемыми науками стали Генетика и Кибернетика, которые противоречили материалистической модели мира. Результатом стало то, что Советский союз в этих отраслях серьезно отстал от западных стран.

\subsection{Внешняя политика:}

\subsubsection{\textbf{Территориальные изменения:(в методичке этого пункта нет, но будто бы знать полезно.)}}

\begin{itemize}
    \item После окончания второй мировой имеющиеся противоречия между державами вспыхнули с новой силой. Администрация Трумана принимает план Маршала. По этому плану Америка помогала восстановлению западной экономики, выдавая кредиты, экспортируя технологии и оборудование. Им воспользовалась Великобритания, Франция, Голландия и др. А в странах, освобождаемых СССР, закреплялась сталинская советская модель управления. Советский союз заключал с каждой договор. Это Польша, Чехословакия, Венгрия, Румыния, Болгария, Албания, Югославия (Югославская федеративная социалистическая республика. Отношения поддерживалась до 1948 года)
    \item Германия не только сократила территории, но и некоторое время находилась под оккупацией 4 стран: СССР, Великобритания, Франция, Америка. Это привело к распаду Германии в 1949 году на 3 государства: Германская демократическая республика (ГДР, восток), Федеративная республика Германии (ФРГ, столица Бонн), Западный и восточный Берлин (само собой, в Берлине).
\end{itemize}

\subsubsection{\textbf{Начало холодной войны.}}

Относительная сплочённость держав-победительниц после войны наблюдалась недолго. Во время Нюрнбергского процесса 1945–-1946, при подписании мирных договоров возникали затруднения из"=за стремлений СССР и США к усилению своего влияния в мире.

В 1946 году Уинстон Черчилль в городе Фултон произнёс речь, в которой Черчилль предлагал устранить коммунистическую доктрину. Эта речь стала основой холодной войны. С этого времени в идеологии двух сторон стали употребляться такие словосочетания, как «холодная война» и «железный занавес». 

\subsubsection{\textbf{Создание НАТО.}}

Таким образом послевоенная Европа разделилась на две противодействующие стороны. В 1949 году возникает НАТО (Североатлантический альянс) на западе, а на востоке в 1955 в результате Варшавского договора будет оформлена организация варшавского договора. 

(1947 стал переломным в оформлении просоветского блока государств. Правительство СССР старалось распространить коммунизм по Европе, а целью США же было не допустить этого и усилить свою власть.)

\subsection{Совет экономической взаимопомощи.}

Экономические отношения СССР с другими странами привело к создаю Совета экономической Взаимопомощи. Это был ответ СССР на план Маршала. Через него СССР оказывал поддержку странам, где к власти пришёл просоветский режим. По масштабам и эффективности она сравниться не могла.

\subsection{Корейская война.}

Напряжение достигло апогея во время Корейской войны 1950-1953 гг. Её истоки уходят корнями во вторую мировую войну. В результате произошёл раскол. Война с переменным успехом велась 3 года. На завершение войны повлияла смерть Сталина. Корейская война была завершена перемирием и установлением границы между северной и южной кореей.