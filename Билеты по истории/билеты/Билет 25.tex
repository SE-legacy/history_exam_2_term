\section{Билет 25. Советский Союз в Годы <<Перестройки>> экономика и внутренняя политика. М, С, Горбачев.}

\subsection{Начало перестройки}

Кризисные явления в жизни общества отчётливо выявились и стали стремительно возрастать. Положение Советского Союза отличалось неопределённостью: СССР претендовал на статус великой державы, но его было всё труднее удерживать. Советский Союз занимал $\frac{1}{6}$ часть суши. По последней переписи население СССР достигло 287 миллионов человек. В городах проживало 67\% населения, причём рост происходил за счёт неславянского населения "--- средней Азии, Закавказья и пр. В российских регионах (включая Белоруссию, Украину и Прибалтику) этот рост был не таким быстрым.

Большую долю ресурсов поглощал огромный военно-промышленный комплекс СССР.

Несмотря на определённые достижения экономического и социального состояния, Советский Союз находился в глубоком предкризисном состоянии. Советская промышленность буксовала и это выглядело ещё более неудобно на фоне ускоряющегося развития стран Запада.

Идея Ленина о мировой революции зашла в тупик. Перед правящими кругами страны встала задача сохранения своей власти в этих условиях. При Андропове проводились чистки, подтягивалась дисциплина, но всё это оказалось безрезультатным. Нарастает идея глубокой реформации общественной жизни в СССР.

\subsubsection{\textbf{М.С. Горбачев}}

Кто такой Горбачёв? Родился в 1931 году в крестьянской семье в Ставропольском крае (на тот момен это была Северковказская область). Показал себя как отличный механизатор. В 18 лет получил орден трудового красного знамени. Чуть позже стал кандидатом в члены КПСС, хотя обычно в этом возрасте только в комсомол вступали. Такие успехи позволили Михаилу Горбачёва поступить в Московский Государственный Университет на юридический факультет. Окончил его и в 1952 году стал членом партии. С этого времени начинается работа в комсомольских и партийных структурах. В 1971 году стал членом секритариата ЦК КПСС, а в 1980 году "--- членом политбюро ЦК КПСС. Горбачёв оказался на вершине государственного управления в молодом возрасте.

В 1985 году после смерти Черненко Горбачёв избран генеральным секретарём ЦК КПСС. Председателем совета министров стал Рыжков. Не было даже определённо сформулированных целей.

\subsubsection{\textbf{Пленум ЦК КПСС в апреле 1985 г.}}

Собирается пленум (общее собрание членов и кандидатов) ЦК КПСС, на котором Горбачёв делает доклад и сообщает о планах реформ, направленных на ускорение развития страны. Выдвигается задача достичь качественно нового состояния советского общества. Нужно модернизировать производство и достичь мирового уровня производительности труда в материальной и духовной жизни людей.

В самой перестройке можно выделить два крупных этапа.

\begin{enumerate}
    \item 1985"=1988
    \item 1989"=1991
\end{enumerate}

\subsection{Первый этап. 1985--1988}

Изменений практически не произошло. Советский режим продолжает функционировать на основе Конституции 1977 года. Председатель президиума Верховного совета "--- Громыко, потом Горбачёв.

Фактическую власть вершили исполкомы советов.

В этот период много говорят о демократизации и гласности "--- более широком и свободном обсуждении внутренних проблем.

\subsection{Чернобыльская катастрофа.}

\subsubsection{\textbf{Катастрофа.}}

26 апреля 1986 г. авария на Чернобыльской атомной электростанции, где при 595 проведении эксперимента взорвался четвертый энергоблок. Первоначально не было ясного понимания, что произошедшее "--- катастрофа не только национального, но и мирового масштаба, однако по мере накопления информации приходило осознание трагизма случившегося. Радиационному загрязнению подверглась значительная часть Украины, но самый тяжелый удар пришелся на Белоруссию (23\% территории, пострадал каждый пятый житель).

\subsubsection{\textbf{Ликвидация. Эвакуация людей.}}

Для ликвидации последствий аварии была создана специальная правительственная комиссия, которая координировала эту работу в масштабах всей страны. Уже в первые после катастрофы дни действовала сеть медицинской помощи, охватившая почти миллион человек. Принято было решение о выселении жителей из города энергетиков Припять. 

В район аварии перебросили войска химической защиты, со всей страны прибывала техника. В Москве, Ленинграде, Киеве круглосуточно работали научные институты, срочно решая целую серию необычных проблем. В послечернобыльские месяцы проявились лучшие качества советских людей: самоотверженность, человечность, высокая нравственность.

\subsubsection{\textbf{Концепция <<саркофага>>.}}

К июлю была разработана концепция «саркофага», а затем в сжатые сроки возведено уникальное укрытие для поврежденного реактора с постоянно функционирующей системой контроля за его состоянием.

\subsubsection{\textbf{Итог:}}

Стремление граждан своевременно узнавать правду привело к расширению круга обсуждения ранее запретных тем, способствовало раскрепощению общественного сознания. Большей информационной открытости в связи с Чернобылем требовала и мировая общественность.

\subsection{Демократизация и гласность.}

\subsubsection{\textbf{Демократизация.}}

В Период 1987–-1988 гг. сформулирована собственно горбачевская стратегия преобразований. Основные усилия направлены на повышение активности всех, заинтересованных в обновленческих процессах. Новым в нем плане впервые в советской истории основное внимание концентрировалось на преобразованиях политической системы, которые должны дать импульс соц.экономическому и духовному развитию общества. В докладе на пленуме М. С. Горбачев говорил, что к середине 1980"=х годов в стране сложился «механизм торможения», сдерживающий соц.эконом. развитие, не позволяющий раскрыть преимущества социализма. Его корни "--- недостатки функционирования институтов социалистической демократии, устаревшие политические установках, в консервативном механизме управления. В качестве главного средства слома «торможения» предлагалось развитие самоуправления народа. Рассматривались вопросы совершенствования работы Советов, профсоюзов, комсомола; повышения роли суда, усиления прокурорского надзора, обеспечения прав и свобод граждан. Впервые за долгие годы предлагались выборы на альтернативной основе.

\subsubsection{\textbf{Гласность.}}
Гласность революционизировала и политизировала общество, резко расширила возможности общественного анализа: снятию запретных тем, возможности задавать любые вопросы и предлагать варианты ответов.  Решения январского пленума пробудили общественную активность, связанного с обсуждением наболевших проблем. Начавшаяся самоорганизация общества проявилась в «неформальных» движениях. Курс на проведение политики гласности дал развитие альтернативной прессы. Тиражи были ограничены, в них проблемы общественной жизни обсуждались в откровенной и резкой форме. Издания такого рода сыграли важную роль в организационной консолидации «неформалов». 

Осень 1987 г. стала определенным водоразделом в развитии общественно-политической ситуации в СССР. Горбачев говорил и о значении внутрипартийных дискуссий, которые помогали выработке нужных решений; Поставлен был вопрос о социальной цене революционных преобразований. На официальном уровне было признано создание к концу 1930х годов административно-командной системы, не только охватившей экономику, но и распространившуюся на надстройку. 

Горбачев призвал довести до конца приостановленный в середине 1960х годов процесс «восстановления справедливости» — реабилитировать невинно пострадавших от сталинских репрессий. Доклад Горбачева привел к радикализации политики гласности, которая, в свою очередь, стимулировала поляризацию общественных настроений и впоследствии — политическое размежевание. Активно велась кампания по «дебрежневизации»: в печати разоблачались факты злоупотреблений и коррупции, в которых замешаны многие «первые лица» эпохи «застоя». 

1987–1988 гг. начинается размежевание относительно проводимого в стране курса. Одни полагали, что преобразования идут слишком медленно и не дают результатов Другие считали, что под флагом «перестройки» происходит «сдача» социализма, при том что цели «реформаторов» остаются туманными. Обе позиции резко осуждены «горбачевцами». 

1988 г. ХIХ партийная конференция, на котором прозвучали действительно разные точки зрения по ключевым проблемам. права человека, правовое государство, разделение властей, парламентаризм. Фактически было заявлено о намерении создать гражданское общество. Новые подходы которые затрагивали два базовых института: государство и партию. Намечавшиеся перемены должны были привести к разграничению функций между ними.

Демократизации общества, усилению влияния граждан на принятие решений были призваны способствовать два новых государственных института: съезд народных депутатов и действующий на постоянной основе парламент (Верховный Совет). 

Осенью 1988 г. развернулась работа по реализации мер, намеченных ХIХ партконференцией. Удовлетворив «просьбу» А. А. Громыко об уходе на пенсию, М. С. Горбачев 1 октября 1988 г. возглавил Президиум ВС СССР, сконцентрировав, таким образом, в своих руках высшую партийную и государственную власть. Та же сессия Верховного Совета утвердила поправки в Конституцию СССР.


\subsection{Второй этап.(Начало реформы политической системы.(в методичке))}

Верховный совет принял изменения и дополнения к Конституции 1977 года. Главные изменения касались избирательной системы и принципа функционирования выборных органов. Выборы становятся альтернативными и состязательными. Весной 1989 года проводятся выборы народных депутатов СССР, а через год "--- народные депутаты союзных, автономных республик и местных советов.

Над Верховным советом появился съезд народных депутатов (аналог съезда советов из ранних этапов истории СССР). Депутаты выбираются весной 1989 года. Выборы показали неготовность населения к альтернативным выборам. Были кандидаты не только от местных органов, но и от общественных организаций.

Состоялся первый съезд народных депутатов СССР с мая по июнь 1989 года. Когда народные депутаты собрались, то выяснилось, что на  съезде два блока "--- блок коммунистов и блок демократов. Председателем верхового совета СССР.

Председателем совета министров стал Рыжков. Съезд утвердил и другие важнейшие должности "--- председателя Верховного суда и генерального прокурора. Съезд рассмотрел основные направления внешней и внутренней политики СССР. В дальнейшем состоялось всего 5 съездов народных депутатов. Они принимали изменения в Коснтитуцию, в избирательную систему\dots

В сфере исполнительной власти был введён институт президентства. Президентом был избран Горбачёв по решению чрезвычайного съезда народных депутатов. В этом смысле Советский Союз копировал американскую модель. В феврале 1991 года вице-президентом СССР стал Янаев.

Деятельность депутатов продолжалась до сентября 1991 года, но в августе 1991 года случился пиздец и после попытки госпереворота народных депутатов стали обвинять либо в пособничестве ГКЧП, либо в бездействии. В итоге народные депутаты объявили о самороспуске. В сентябре 1991 года их деятельность прекратилась.

Параллельно с общесоюзными перестраивались органы власти в республиках примерно те же изменения. В некоторых республиках к власти приходили оппозиционеры и, в частности, националисты. Большинство союзных республик объявляют о суверенитете и претендуют на независимость. Наиболее активными эти события были в Прибалтике. Кроме того в Молдавии и на южном Кавказе шли такие процессы. После выборов в верховные советы этих республик победили оппозиционеры. В центральной Азии победили сторонники СССР.

В русских республиках не было преимущества ни у КПСС, ни у оппозиционных сил. Это привело к затруднению работы законодательных органов в России, на Украине и в Белоруссии. При низком уровне политической культуры решение даже незначительных задач вызывало большую полемику.

\subsection{Трудности перехода к рынку.}

В середине 1990 г. в окружении Н. И. Рыжкова \textbf{были разработаны два варианта перехода к рынку}:

\begin{enumerate}
    \item Первый "--- <<Основные направления>> Абалкина.
    \item Второй "--- <<500 дней>> Явлинского.
\end{enumerate}

 \textbf{Они обе включали в себя:}

\begin{itemize}
    \item приватизацию государственной собственности
    \item поддержку малого и среднего предпринимательства и создание конкурентной среды
    \item восстановление товарно"=денежного баланса
    \item плавную реформу ценообразования
    \item индексацию денежных доходов и социальную защиту лиц, оказавшихся за чертой бедности
    \item реорганизацию службы трудоустройства в связи с неизбежным появлением безработицы
\end{itemize}

Различия в программах касались двух моментов. Программа «500 дней» предполагала проведение реформы при заключении лишь экономического союза между суверенными республиками, отодвигая на дальний срок политический договор. А также программа Явлинского опиралась на настроения нетерпения и считала возможным переход к рынку за 500 дней. Программа Абалкина была рассчитана на 6-8 лет.

Сторонники программ начали конфликтовать.

Кризис народного хозяйства требовал принятия срочных мер, однако в Москве не было единства в том, какими они должны быть. Российский парламент одобрил программу «500 дней», Верховный Совет СССР принял свой план рыночного реформирования («Основные направления»). В результате не просто не было согласованных действий, а отношения между союзными и республиканскими властями приобрели остро конфронтационный характер, блокируя любые реформаторские начинания. 

В 1991 г. основных продуктов в расчете на душу населения в целом было столько же или несколько меньше, чем в 1985, хотя уже тогда этот уровень признавался недостаточным. Ажиотажный спрос на продукты и как следствие всяческие дефициты: «про запас» скупалось то, что имело относительно продолжительный срок хранения.В связи с нехваткой товаров в ряде регионов отменялась талонная система, так как власти не могли обеспечить население даже по заявленным скудным нормам.