\section{\textbf{ДОДЕЛАТЬ} Билет 25. Советский Союз в Годы «Перестройки" экономика и внутренняя политика. М, С, Горбачев.}

\subsection{Начало перестройки}

    Кризисные явления в жизни общества отчётливо выявились и стали стремительно возрастать. Положение Советского Союза отличалось неопределённостью: СССР претендовал на статус великой державы, но его было всё труднее удерживать. Советский Союз занимал $\frac{1}{6}$ часть суши. По последней переписи население СССР достигло 287 миллионов человек. В городах проживало $67\%$ населения, причём рост происходил за счёт неславянского населения "--- средней Азии, Закавказья и пр. В российских регионах (включая Белоруссию, Украину и Прибалтику) этот рост был не таким быстрым.
    
    Большую долю ресурсов поглощал огромный военно-промышленный комплекс СССР.
    
    Несмотря на определённые достижения экономического и социального состояния, Советский Союз находился в глубоком предкризисном состоянии. Советская промышленность буксовала и это выглядело ещё более неудобно на фоне ускоряющегося развития стран Запада.
    
    Идея Ленина о мировой революции зашла в тупик. Перед правящими кругами страны встала задача сохранения своей власти в этих условиях. При Андропове проводились чистки, подтягивалась дисциплина, но всё это оказалось безрезультатным. Нарастает идея глубокой реформации общественной жизни в СССР.

    \subsubsection{М.С. Горбачев}
    Кто такой Горбачёв? Родился в 1931 году в крестьянской семье в Ставропольском крае (на тот момен это была Северковказская область). Показал себя как отличный механизатор. В 18 лет получил орден трудового красного знамени. Чуть позже стал кандидатом в члены КПСС, хотя обычно в этом возрасте только в комсомол вступали. Такие успехи позволили Михаилу Горбачёва поступить в Московский Государственный Университет на юридический факультет. Окончил его и в 1952 году стал членом партии. С этого времени начинается работа в комсомольских и партийных структурах. В 1971 году стал членом секритариата ЦК КПСС, а в 1980 году "--- членом политбюро ЦК КПСС. Горбачёв оказался на вершине государственного управления в молодом возрасте.
    
    В 1985 году после смерти Черненко Горбачёв избран генеральным секретарём ЦК КПСС. Председателем совета министров стал Рыжков. Не было даже определённо сформулированных целей.

    \subsubsection{Пленум ЦК КПСС в апреле 1985 г.}
    Собирается пленум (общее собрание членов и кандидатов) ЦК КПСС, на котором Горбачёв делает доклад и сообщает о планах реформ, направленных на ускорение развития страны. Выдвигается задача достичь качественно нового состояния советского общества. Нужно модернизировать производство и достичь мирового уровня производительности труда в материальной и духовной жизни людей.
    
    В самой перестройке можно выделить два крупных этапа.
    
    \begin{enumerate}
        \item 1985"=1988
        \item 1989"=1991
    \end{enumerate}

    \subsection{Первый этап. 1985-1988}
    Изменений практически не произошло. Советский режим продолжает функционировать на основе Конституции 1977 года. Председатель президиума Верховного совета "--- Громыко, потом Горбачёв.
    
    Фактическую власть вершили исполкомы советов.
    
    В этот период много говорят о демократизации и гласности "--- более широком и свободном обсуждении внутренних проблем.
    
\subsection{Второй этап.}
    Верховный совет принял изменения и дополнения к Конституции 1977 года. Главные изменения касались избирательной системы и принципа функционирования выборных органов. Выборы становятся альтернативными и состязательными. Весной 1989 года проводятся выборы народных депутатов СССР, а через год "--- народные депутаты союзных, автономных республик и местных советов.
    
    Над Верховным советом появился съезд народных депутатов (аналог съезда советов из ранних этапов истории СССР). Депутаты выбираются весной 1989 года. Выборы показали неготовность населения к альтернативным выборам. Были кандидаты не только от местных органов, но и от общественных организаций.
    
    Состоялся первый съезд народных депутатов СССР с мая по июнь 1989 года. Когда народные депутаты собрались, то выяснилось, что на  съезде два блока "--- блок коммунистов и блок демократов. Председателем верхового совета СССР.
    
    Председателем совета министров стал Рыжков. Съезд утвердил и другие важнейшие должности "--- председателя Верховного суда и генерального прокурора. Съезд рассмотрел основные направления внешней и внутренней политики СССР. В дальнейшем состоялось всего 5 съездов народных депутатов. Они принимали изменения в Коснтитуцию, в избирательную систему\dots
    
    В сфере исполнительной власти был введён институт президентства. Президентом был избран Горбачёв по решению чрезвычайного съезда народных депутатов. В этом смысле Советский Союз копировал американскую модель. В феврале 1991 года вице-президентом СССР стал Янаев.
    
    Деятельность депутатов продолжалась до сентября 1991 года, но в августе 1991 года случился пиздец и после попытки госпереворота народных депутатов стали обвинять либо в пособничестве ГКЧП, либо в бездействии. В итоге народные депутаты объявили о самороспуске. В сентябре 1991 года их деятельность прекратилась.
    
    Параллельно с общесоюзными перестраивались органы власти в республиках примерно те же изменения. В некоторых республиках к власти приходили оппозиционеры и, в частности, националисты. Большинство союзных республик объявляют о суверенитете и претендуют на независимость. Наиболее активными эти события были в Прибалтике. Кроме того в Молдавии и на южном Кавказе шли такие процессы. После выборов в верховные советы этих республик победили оппозиционеры. В центральной Азии победили сторонники СССР.
    
    В русских республиках не было преимущества ни у КПСС, ни у оппозиционных сил. Это привело к затруднению работы законодательных органов в России, на Украине и в Белоруссии. При низком уровне политической культуры решение даже незначительных задач вызывало большую полемику.
    
