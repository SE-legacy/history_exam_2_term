\section{Билет 6. Завершающий период Гражданской войны (1919-1922гг) и её итоги}

\subsection{Основные события Гражданской войны на Восточном и Южном фронтах в 1919--1920 гг.}

\subsubsection{Южный фронт}

Войска южного фронта, получившие подкрепление, переходят в наступление против донской армии.
24 января "--- После оттеснения было отправлено циркулярное письмо войскам на дону с требованием провести массовый террор против казаков, принимавших участие в борьбе с советской властью.
Ответом на такие действие стало восстание казаков в марте 1919 года, серьезно затруднившее наступление войск южного фронта.

Деникин (глава добровольческой армии) пытался добиться централизации управления всеми антисоветскими силами. По соглашению с Красновым, добровольческая и донская армия объединились. К ним также присоединился черноморский флот. Совместными силами они смогли остановить продвижение войск южного фронта.

\subsubsection{Восточный фронт}

В конце 1918 года на Украине возникнает директория украинской НР во главе с Петлюрой. Они оказывают сопротивление как большевикам, так и армии Деникина.
Армия директории (Украинская народная армия) состояла из 30 тысяч человек.

Советско-Украинский фронт предпринимает наступление в январе-феврале 1919 года. Они занимают Харьков, Киев и продвигаются в сторону Львова.
В это же время, войска интервентов постепенно оставляют Херсон, Николаев и отступают к Одессе. В этих войсках под влиянием советской пропаганды начинаются волнения.
В конце марта 1919 года состоялась Парижская мирная конференция, на которой союзники стран Антанты принимают решение об эвакуации войск с территории России.
На восточном фронте войска продолжают наступление на города Уральск, Оренбург, Уфа в тяжелых сражениях с армией Колчака. К концу февраля войскам восточного фронта удалось соединиться с туркестанскими войсками.

Масштабные боевые действия в марте 1919 года. Колчак предпринимает массированные наступления нескольких армий. Он отбивает Уфу и другие города, движется к Волге. Он требует от войск отбросить войска красной армии за Волгу, попытаться разгромить на подходах к ней и захватить важнейшие переправы. Это вызывает ответные действия советского правительства. Ленин призывает направить все силы на разгром армий Колчака. Восточным фронтом стал руководить М.В. Фрунзе (после назначения Каменева на пост главнокоманующего). Постепенно усилив восточный фронт, начинается контрнаступление. Красная армия занимает Бугуруслан, разбивает корпус генерала Капеля (?). Большую роль сыграла армия Чапаева. Его дивизия успешно воевала с казаками до 5 сентября 1919 года, когда он был убит.
Тем не менее, общий успех красной армии к сентябрю 1919 года был очевиден. После взятия Ижевска окончательно стратегическая инициатива переходит к восточному фронту.
В январе 1920 года армия Колчака была разгромлена. Части красной армии выходят через юг Сибири к Байкалу. Но дальше они не проходят, так как к востоку от Байкала находятся армии Японии и США. В апреле 1920 года создаётся дальневосточная республика.

В средней азии создаётся туркестанский фронт во главе с Фрунзе. Он нанёс окончательное поражение белым. Воспользовавшись междуусобной войной туркестанских племен было ликвидировано какое-то ханство и создается Хорезанская НР, а также Бухарская НР.

\subsubsection{Снова южный фронт}

Поскольку к январю 1920 года красная армия добилась успехов в борьбе с Колчаком, она была переброшена на южный фронт. 
В июле 1919 года Деникин организовывает мощное наступление. Они должны двигаться через курск и воронеж с выходом к Москве. В июле 1919 года на пленуме РКПБ зачитывается письмо Ленина, которое начинается как «Все на борьбу с Деникиным». Южный фронт получает пополнение в виде состава, патронов и снарядов. Тем временем, добровольческая армия к сентярю захватила Курск и Воронеж, подошла к Орлу. Только тут красной армии удалось остановить армию Деникина и перейти в контрнаступление. В январе 1920 вооруженные силы юга россии были окончательно разгромлены. Деникин 4 апреля 1920 года передал командование генералу Врангелю, который начал укрепляться в Крыму.
К 1920 году были разбиты основные противобольшевистские силы, что позволило укрепиться власти большевиков.

\subsection{Советско"---Польская Война.}

Поначалу, наступление польских войск ведется вместе с украинскими войсками Петлюры. Только к середине 1920 года удалось предотвратить наступление поляков. Переводятся части из Белорусии под командованием Тухачевского. Он решает предпринять наступление без достаточной подготовки. Наступление быстро было разбито. Несмотря на провал, оно смогло отвлечь польские войска и юго-западный фронт перешёл в контрнаступление. 
Польское командование значительно укрепило свои войска и в августе не только остановило контрнаступление, но и отбросило его назад. Советское правительство было вынуждено пойти на переговоры и 18 марта 1921 года был заключен Рижский мир между СССР, украинской ССР и Польшей. Он установил границы между Польшей и Украиной.

\subsection{Правительство юга России и П. Н. Врангель}

После заключения мира, красная армия сосредоточила силы на разгроме Врангеля в Крыму. К началу ноября 1921 года проведена операция, завершившаяся занятием крымского полуострова, после чего остатки белогвардейских сил эвакуировались в турцию.
Ещё один центр войны – Закавказье. В результате боевых действий красной армии создаются республики в Армении, Грузии и Азербайджане.

\subsection{Наступление Юденича на петроград.}

В Эстонии белогвардейскую армию возглавлял генерал Н. Н. Юденич.
Белогвардейские и эстонские войска ведут наступление в апреле 1920 года на Петроград.
Правительство Латвии, Эстонии не поддерживали белое движение, из-за конфликта интересов. Белые не были готовы признать существование новых государств в прибалтике. Переговоры дали  большевикам в 1919 году возможность перебросить большую часть войск на восток. С Латвией и Эстонией были заключены договоры, признававшие независимость этих республик.

\subsection{Советская власть и крестьянство. Рабочие в годы <<военного коммунизма>>. Сопротивление советской власти на завершающем этапе войны: Крондштадтское, Тамбовское и Западносибирское восстания}

На рубеже 1918-1919 были отменены комбеды, что уменьшило сопротивление крестьянства, однако продолжались топливный и продовольственный кризис. В городах вводится нормированное снабжение продовольствием и товаром (карточная система). Ленин выдвигает «Не смей командовать середняком!». В соответствии с этой новой политикой ВЦИК и Совнарком принимают решение об освобождении натурального налога в 36 губерниях.

Сохранение экстренных мер военно-коммунистического характера вызвало недовольства. Крестьяне были недовольны сохранившейся продразвёрсткой. К ним присоединяются рабочие. Начинаются политические восстания. Особенно крупные – Кронштадтское в марте 1922 года.

\subsection{Милитаризация промышленности}

Снабжение армии осуществлялось за счет национализации производств и трудовой повинности, за счёт продовольственных отрядов, изымавшие излишки у кулаков, вводится карточная система. Действует система продразвёрстки (обязательства возлагаемые на крестьянство поставлять продукцию государству, фактически не ограничивался размер продразвёрстки).

Трудовая повинность была первоначально только у буржуазии, но с октября 1918 г. для всех лиц от 16 до 50 лет на учёте в отделах распределения рабочей силы.

\subsection{Кризис военно-коммунистической системы}
Экономическое положение к весне 1921 года невероятно тяжелое. Весной 1921 года начинается голод, особенно тяжелый в Поволжье и южном Приуралье. От него пострадало 20\% населения и 5 млн человек погибло. Советское правительство обращается за помощью к другим странам. Наибольшую помощь внесла американская компания ARA.

Сохранение экстренных мер военно-коммунистического характера вызвало недовольства. Крестьяне были недовольны сохранившейся продразвёрсткой. К ним присоединяются рабочие. Начинаются политические восстания. Особенно крупные – Крондштатское в марте 1922 года.

\subsection{Меры советской власти в области культуры и в отношении церкви за годы революции и Гражданской войны}

Несмотря на регулярное закрытие храмов и репрессии священнослужителей с 1918 года церковь также внесла помощь. Правительству это не нравится и в 1922 году издаётся декрет по которому происходит принудительное изъятие церковных предметов (?).

\subsection{Исторические итоги Гражданской войны}

Итоги:
\begin{enumerate}
	\item Огромные человеческие жертвы (около 13 млн. чел.)
	\item Величайшее хозяйственное разорение
	\item Голод 1921--1923 г.
	\item Огромное количество беспризорников (8-9 млн. чел.) 
	\item Сохранение диктатуры российской коммунистической партии большевиков
\end{enumerate}

\subsection{Причины победы большевиков}

Причины победы большевиков и исходы:
\begin{enumerate}
	\item Энергичные действия советского правительства по созданию нового государства и армии
	\item Продовольственная диктатура
	\item Единство и централизация армии
	\item Активная пропагандистская и идеологическая работа
	\item Безжалостный террор
\end{enumerate}

Причины поражения белых:
\begin{enumerate}
	\item Отсутствие единства в армии
	\item Поддержка интервентов, расценивавшаяся как предательство
\end{enumerate}
