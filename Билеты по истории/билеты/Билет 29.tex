\section{\textbf{Проблемный вопрос} Билет 29. Социально-экономические реформы 1990-х гг.}
\subsection{Программа экономических реформ Е. Т. Гайдара, её реализация и последствия}
В конце ноября 1991 года Гайдар изложил план первоочередных действий правительства по стабилизации экономического положения: отпуская цены и зарплату при одновременном проведении жесткой финансовой политики, стабилизировать экономику и восстановить ее управляемость на рыночной основе.

2 января 1992 цены на подавляющее большинство товаров (за исключением хлеба, молока, спиртного, а также коммунальных услуг, транспорта и энергоносителей) были освобождены, а регулируемые - повышены. Освобождение привело к тому, что на полках магазинов появились исчезнувшие ранее продукты и товары по заоблачно завышенным ценам. На многие основные товары к концу 1992 года они выросли в 36 раз. Резко снизилась покупательская способность.

29 января 1992 г. Ельцин подписал указ о свободе торговле (не <<о свободной торговле>>, а <<о свободе торговле>> --- прим. ред.). Все желающие получили возможность заниматься торговлей, что привело к появлению крупных мест сбыта б/у товаров.

Введён 28-процентный налог на добавленную стоимость.

В декабре 1992 года VII Съездом народных депутатов Егор Гайдар не был утвержден на пост председателя Совета министров. После утверждения главой правительства Виктора Черномырдина Гайдар был отправлен в отставку.

Деятельность Егора Гайдара оценивается неоднозначно. С одной стороны, его реформа цен в январе 1992 года, фактически означавшая отказ от государственного регулирования цен на большинство товаров, включая товары первой необходимости, позволила практически мгновенно наполнить полки магазинов, полностью опустевшие в предшествовавшие годы. Однако при сохранении доходов населения неизменными, это привело к катастрофическому падению уровня жизни.

Реформаторам удалось сократить дефицит государственного бюджета и перевести советскую плановую экономику на рельсы свободного рынка, но побочным эффектом их действий стали гиперинфляция и экономический кризис.
\subsection{Ваучерная приватизация. Залоговые аукционы.}
В последние дни 1991 г. указом Ельцина были утверждены основные положения программы приватизации. 29 января 1992 г. в его развитие был подписан важный указ, которым утверждались основные нормативные документы, регламентирующие порядок главных приватизационных процедур: проведение конкурсов и аукционов, порядок оплаты и т.п.

В феврале — марте 1992 г. на основе этих документов набирает темп «малая приватизация» (предприятия торговли, общественного питания, сферы обслуживания).

14 августа опубликован Указ Президента, предопределивший начало и содержание первого, «ваучерного» этапа приватизации, продлившегося 22 месяца, до 1 июля 1994 г. Ваучерная, или чековая модель приватизации предусматривала преобразование крупных и средних государственных предприятий в акционерные общества с их последующей передачей непосредственно гражданам, среди которых работники трудового коллектива приватизируемого предприятия получали льготы.

Для участия населения в приобретении акций вводились приватизационные чеки — ваучеры, которые символизировали равенство стартовых условий всех участников приватизации.

Полученные ваучеры населению предстояло обменять на акции предприятий, что означало бы юридическое вхождение в права собственника. Однако поскольку не были разъяснены принципы приватизационного процесса, в стране началось создание чековых инвестиционных фондов (ЧИФ), которые аккумулировании значительных пакетов ваучеров населения с целью извлечения прибыли. Многие из них работали по-чёрному и закрылись, забрав с собой сбережения граждан.

К 1 июля 1994 г. было разгосударствлено 70\% промышленных предприятий, а доля государственной собственности в общем объеме стоимости имущества сократилась до 35.

Негативное восприятие приватизации обусловлено расхождением между публично декларированными принципами, целями, задачами и теми непосредственными результатами приватизации, с которыми столкнулось большинство граждан, а также преступными методами многих активных участников приватизации.

Отрицательным следствием избранной модели разгосударствления стал колоссальный рост преступности, связанной с приватизацией. Вице-премьер также полагал, что преобразования способствовали подрыву национальной безопасности, что проявилось.

22 июля 1994 г был издан Указ Президента, утвердивший «Основные положения государственной программы приватизации государственных и муниципальных предприятий после 1 июля 1994 года». Он положил начало второму, «денежному» этапу приватизации. С того времени предприятия или пакеты их акций стали передаваться исключительно за деньги. Приватизация иму щества осуществлялась через денежные, специализированные аукционы, коммерческие и инвестиционные конкурсы, по закрытой подписке.
\subsection{Свобода внешней торговли, свобода выезда за рубеж, разрешение свободного обращения иностранной валюты}
Сняты были количественные ограничения по экспорту готовой продукции (топливо и сырьё вывозились по квотам). Был установлен нулевой импортный тариф, отчего в страну хлынул поток товаров самого различного ассортимента и качества. Свободный импорт в начале 1992 г. сыграл роль катализатора в развитии частной рыночной торговли. 
\subsection{Негативные последствия реформ}
Безработица, деиндустриализация, криминализация общества, падение жизненного уровня населения, имущественное расслоение

Неблагополучно складывалась ситуация с денёжными доходами граждан. Из-за либерализации цен в огне инфляции сгорели многолетние сбережения населения. Отрицательный общественный резонанс вызвал невнимание властей к этой острой проблеме.

Темпы инфляции были настолько высокие, что государство не смогло выполнить социальные обязательства перед пенсионерами, инвалидами и бюджетниками, что вызвало ещё большую волну недовольства.

В промышленности наблюдалось сокращение объёмов выпускаемой продукции при значительном росте цен на изделия, а государство тем временем отказалось от финансирования нерентабельных предприятий. Чисто «рыночное» решение проблемы требовало банкротства несостоятельных плательщиков, однако её масштабность делала неизбежным государственное вмешательство.

Основной формой торговли промышленным оборудованием и сырьем, составлявшей 60—70\% её объема, стали прямые сделки между производителями, осуществлявшиеся в первые два года реформ зачастую в натуральной форме (бартер). После введения конвертируемости рубля бартер стал уменьшаться.

Показателем ущербности навязанной модели приватизации являются невероятно низкие доходы, полученные от нее государством. Приватизация принесла России 5\% реальной стоимости предприятий. По оценке Государственной думы, потери, нанесенные приватизацией России, в 2,5 раза превысили потери СССР в Великой Отечественной Войне.


\subsection{Финансовые пирамиды}
{\it Мавроди всех наебал короче.}

\subsection{Рост зависимости экономики от международных цен на энергоносители}
{\it Хз, что тут сказать. Основным экспортируемым товаром стало сырьё по причине пиздеца в промышленности и необходимости лёгких денег для восстановления экономики, а самый ценный товар из добываемого в России сырья --- это, кроме редкоземельных металлов и урана, нефть и газ. Соответственно, экономика стала зависеть от цен на энергоносители, что стало одной из причин кризиса 1998 года.}

\subsection{Финансовый кризис 1998 года}
В начале 1998 г. в экономике страны сложилась крайне непростая ситуация. Государственный долг России перед иностранными кредиторами составил 123,5 млрд. долларов. Из-за внешнего долга возникла зависимость нашей страны от международных организаций, прежде всего – МВФ, который фактически контролировался США. Каждый год в МВФ приходилось утверждать бюджет, любое стремление к самостоятельной политике угрожало дефолтом, т.е. банкротством.

Летом 1998 г. правительство должно было выплатить более 60 млрд. долларов по внешнему и внутреннему долгу. В тот же период доходы бюджета составляли только 20 млрд. долларов. К тому времени экономика страны уже приобрела сырьевую направленность, а в начале 1998 г. мировые цены на нефть резко упали, ниже 10 долларов за баррель. Обрабатывающая промышленность была разрушена и не приносила существенной прибыли. Население обнищало и не могло стимулировать внутренний спрос.

17 августа 1998 г. премьер-министр С.В. Кириенко объявил о дефолте – прекращении выплат по обязательствам государства. Это привело к девальвации – курс рубля по отношению к доллару за полгода упал более чем в 3 раза – c 6 руб. перед дефолтом до 21 руб. к концу года.

Резкое падение курса рубля значительно сократило импорт и повысило конкурентоспособность отечественного производства. Произошло импортозамещение. В 1999 г. в России начался экономический рост.
