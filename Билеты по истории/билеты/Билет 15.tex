\section{Билет 15. СССР в начальный период второй мировой войны(1939 – 1941 гг.): внутреннее развитие и внешняя политика.}

Причины Второй Мировой войны заключались в противоречии ведущими мировыми странами, нарастании конфликта. В мире сложилось 3 противостоящих силы: Фашистские режимы (Германия, Италия, Испания), Советский союз, Демократические режимы. Они конкурируют между собой за влияние. Германия стремится к реваншу в первой мировой войне. Гитлер говорил про объединение всех территорий, населённых немцами. Это расширение подразумевало территории не населённые немцами, но требуемые Германии. Это расширение рассматривалось, в основном, на восток от Германии.


\subsection{Пакт Молотова-Риббентропа}

23 августа 1939 г. в Москве был заключен советско-германский договор о ненападении,
незамедлительно вступавший в силу и рассчитанный на 10 лет (пакт Риббентропа"=Молотова). К нему был приложен секретный протокол о разграничении сфер влияния в Восточной Европе. Интересы Советского Союза были признаны Германией в Прибалтике
(Латвия, Эстония, Финляндия) и Бессарабии

1 сентября 1939 года началась Вторая Мировая война. В этот день войска
Вермахта развернули боевые действия против Польши. Используя подавляющее
превосходство в численности и технике, нацистское командование добилось
крупных результатов. В тот же день Великобритания и франция объявляют Германии войну. Но оказать эффективную помощь Польше они не могли.

В это же время Советское правительство объявляет о необходимости «взять под защиту» ту часть восточной Польши, которая населена не поляками, а другими нациями. 17 сентября в Польшу вступают отряды рабоче"=крестьянской красной армии. 29 сентября Польша перестаёт существовать как суверенное государство

28 сентября был заключен советско-германский договор «О дружбе и границе», закрепивший эти земли в составе Советского Союза.

\subsection{Вхождение в состав СССР Литвы, Латвии, Эстонии и Бессарабии}

Советский Союз в эти же дни заключает договоры о дружбе с прибалтийскими республиками. 28 сентября с Эстонией, 5 октября с Латвией, 10 октября с Литвой. По этим договорам СССР получил право на размещение войск в Прибалтике с целью защиты прибалтийских республик от агрессии.

14 июня 1940 года правительство СССР требует от Литвы, 16 июня от Латвии и Эстонии назначить просоветское руководство. В республиках устанавливаются народные правительства, установившие управление коммунистической партией. После взаимных консультаций между СССР и Германии Советский Союз вводит войска в Бессарабию и северную Буковину в начале июля.

\subsection{Зимняя война}

30 ноября 1939 -- начало войны СССР с Финляндией.

Войска Ленинградского военного округа атакуют в условиях зимней стужи без должной подготовки атакуют линию генерала фон Маннергейма. Этот штурм привёл к взятию линии Маннергейма с большими потерями.

Добившись успеха, руководство СССР пошло на заключение мира с Финляндией 12 марта 1940 года. Финляндия передавала Советскому Союзу значительную часть своих территорий в Карелии. Была образовано Карело"=Финская ССР.

Советский союз в ходе Финской войны понёс большие людские потери (до 75 тыс убитыми и до 100 тыс ранеными и обмороженными). 14 декабря 1939 года лига наций исключила СССР из своего состава. Эта война показала слабость советских вооружённых сил, ударила по репутации СССР.

\subsection{Положение в армии}
В высшем военном руководстве шли споры между сторонниками кавалерии и механизации армии.
Война с Финляндией показала необходимость перемены. В мае нарком обороны Ворошилов покидает свой пост, его замещает Сергей Константинович Тимошенко. В тот же день издан указ президиума Верховного совета о восстановлении генеральских и адмиральских званий.

Тимошенко критикует наркомат, требует ускорения перевооружения армии. Советская военная доктрина к 1941 году исходила из того, что войну можно выиграть только успешными наступательными операциями. Оборона по всем фронтам советским руководством не рассматривалась.

