\section{Билет 18. Завершающий период Великой Отечественной войны (декабрь 1943 — май 1945 г.), Окончание Второй мировой. Итоги войн для СССР.}


Важнейшие тенденции завершающего периода ВОВ:
\begin{itemize}
    \item Возрастающая мощь стран Антигитлеровской коалиции. Им удалось в несколько раз превысить возможности Германии и её союзников
    \item Резкое ухудшение возможностей Германии. Истощаются материальные и людские ресурсы. Командование Вермахта перешло к жёсткой позиционной борьбе: крепости, ДОТы\dots
\end{itemize}

\subsection{Помощь союзников СССР по ленд-лизу.}

Однако враг был по-прежнему силён. По людским ресурсам превосходство было незначительно, по танкам и штурмовым орудиям был паритет, по самолётам СССР превосходил Германию в несколько раз. Это связано с ростом производства военной техники в 4-8 раз и помощью союзников (ленд-лизом: lend, а не land -- займы). Ленд-лиз --- название закона, принятого в США ещё в 1941 году. Он предусматривал поставки боеприпасов, техники, продовольствия, медикаментов, стратегического сырья без предварительной оплаты. На условиях беспроцентных долгосрочных кредитов вооружение должно было быть оплачено после завершения войны.

СССР получил более 18 тысяч самолётов, 18,5 тыс. ед. бронетехники, 400 тыс. грузовых и легковых автомобилей, локомотивы и другие виды военных предметов и стратегических материалов. Объём ленд-лиза превысил 10 миллиардов долларов.

\subsection{Депортация отдельных народов в советском союзе и коллаборационизм в СССР}

Советский Союз мобилизовал все возможные ресурсы. Но был ещё силовой (репрессивный) ресурс, который тоже использовался. Как и в других странах, Советским Союзом практиковались массовые принудительные переселения: переселялись поляки и другие не принявшие новую власть представители территорий, присоединённых к СССР. Все они отправлялись в отдалённые регионы Советского Союза.

В западных регионах СССР развернулось мощное коллаборационистское движение. С ними советская власть вела жестокую борьбу как в годы Великой Отечественной, так и после. Стали осуществляться переселения и депортации целых народов, которые (по мнению государства) поддерживали  Гитлера.

В августе 1943 года советское правительство, опасаясь восстания, ликвидировало республику немцев Поволжья, а сами немцы были расселены по отдалённым регионам.

Депортации практиковались в северном Кавказе и в Крыму. Карачаевцев, балкарцев, чеченцев и ингушей обвинили в коллаборационизме. Такая же участь постигла калмыков, крымских татар, греков и ещё некоторые диаспоры, проживавшие здесь. Как правило, депортация сопровождалась ликвидацией национальных автономий, русификацией местной топономики.

\subsection{Советские стратегические наступательные операции 1944 г.}

Начало 1944 года. Осуществляются стратегические операции советских войск в составе четырёх фронтов. Открыт выход на государственную границу в районе Карпат.В январе, феврале освобождена правобережная Украина. Весной 1944 года освобожден Крым, Одесса.

Волховский, Прибалтийский и ещё какой-то фронты 27 января 1944 года окончательно сняли блокаду Ленинграда. Она продолжалась более 800 дней и унесла огромное количество жизней. Больше всего там погибло мирного населения. Официальное признанное число погибших в блокаде --- 630 тысяч человек, по современным оценкам --- более миллиона жертв. 90\% из них связаны со смертью от голода или заболеваний, связанных с ним.

Советские войска разворачивают мощное наступление в Карелии, Прибалтике, Белоруссии, Румынии.

Финляндия подписывает перемирие с СССР в сентябре 1944, а 4 марта 1945 года объявит войну Германии.

В ходе операции по освобождению Белоруссии "Багратион" (23 июня - 29 августа 1944) была разгромлена группа армий "Центр", освобождена Литва, Латвия и восток Польши.

\subsection{Висло"=Одерская операция}

Висло"=Одерская стратегическая наступательная операция -- стратегическая наступательная операция советских войск на правом фланге советско"=германского фронта в 1945 году. Началась 12 января, завершилась 3 февраля. Проводилась силами 1"=го Белорусского (командующий -- Маршал Советского Союза Георгий Жуков) и 1"=го Украинского фронтов (Маршал Советского Союза Иван Конев).

В ходе Висло-Одерской операции от немецких войск была освобождена территория Польши к западу от Вислы и захвачен плацдарм на левом берегу Одера, использованный впоследствии при наступлении на Берлин. 

Венгрия, Болгария, Югославия и Чехословакия при советской помощи освобождаются от нацизма. Фронт приблизился с запада и востока к границам Германии. Она оказалась фактически изолирована в тисках. Проводится тотальная мобилизация, в армию забирают подростков и стариков.


\subsection{Крымская (Ялтинская) конференция}

С 4 по 11 февраля 1945 года состоялась Ялтинская конференция. На ней согласованы действия по совершению разгрома Германии и послевоенному урегулированию. Советский союз брал на себя обязательства вступить войну с Японией максимум через 3 месяца после окончания войны в Европе.

Основные её пункты:

\begin{itemize}
    \item Создание ООН (СССР достаётся 3 места за РСФСР, Белоруссию и Украину)
    \item Восточная граница Польши пройдет по <<Линии Керзона>>
    \item Раздел Германии на 4 зоны оккупации
    \item Вступление СССР в войну с Японией, в обмен на Курильские острова, Южный Сахалин,  не позднее, чем через 3 месяца, после конца войны с Германией
    \item Репарации от Германии СССР составят 10 млрд долларов -- половина от общих репараций.
\end{itemize}

\subsection{Битва за Берлин}

В апреле 1945 г. советские войска начали Берлинскую операцию.
Она была нацелена на взятие столицы Германии и окончательный разгром фашизма. Войска 1-го (командующий маршал Г.К. Жуков), 2-го (командующий маршал К.К. Рокоссовский) Белорусских и 1-го Украинского (командующий маршал И.С. Конев) фронтов уничтожили берлинскую группировку противника, взяли в плен около 500 тыс. человек,
огромное количество военной техники и оружия. Фашистское руководство было полностью деморализовано, А. Гитлер покончил жизнь самоубийством. Утром 1 мая было завершено взятие Берлина и над рейхстагом (германский парламент) водружено Красное Знамя -- символ Победы советского народа.

8 мая 1945 г. в пригороде Берлина Карлсхорсте наспех созданное новое правительство Германии подписало Акт о безоговорочной капитуляции. 9 мая были разгромлены остатки немецких войск в районе Праги, столицы Чехословакии. 

\subsection{Потсдамская конференция (17 июля - 2 августа 1945 г.)}

Состоялась Постдамская конференция, где главы Антигитлеровской коалиции приняли решение о послевоенном устройстве в Европе.

Участники конференции разработали принципы, нацеленные на осуществление демилитаризации, денацификации и демократизации Германии - план искоренения германского милитаризма и нацизма. Он включал ликвидацию германской военной промышленности, запрещение германской национал-социалистической партии и нацистской пропаганды, наказание военных преступников. Было достигнуто соглашение о репарациях с Германии.

Конференция рассмотрела ряд территориально-политических вопросов. СССР передавался Кенигсберг (столица Восточной Пруссии). Территория Польши значительно расширялась на западе за счет Германии (польско-германская граница была установлена по рекам Одер-Нейсе). 

\subsection{Разгром Квантунской армии}

8 августа 1945 года Советский Союз объявляет войну Японии, захватившей Корею, значительную часть Китая и многие острова в Тихом океане.

Забайкальский, 1-й и 2-й Дальневосточный фронты при поддержке Тихоокеанского фронта под командованием маршала Василевского вторгаются на территорию Японии. У Советской армии было большое численное превосходство над Квантунской армией. При поддержке Монгольской народной республики в короткие сроки были освобождены Корея, Маньчжурия и Курильские острова.

Большую роль сыграла предшествовавшая борьба союзников против Японии, а также сброс ядерных бомб на Хиросиму и Нагасаки 6 и 9 августа соответственно. Применение ядерного оружия деморализовало японское правительство и общество.

2 сентября 1945 года японцы капитулировали.

\subsection{Значение, факторы, цнеа победы в войне. Исторические итоги войны}

Победа стран Антигитлеровской коалиции имела всемирное историческое значение. Она оказала влияние на судьбу человечества и продолжает оказывать до сих пор.

В ходе четырёхлетней борьбы гитлеровская Германия потеряла 80\% личного состава и 75\% техники на востоке. Цена Победы была огромная. По приблизительным подсчётам Советский Союз потерял по самым скромным оценкам 27 миллионов человек. Из них на фронте в боях погибло около 10 миллионов, половина из которых --- гражданское население, уничтоженное бомбёжками и в концентрационных лагерях. Около 6 млн советских граждан было угнано. 1710 городов, свыше 70 тысяч сёл и деревень Советского Союза уничтожено. Потрачено почти 2 миллиарда советских рублей, 30\% национального богатства утеряно. После войны осталась масса инвалидов, людей с осколочными ранениями и детей-сирот.

После разгрома германии в мире организовалось два противобортсвующих блока -- демократический и социалистический.