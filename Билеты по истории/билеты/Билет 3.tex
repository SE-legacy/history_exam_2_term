\section{Билет 3. Октябрьская революция 1917 г.: причины и основные события на изначальном этапе (1917-1918 гг.)}

\subsection{Причины}

Фактическое отсутствие организованной центральной власти. Временное правительство во главе с Керенским не полностью контролирует армию (где действуют солдатские комитеты), провинцию (где формируются сепаратистские движения), даже некоторые губернии (где работают советы, имеющие больший авторитет). И уже к концу сентября Керенскому никто не желает подчиняться

Обострение социально-экономических проблем и затягивание их решения.

Продолжение крайне непопулярной войны

\subsection{Углубление дестабилизации российского общества в сентябре – октябре 1917 г.}

(Большевики, выступившие против Корнилова, были реабилитированы в общественном мнении и набирали сторонников. В сентябре-октябре численность РСДРП доходила до 350 тысяч сторонников.)

Разочарование во временном правительстве, низкий уровень жизни, политическая нестабильность и огромная инфляция подпитывали революционные настроения населения.

(за год потребительская способность упала в 7 раз, отсюда же падения предпринимательской активности)

К осени 1917 года растет нехватка продовольствия. Растёт число бастующих – 2,4 млн человек в сентябре-октябре. На селе вырастают массовые крестьянские выступления. Крестьяне надеялись на то, что весной они, наконец, получат помещичьи земли, но не дождавшись результатов начинают силой захватывать территории.

Начинается подготовка к вооруженному восстанию. Ленин считает, что весь мир находится на пороге пролетарской революции. Ленин не желал ждать даже второго съезда советов, так как на нём могли не поддержать идеи Ленина. Постепенно Ленин объясняет своим приверженцам и центральному комитету необходимость вооруженного восстания. Он возвращается в Петроград для поиска поддержки ЦК. Состоялось собрание, на котором большинство членов ЦК поддерживают идеи Ленина, кроме Зиновьева и Каменева. 10 октября берётся курс на подготовку вооруженного восстания.

\subsubsection{\textbf{12 октября}}
Создаётся военный революционный комитет (ВРК) при петроградском совете.

\textbf{Состав:} большевики и левые эсеры, согласные с идеями Ленина.

Во временное правительство поступают сообщения о подготовке вооруженного выступления и оно пытается на непрерывных заседаниях переговорами предотвратить его, однако разногласий слишком много и политических сил у временного правительства недостаточно. 

Ещё весной 1917 года создаются отряды красной гвардии под контролем большевиков (к осени 100 тысяч человек).

Осенью 1917 года немецкое командование переходит в наступление на западном фронте и Керенский отправляет часть армии на фронт, уменьшая свою поддержку внутри Петрограда. 

Авангардом большевистких сил были моряки балтийского флота, которые еще в середине ноября заявили о неподдержке временного правительства.

Керенский пытается стягивать немногочисленные силы, дабы усилить поддержку. В его руках находились небольшие формирования из юнкеров (учеников военных училищ) и некоторые казачьи партии.

\subsubsection{\textbf{В ночь на 24 октября}}

Керенский пытается захватить Смольный институт благородных девиц, занятый петроградским советом.

В это же время выходит распоряжение о закрытии большевистских газет, аресте членов военно-революционного комитета.

\subsubsection{\textbf{24 октября}}

Военно-революционный комитет начинает захватывать стратегически важные объекты Петрограда (почта, телеграф, мосты, крепости). Керенский понимает, что ситуация складывается не в его пользу и утром 25 октября уезжает на фронт, дабы использовать силы армии для подавления восстания.

\subsubsection{\textbf{Днём 25 октября}}

Войска окружают мариинский дворец, где заседает предпарламент. Тогда же Ленин пишет обращение от ВРК к гражданам России, в котором говорится о ликвидации временного правительства и перехода власти к ВРК.

Революционные части стягиваются к зимнему дворцу (где заседает временное правительство). В 7 часов вечера временному правительству предъявляется ультиматум о сложении полномочий. После отказа начинаются выстрелы по зимнему дворцу. После этого вновь предлагается ультиматум, после которого начинаются артиллерийские обстрелы. Потом отряды проникают в Зимний дворец и арестовывают временное правительство

\subsubsection{\textbf{25 октября}}

Второй Всероссийский съезд советов. В нём доминируют большевики и левые эсеры.

Правые эсеры и меньшевики высказывают резкое осуждение действий большевиков и уходят из съезда.

После выступления со всеми новостями о большевистских действиях, всё настроение съезда склонилось в сторону Ленинских идей.

Утром объявляется о переходе власти ко второму съезду советов, а на местах власть переходит к советам рабочих и солдатских депутатов.

\subsubsection{\textbf{26 октября}}

Первые декреты

\begin{enumerate}
    \item О мире: предложение к немедленным мирным переговорам «без аннексий и контрибуций». 
    \item О земле: (во многом копировал эсеровскую аграрную программу) отмена частной собственность на землю, она теперь разделялась между крестьянами по трудовой и потребительской норме. Запрет аренды земли и применение наёмного труда.
\end{enumerate}

По вопросам управления государством было провозглашено создание большевистского временного правительства, которое называлось «Советом народных комиссаров». Туда вошли Ленин (Ульянов), ставший председателем, Рыков, Милютин, Троцкий (Бронштейн), Скворцов (Степанов), Луначарский, Сталин (Джугашвили).

Создан комитет по военно-морским делам (Антонов, Авсеенко, Крыленко, Дыбенко).

\subsubsection{\textbf{Высший законодательный орган власти:}}

Всё ещё съезд советов.
Так как этот орган был непостоянным, в период бездействия съезда его обязанности выполнял Всероссийский Центральный Исполнительный Комитет (ВЦИК).

\subsubsection{\textbf{Состав ВЦИК:}}

2/3 большевики, 1/3 эсеры.

\textbf{Представитель:} Каменев.

Над всем стоит партия РСДРП(б), которая так же имеет собственную структуру.

\subsubsection{\textbf{27--28 октября}}

Корпус генерала Краснова приближается к Петрограду, дабы остановить формирование большевистской власти.

Меньшевики создают «Комитет спасения родины революцией», которые должны были присоединиться к армии Краснова.

Против большевистских выступлений выступает ВИКЖель (всероссийский комитет железнодорожников). Начинаются переговоры.

\subsubsection{\textbf{30--31 октября}}


Красная гвардия сумела остановить войска Краснова. Это придало мотивации большевикам и переговоры были разорваны.

Начале ноября кадеты объявляются вне закона (они не принимали перемены большевиков). Большевики приходят к власти в Москве, а затем и в других городах и губерниях.

\subsection{Результаты:}

Две точки зрения:

\begin{enumerate}
    \item К власти в результате октябрьской революции пришло истинное правительство, отражавшее интересы рабочих и крестьян. Началось построение социалистического государства.
    \item К власти в результате октябрьской революции пришли левые радикалы, по своему понимавшие народные интересы и не признававшие компромиссов с другими политическими силами. И их непримиримая позиция привела к распаду страны и гражданской войне.
\end{enumerate}