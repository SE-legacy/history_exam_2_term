\section{Билет 11. Социальная политика и ее реализация в 1920"=е гг}

\subsection{хз как назвать потом еще подумаю}

Составной частью плана перехода от капитализма к социализму была культурная революция. Общественное сознание должно быть перестроено на основе Марскистско"=Ленинской идеалогии, нужно было сформировать единый способ понимания мира. Эта работа по воспитанию молодёжи по принципам коммунизма начинается в 1920"=е годы. Начинать движение приходилось в катастрофических последствиях гражданской войны.

\subsection{Борьба с проблемой беспризорников.}

Крупнейшей проблемой были беспризорники. По отчёту ВЧК, огромное количество беспризорных детей мигрировали на юг. Против этого был поставлен вооруженный заслон. В среде беспризорников распространялась преступность, алкоголизм, наркомания и др. 

Глава ВЧК предпринимает попытку помощи беспризорным детям. В 1921 году создается Комиссия по улучшению жизни детей, в которую вошли представители нескольких народных комиссариатов. На неё возлагались цели по предоставлению детям жилья и охране их жизни. Проводилось распределение по интернатам. Появилось несколько таких видов: детские дома, детские городки, трудовые колонии, трудовые коммуны. 

К борьбе с детской беспризорностью были привлечены профсоюзы, комсомолы и женотделы. На местах при губернских исполкомах создавались комиссии по делам несовершеннолетних. Также губернские власти ведали детскими домами и другими делами содержания. При вокзалах существовали распределительные пункты, которые ловили «зайцев». На вокзалах и пристанях дежурили инспекторы по беспризорникам.

1921--1923 гг проблема еще больше обострилась. Власти стали переселять детей в возрасте 6--14 лет в города не пострадавшие от голода. В 1922 году все существовашие детские дома могли принять только 500 тысяч человек, а счет беспризорников шёл на миллионы. Так что решение проблемы беспризорных распространилась на комсомолы, предприятия, создана общественная организация «Друг детей», организовавшая ночлежки.

\subsection{Перевоспитывание <<дефективных>> подростков. Деятельность Макаренко}

Чтобы перевоспитывать «дефективных» подростков, создавались новые методы перевоспитания. Такие методы формировала Крупская.

Учреждения по перевоспитанию преступных детей также возникали. Такой стала «Колония имени Горького», которую создал Макаров(мб Макаров = Макаренко хз). Её смысл состоял в том, чтобы трудных подростков перевоспитывать на основе «сознательной дисциплины», т.е. объяснять, почему какие-то дела плохие. Макаренко старался строить из них коллектив, способный воспитать каждого его члена. Его идеи критиковались ещё тогда (???почему???).

\subsection{Эмаксипация женщин}

Революция 1917 года ускорила процесс эмансипации женщин в России. Проводилась политика марксистского феминизма. По его идее гендерного неравенства быть не должно. Важнейшую роль реализации этой практики сыграла А. Колонтай. Женщины уже во время революции 1917 года получили избирательные права (на выборах учредительного собрания), а затем избирательные права женщин были закреплены в конституции 1918 года. Затем важную роль играли женотделы. При советах работали женские комиссии. Таким путём достигалось привлечение женщин к управлению и коммунистическому строительству. Легализованы аборты, упрощен процесс развода. Утвержден закон о труде, в котором были пункты, защищавшие права женщин (например отпуск за 8 недель до и после родов). Созданы ясли, детские сады. Советская конституция гарантировала равную оплату за труд для женщин и мужчин. 

Но в 1930-х годах политика эмансипации было свёрнуто. Запрещены аборты, усложнён процесс развода. Тогда же вернулись к традиционалисткой гендерной идеалогии. Сталин упразднил женотделы.

\subsection{Становление гос системы здравоохранения.}

 В годы НЭПа система здравоохранения носила смешанный характер. Тогда создавались страховые фонды, финансировавшиеся за счёт работодателей. После того, как НЭП был свёрнут, система здравоохранения стала полностью государственной. 
 
 Отпуск медикаментов и перевязочных средств был только в государственных клиниках (но бесплатно). 
 
 Частные аптеки просуществовали только до 1925 года, когда их национализировало государство. 
 
 Разрабатывались общегосударственные правила по санитарным мерам. С 1919 года создавались меры противодействия тифу.

 \subsection{Соц стратификация общества.}

 В 1920 годы изменилась социальная стратификация общества. Сословный строй уничтожила февральская революция, а после гражданской войны основными категориями СССР были: рабочие и служащие (17\% к 1930 годам), крестьяне и ремесленники единоличники (до 79\%), колхозное и кооператорное крестьянство (3\%), мелкая городская буржуазия и торговцы, кулаки (НЭПманы, 4,5\%).

 \subsection{Биржи труда, кодекс о работе и труде.}

 По мере восстановления промышленности на фабрики и заводы возвращались рабочие. Кодекс о работе и труде вернул свободный выбор рабочего места. Однако слабая промышленность вызывала безработицу. Поэтому власть создаёт биржи труда. 

 \subsection{НАДО ДОДЕЛАТЬ ИБО ЭТО НЕ ВСЕ ПУНКТЫ ОПИСАННЫЕ В МЕТОДИЧКЕ}

 даже лучше еще предыдущий пункт дополнить 

 Последнее что в лекции было:

 Сельское население также изменилось. Доля кулаков (зажиточные крестьяне, использующие наёмный труд, дававшие деньги или продукты в долг) – 4\%, середняки – 62\% (Не эксплуатирует чужого труда, работает сам), бедняцкие, пролетарские хозяйства – 33\% (безлошадные, безземельные крестьяне).