\section{Билет 11. Социальная политика и ее реализация в 1920"=е гг}

\subsection{хз как назвать потом еще подумаю}

Составной частью плана перехода от капитализма к социализму была культурная революция. Общественное сознание должно быть перестроено на основе Марскистско"=Ленинской идеалогии, нужно было сформировать единый способ понимания мира. Эта работа по воспитанию молодёжи по принципам коммунизма начинается в 1920"=е годы. Начинать движение приходилось в катастрофических последствиях гражданской войны.

\subsection{Борьба с проблемой беспризорников.}

Крупнейшей проблемой были беспризорники. По отчёту ВЧК, огромное количество беспризорных детей мигрировали на юг. Против этого был поставлен вооруженный заслон. В среде беспризорников распространялась преступность, алкоголизм, наркомания и др. 

Глава ВЧК предпринимает попытку помощи беспризорным детям. В 1921 году создается Комиссия по улучшению жизни детей, в которую вошли представители нескольких народных комиссариатов. На неё возлагались цели по предоставлению детям жилья и охране их жизни. Проводилось распределение по интернатам. Появилось несколько таких видов: детские дома, детские городки, трудовые колонии, трудовые коммуны. 

К борьбе с детской беспризорностью были привлечены профсоюзы, комсомолы и женотделы. На местах при губернских исполкомах создавались комиссии по делам несовершеннолетних. Также губернские власти ведали детскими домами и другими делами содержания. При вокзалах существовали распределительные пункты, которые ловили «зайцев». На вокзалах и пристанях дежурили инспекторы по беспризорникам.

1921--1923 гг проблема еще больше обострилась. Власти стали переселять детей в возрасте 6--14 лет в города не пострадавшие от голода. В 1922 году все существовашие детские дома могли принять только 500 тысяч человек, а счет беспризорников шёл на миллионы. Так что решение проблемы беспризорных распространилась на комсомолы, предприятия, создана общественная организация «Друг детей», организовавшая ночлежки.

\subsection{Перевоспитывание <<дефективных>> подростков. Деятельность Макаренко}

Чтобы перевоспитывать «дефективных» подростков, создавались новые методы перевоспитания. Такие методы формировала Крупская.

Учреждения по перевоспитанию преступных детей также возникали. Такой стала «Колония имени Горького», которую создал Макаренко. Её смысл состоял в том, чтобы трудных подростков перевоспитывать на основе «сознательной дисциплины», т.е. объяснять, почему какие-то дела плохие. Макаренко старался строить из них коллектив, способный воспитать каждого его члена. Его идеи за "несоответствие советской системе". Затем, ввиду многочисленнных обвинений Макаренко был вынужден покинуть пост руководителя колонии в 1928 году.

\subsection{Эмаксипация женщин}

Революция 1917 года ускорила процесс эмансипации женщин в России. Проводилась политика марксистского феминизма. По его идее гендерного неравенства быть не должно. Важнейшую роль реализации этой практики сыграла А. Колонтай. Женщины уже во время революции 1917 года получили избирательные права (на выборах учредительного собрания), а затем избирательные права женщин были закреплены в конституции 1918 года. Затем важную роль играли женотделы. При советах работали женские комиссии. Таким путём достигалось привлечение женщин к управлению и коммунистическому строительству. Легализованы аборты, упрощен процесс развода. Утвержден закон о труде, в котором были пункты, защищавшие права женщин (например отпуск за 8 недель до и после родов). Созданы ясли, детские сады. Советская конституция гарантировала равную оплату за труд для женщин и мужчин. 

Но в 1930-х годах политика эмансипации было свёрнуто. Запрещены аборты, усложнён процесс развода. Тогда же вернулись к традиционалисткой гендерной идеалогии. Сталин упразднил женотделы.

\subsection{Становление гос системы здравоохранения.}

 В годы НЭПа система здравоохранения носила смешанный характер. Тогда создавались страховые фонды, финансировавшиеся за счёт работодателей. После того, как НЭП был свёрнут, система здравоохранения стала полностью государственной. 
 
 Отпуск медикаментов и перевязочных средств был только в государственных клиниках (но бесплатно). 
 
 Частные аптеки просуществовали только до 1925 года, когда их национализировало государство. 
 
 Разрабатывались общегосударственные правила по санитарным мерам. С 1919 года создавались меры противодействия тифу.

 \subsection{Соц стратификация общества.}

 В 1920 годы изменилась социальная стратификация общества. Сословный строй уничтожила февральская революция, а после гражданской войны основными категориями СССР были: рабочие и служащие (17\% к 1930 годам), крестьяне и ремесленники единоличники (до 79\%), колхозное и кооператорное крестьянство (3\%), мелкая городская буржуазия и торговцы, кулаки (НЭПманы, 4,5\%).

 \subsection{Биржи труда, кодекс о работе и труде.(будто бы имеет смысл дополнить, но мне кажется не стоит).}

 По мере восстановления промышленности на фабрики и заводы возвращались рабочие. Кодекс о работе и труде вернул свободный выбор рабочего места. Однако слабая промышленность вызывала безработицу. Поэтому власть создаёт биржи труда. 

 \subsection{Феномен лишенцев.}

При выборах депутатов в органы власти преимущество попрежнему имели рабочие. Часть населения, как и раньше, была лишена избирательных прав («лишенцы»).

Лишение избирательных прав резко расширилось в конце 1920"=х годов. Во время избирательной кампании 1928–-1929 гг. в СССР в списки лишенцев были внесены 3,7 млн человек, или 4,9\% взрослого населения (4,1\% жителей сельской местности и 8,5 "--- горожан).

\subsection{Деревенский социум: бедняки, середняки и кулаки.}

В системе налогообложения основная тяжесть приходилась на частных предпринимателей в городе и кулаков "--- в деревне. Бедняки от уплаты налогов освобождались, середняки платили половину.

Кулацкими считались хозяйства, применявшие наемный труд и машины с механическим приводом, а также занимающейся торговлей. В 1929 г. на их приходилось 2,5--3\% общего числа крестьянских дворов.

Основные слои деревни "--- середняки.

В соответствии с постановлениями конца 20"=х "--- начала 30"=х годов прекращалось кредитование и усиливалось налоговое обложение частных хозяйств, отменялись законы об аренде земли и найме рабочей силы. Было запрещено принимать кулаков в колхозы. Все эти меры вызывали их протесты и террористические действия против колхозных активистов. В феврале 1930 г. был принят закон, определивший порядок ликвидации кулацких хозяйств. В соответствии с ним слои кулачества разделяли на три категории. В первую включались организаторы антисоветских и антиколхозных выступлений. Они подвергались аресту и суду. Наиболее крупных кулаков, отнесенных ко второй категории, надлежало переселять в другие районы. Остальные кулацкие хозяйства подлежали частичной конфискации, а их владельцы "--- выселению на новые территории из областей прежнего проживания. В процессе раскулачивания были ликвидированы 1--1,1 млн. хозяйств (до 15\% крестьянских дворов).

\subsection{Вопросы общественной морали: советские праздники, церковь.}

\subsubsection{\textbf{Советские праздники, советизация имен и топонимики.}}

План монументальной пропаганды предусматривал создание серии памятников революционерам и прогрессивным деятелям «всех времен и народов». Общественные здания, учреждения, а также праздники и другие массовые мероприятия оформлялись знаменами, плакатами, транспарантами, содержание которых пропагандировало цели новой власти, величие труда, союз рабочих и крестьян («Что несет революция трудящимся»; «Мир народов будет заключен на развалинах буржуазного владычества»; «Заводы — трудящимся»; «Земля — крестьянам» и т.п.)

\subsubsection{\textbf{Политика советского государства по отношении к церкви.}}

Советское правительство нанесло удар по Русской православной церкви и поставило ее под свой контроль, несмотря па декрет об 253 отделении церкви от государства. В 1922 г. под предлогом сбора средств для борьбы с голодом была конфискована значительная часть церковных ценностей. Усиливалась антирелигиозная пропаганда, разрушались храмы и соборы. Начались преследования священников. Патриарх Тихон был заключен под домашний арест. После смерти Тихона в 1925 г. правительство воспрепятствовало избранию нового патриарха. Местоблюститель патриаршего престола митрополит Петр был арестован. Его преемник, митрополит Сергий, и 8 архиереев вынуждены были проявить лояльность к советской власти. В 1927 г, они подписали Декларацию, в которой обязывали священников, не признававших новую власть, отойти от церковных дел.

В конце 20"=х годов усилилось государственное регулирование деятельности религиозных объединений. К этому времени почти все религиозные организации объявили о своей лояльности к новому строю. Началась разработка союзного закона о религиозных культах. Обсуждение его проекта проводилось в ведомствах, осуществляющих «церковную политику»: НКВД, Президиуме ЦИК СССР. В ходе обсуждения развернулась дискуссия о перспективах религии в советском обществе, о характере деятельности культовых организаций, о формах антирелигиозной пропаганды. Утверждалось, что работа многих церковных общин приобрела антисоветский характер. Предлагалось усилить борьбу с ними как с контрреволюционной силой. Было решено сохранить существующее в республиках законодательство по отношению к религии.

\subsection{Обновленчество.}

Протест церковников против политики Тихона проявился и в так называемом «обновленческом» движении (возглавлялось протоиереем А. И. Введенским), участники которого требовали прекращения «Гражданской войны Церкви против государства».

\subsection{Позиция патриарха Тихона по отношению к советской власти.}

Патриарх Тихон направил правительству послание, в котором предал большевистскую власть анафеме; принятые в отношении Церкви акты он назвал проявлением «самого разнузданного своеволия и сплошного насилия на святой Церковью». 
