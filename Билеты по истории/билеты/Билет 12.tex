\section{Билет 12. Индустриализация и Коллективизация.}

\subsection{Провозглашение курса на форсированную индустрилизационную экономику.}

С 1928 года советское правительство провозглашает курс на форсированную индустриализацию экономику. 

\textbf{Причины:}

\begin{enumerate}
    \item В годы НЭПа внутрипартийная борьба происходила в том числе из-за методов построения социализма. В результате одержала победу точка зрения Сталина, заключавшаяся в том, что социализм в отдельно взятой стране возможен. С конца 1920-х годов власти активно этим занимаются.
    \item Необходимость изменений вызывалась хроническими трудностями в экономики. В конце 1920-х годов экономика достигла результатов 1916 года, но запасы были исчерпаны. Росло городское население, а значит и требовалось увеличение продукции. Иностранный капитал практически отсутствовал. Всё это происходило на фоне стагнации сельского хозяйства. Сбор зерна не менялся год от года. Всё это влияло на начало индустриализации и коллективизации.
\end{enumerate}

С 1927 года начался дефицит продукции в городах, вызванный тем, что крестьяне не хотели сдавать зерно государству. Чтобы решить эти проблемы, сталин предложил перестроить аграрную экономику на основе коллективных хозяйств. Его идеи критиковали некоторые члены партии (Бухарин, Рыков, Томский, и другие члены оппозиции). Но в 1929 году центральный коммитет окончательно одобрил сталинскую программу.

\subsection{Цели индустиализации:}

\begin{enumerate}
    \item Ликвидация технико-экономической отсталости от передовых западных стран
    \item Создание мощного военно-промышленного комплекса
\end{enumerate}

Акцент делался на первоочередном развитии базовых отраслей: энергетика, топливная, металлургическая, химическая, машиностроительная. 

\subsection{Методы индустриализации:}

\begin{enumerate}
    \item Централизованное планирование и полный контроль со стороны государства.
    \item Выкачивание финансовых ресурсов из деревни (коллективизация стала одним из главных инструментов).
    \item Продажа заграницу максимального количества полезных ископаемых, продовольствия (вместе с ограничением внутреннего потребления), музейных и церковных ценностей.
    \item Масштабная пропагандистская кампания и использование энтузиазма населения.
    \item Безжалостная борьба с вредителями, репрессии в отношении противников политики партии.
\end{enumerate}

\subsection{Первая пятилетка}

Госплан (главный орган государственного планирования) выработал первый пятилетний план (1928 – 1932). По этому плану требовалось создать 1500 крупных промышленных предприятий, из которых выделили 50-60 самых важных, которые получали все ресурсы в первую очередь. Стоимость этих 50-60 предприятий требовала более половины всего бюджета на промышленность. Но из этих предприятий было 14 самых важных строек, которые курировал лично Орджоникидзе.

\subsection{Трудности и противоречия <<Большого скачка>>.}

Искусственное взвинчивание темпов Роста промышленности привело к серьезному нарушению баланса между отраслями:
\begin{itemize}
    \item Рапыление ресурсов на строительство не предусмотренных планом объектов.
    \item Нехватка механизмов и строительных материалов.
    \item Нехватка инженеров и техников.
    
\end{itemize}

Но большевикам удалось создать обстановку вдохновенного труда. Строители, мечтая преобразить новостройки в <<город"=сад>>, жили в  бараках и землянках, скудно питались и тд.

Тем не менее новые заводы строились очень быстро.

Так же проблемой было освоение новых предприятий из-за того, что многим рабочим в стране катастрофически не хватало квалификации (с трудом осваивались с непривычной заводской техникой). Запускать новые предприятия помогали специалисты из других стран.

\subsection{Вторая пятилетка (1933--1937)}

Вступают в строй промышленные гиганты: Челябинский тракторный завод, Уральский вагоностроительный завод (Свердловск), Авиазаводы (Москва, Харьков, Куйбышев, Саратов).
Форсированная индустриализация отразилась на социальных процессах. Выросла численность городского населения (32\% в 1939 г). Вес национального производства достиг 52\% в 1937 году, из-за чего страна стала индустриальной. 

\subsubsection{\textbf{Энтузиазм населения и Стахановское движение.}}

Эти успехи были достигнуты колоссальным усилием всей страны. Велась работа партийных, комсомольских органов, людей мотивировали отдавать все силы. Появилось Стахановское движение в честь Алексея Стаханова, шахтёра, который за смену перекрыл план в 14 раз. Введено социалистическое соревнование, когда разные бригады на одном предприятии соревновались в продуктивности.

\subsubsection{\textbf{Применение заключенных в качестве рабочей силы}}

Большим и бесплатным ресурсом рабочей силы стали заключенные в лагеря при ОГПУ. Они привлекались к строительным и другим работам на крупнейших стройках.

\subsection{Итоги индустриализации.}

В эпоху индустриализации полностью уничтожены последние остатки рыночной экономики. Вся деятельность предприятий контролировалась наркоматами. В конце 1930-х годов числилось более 20 наркоматов.


\subsection{<<Революция сверху>> в советской деревне (как я понял так коллективизацию обозвали). Хлебозаготовки(про них нет инфы) и черзвычайные меры.ы}

(сюда можно и нужно впихнуть про хлебо заготовки, но у меня про них ничего нет)

Требовалась и перестройка аграрного сектора. С 1928 года начались репрессии крестьян-единоличников, вынуждая их вступать в колхозы и совхозы. Колхоз"--- коллективное хозяйство, рассматривался как добровольное объединение крестьян. Совхоз"--- советское хозяйство"--- сельскохозяйственная единица, на которую государство нанимает рабочих и платит им фиксированную зарплату.

С 1929 года Сталин объявляет курс на сплошную коллективизацию сельского хозяйства и объявляет о ликвидации кулачества как класса. Для создания колхозов и совхозов власти применяли как принудительные меры, так и раздачей льгот. При ликвидации кулаков применялись принудительные меры изъятия.

\subsection{Раскулачивание как часть коллективизации.}

Обострило ситуацию постановление(Политбюро ЦК) 1930 года "О мероприятиях по ликвидации кулацких хзяйств в районах сплошной коллективизации" (оно отменяло закон об аренде и применении наемного труда, предписывая конфисковать у кулаков средства производства, скот, хоз и жилые постройки и тд). Примерно 10\% от всех раскулаченных приговаривались к заключению в лагеря.

К 1930 году с первоначального места жительства было выселено более 3 млн. крестьян.

Те, кто вступал в колхозы и совхозы, власть на время освобождала от налогов. Переход к колхозам и совхозам привел к снижению производительности сельского хозяйства из-за того, что крестьянство не хотело делить свой скот с кем-либо еще.

\subsection{Голод 1932--1933 гг.}

Голод 1932-1933гг охватил ряд зерновых районов на Украине и в России и унес жизни 5 млн чел (впихивать сюда подробную статистику смертей и тд не считаю нужным). Немало было случаев людоедства.

\subsection{Колхозная система.}

Система паспартов 1932г (конторлируемая миграция)
колхозники получали паспорта только с согласия правления колхоза.

Сильный приток крестьян в колхозы (практически все)

Получили силу законы об обязательной поставке продукции государству, закупки происходили по фиксированным ценам, а в деревнях проводился набор рабочих на промышленные предприятия. 

С 1935 года колхозникам для переезда требовалась справка от колхоза, которую выдавали неохотно.

Происходящее в колхозах контролировалось не только советами, но и партийными организациями.

\subsection{Итоги коллективизации.}

Коллективизация окончательно уничтожила частную собственность и превратила крестьянина в колхозника. Уровень жизни оставался низким, люди жили бедно. 