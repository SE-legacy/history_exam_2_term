\section{Билет 4. Построение советского государства и социально-экономические преобразования большевиков в 1917-1918 гг. Политика военного коммунизма.}

\subsection{Июльские события"---первая попытка захвата власти большевиками(?)}

\subsubsection{\textbf{Ноябрь 1917 г.}}

 Происходят выборы в учредительное собрание. Они проходят по «четырёхвостке» (прямые, тайные, всеобщие, равные). На этих выборах наибольшее количество голосов (40\%) получают эсеры. Большевики (22\%). Дальше меньшевики, кадеты, беспартийные и т.д.
 
Надежды Ленина, о том, что по результатам выбором будет колоссальные поддержка большевиков, не оправдалась.

\subsubsection{\textbf{3 января 1918 г.}}

ВЦИК издаёт постановление: всякая попытка установить любую форму власти кроме советов, будет рассматриваться как попытка революционного свержения власти.

\subsubsection{\textbf{5 января}}

В Таврическом дворце собирается учредительное собрание. Большинство делегатов учредительного собрания отказываются принимать временную власть и требуют передать её им.

Ленин предлагает делегатам выговориться, но утром никого не пускать. Так и произошло 6 января. Позднее по указу Ленина произошёл роспуск учредительного собрания.

На третьем всероссийском съезде советов принимается «Декларация прав трудящихся и прав народа». По ней права управления переходят к трудящимся, а эксплуататоры никаких прав не получали.

\subsection{Начались экономические преобразования.}

\subsubsection{\textbf{14 ноября}}

Принят «Рабочий контроль».

\subsubsection{\textbf{2 декабря}}

Cоздаётся ВСНХ. (высший совет народного хозяйства).

\subsubsection{\textbf{14 декабря}}

Принимается декрет о национализации банков

\subsubsection{\textbf{Апрель 1918г}}

Национализация внешней торговли.

\subsubsection{\textbf{Май 1918}}

Национализация всех промышленных предприятий. В результате этих действий вся промышленность оказалась парализована. Это усиливает продовольственный дефицит. Вводится государственная монополия на оборот хлеба, то есть, продовольственная диктатура. Городские отряды отправляются в деревни и изымают «излишки хлеба».

\subsubsection{\textbf{11 июня 1918 г.}}

Выходит декрет об организации деревенской бедноты. В селах и деревнях создаются КомБеды (комитеты бедноты), отбиравшие продукты сельского хозяйства у зажиточных крестьян (кулаков). Эти меры легли в основу военного коммунизма. Эта политика характеризуется принудительным трудом, учётом и контролем, государственным распределением.

\subsection{Увеличивается роль репрессивных органов.}

Появляется всероссийская чрезвычайная комиссия.

\textbf{Задача:} борьба с контрреволюционерами.

\textbf{Во главе:} Дзержинский.
Применяются всевозможные репрессивные методы.

\subsection{Внешняя политика}

\subsubsection{\textbf{6 декабря 1917 г.}}

Совнарком признаёт независимость Финляндии.

Отталкиваясь от декрета о мире народный комитет об иностранных делах (Троцкий) инициирует переговоры с Германией в конце 1917 года. После подавление восстания Корнилова его место в совете народных депутатов занял Духонин. Во время переговоров последний отказался вести переговоры. Его убили из-за несогласия и на его место был поставлен прапорщик Крыленко, направивший парламентёров для переговоров с иностранным командованием.

\subsubsection{\textbf{2 декабря 1917г.}}

Подписано перемирие. Прекращаются военные действия против четверного союза.

Антанта, недовольная этим, предпринимает шаги по предотвращению выхода России из союза. Они договариваются с радикальными группировками, противостоящие нынешней власти.

Япония высаживает десант во Владивостоке, тем самым начиная интервенцию.

Вудро Вильсон обращается к конгрессу с призывом не допустить занятие территорий России Германией, признать существующие правительства в Финляндии, Польше. Эти тезисы неприемлемы для Ленина, нарастает конфронтация.

К 1918 году армия теряет боевую мощь.

\subsubsection{\textbf{15 января 1918г.}}

Совнарком выпускает декрет о создании «Рабоче-крестьянской красной армии», а затем «Рабоче-крестьянский красный флот», создающиеся на добровольной основе.

Пока длится перемирие, начинаются полноценные мирные переговоры. Они затягиваются, так как германское командование видит невозможность русской армии сопротивляться и настаивает на признании оккупированных германскими войсками территорий: Польша, Прибалтика. Ленин с этим не согласен.

\subsubsection{\textbf{11 января 1918г.}}

Троцкий издает приказ: «Войну прекращаем, мир не заключаем, армию демобилизуем». Но после тупика военных переговоров, германия расторгает перемирие и с 18 января начинается наступление немецкой армии. В течение 2 недель немецкая армия существенно продвинулась и практически приблизились к Петрограду.

Ленин призывает всё население встать на защиту Отечества. 23 февраля начинается массовая запись добровольцев в РККА. Но этого недостаточно, чтобы остановить германское наступление.

Советская делегация на переговорах с четверным союзом, 3 марта подписывает мирный договор в городе Брест. 15 марта на чрезвычайном всероссийском съезде советов ратифицируется. По этому договору отторгаются Польша, Эстляндия, Лифляндия, Курляндия, западная часть Белорусии, Мандзунские острова, Карс, Батум (в закавказье). Кроме того Россия обязуется выплатить контрибуции в размере 6 млн марок в золоте, демобилизовать армию и признать правительство Украины. Однако точные границы по брестскому миру не были обозначены, поэтому Германия этим воспользовалась и продолжила наступление.

К апрелю 1918 года войска германии занимают все территории Украины, Крым и Ростов"=на"=Дону.

Тем временем Брестский мир позволил на какое-то время сконцентрировать силы на других задачах. В июле 1918 года на пятом съезде советов принимается конституция, в которой закрепляется название «Российская Советская Федеративная социалистическая республика» (РСФСР). Эта конституция закрепила в РСФСР диктатуру пролетариата и беднейшего крестьянства. Часть населения лишалась прав: торговцы, служители культов, использовавшие наёмный труд, бывшие полицейские. Выборы были неравные, так как один депутат пролетариата выбирался из 25 тысяч, а один депутат из крестьянства из 125 тысяч. Конституция зафиксировала существовавшие органы правления.

Партия РСДРП(б) переименовывается в Российскую социалистическую партию большевиков.