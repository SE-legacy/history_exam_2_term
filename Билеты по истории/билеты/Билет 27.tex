\section{Билет 27. Предпосылки и обстоятельства распада СССР.}

\subsection{Предпосылки}
\begin{enumerate}
    \item Этническая разнородность и подавление автономий, 
    \item Многолетняя неэффективность и стагнация экономики
    \item Кризис идеологии коммунизма
    \item Развитие гласности и демократизации общества, вызвавшая бурю обсуждений проблем
\end{enumerate}

\subsection{Обострение межнациональных конфликтов в период перестройки}
По мере того, как в конце 1980"=х происходила демократизация политической системы, на поверхность вышел национальный вопрос. В советском союзе существовали нации, между которыми вновь обострилась старая вражда. Идея национального самоопределения опять стала популярной. Отчасти этому способствовала гласность и реабилитация репрессированных --- тех, кто когда-то раньше пострадал за свою деятельность национально"=освободительного толка. Началась борьба за национальное самоопределение. Первое подобное событие происходит в 1987 году, когда правительство Горбачёва сняла с должности главу Казахстана --- Кунаева (очень уважаемый человек). Для Горбачева во время перестройкой Кунаев был неудобной фигурой, так как он был союзником старой команды Брежнева. В ответ на это в Алма"=Ате происходят массовые акции протеста, которые разгоняются властями. 

1988 году областной совет Нагорно"=Карабахской автономной области принимает решение о выходе из Азербайджана и вступление в Армению. Это вызвало взрыв взаимных обвинений. В Азербайджане растут антиармянские настроений и начинаются погромы в Сумгаите. Жертвами стали от нескольких десятков до нескольких сотен человек. По распоряжению Горбачева для предотвращения беспорядков туда вводятся войска. Появляется демонстрация в Тбилиси в 1989 году. Тогда же происходит конфликт в Узбекистане между узбеками и турками. Власти, занятые перестройкой, не видели способа быстрого преодоления этого кризиса. 

\subsection{Поиск решения национальных проблем. <<Парад суверенитетов>>}
Правительство Горбачева сделало новый союзный договор, пытаясь сохранить структуру СССР, повысив республики до федераций и дать им больше прав. Однако это не помогает. Союзные республики принимают документы о суверенитете. Начинается <<Парад суверенитетов>>. До августа 1991 года из состава СССР выходят Эстония, Латвия, Литва, Грузия, Азербайджан. 

12 июня 1990 года первым съездом народных депутатов СССР принимается декларация о российском суверенитете.

Результатом парада стал выход из СССР ряда республик, их отказ подчиняться советскому законодательству, в частности уплате налогов. Это ещё больше усилило кризис советской экономики.

\subsection{Референдум}

В таких обстоятельствах 17 марта 1991 года состоялся референдум с одним вопросом: «Считаете ли вы необходимым сохранение СССР, как обновленной федерации равноправных суверенных республик, которым будут в полной мере гарантироваться права и свободы человека любой национальности». Этот референдум не проводился в 6 республиках, где верховные органы власти уже объявили о независимости: Эстония, Латвия, Литва, Грузия, Армения, Молдавия. 

По результатам референдума, «За» выступило 76\% всех участников референдума. Самые высокие результаты «За» были в кавказских республиках и Белорусии, ниже всего было в РСФСР и Украине. К лету 1991 года уже большинство союзных республик приняло законы о суверенитете, что ускорило разработку нового союзного договора, который в историю вошел как «Ново"=огарёвский процесс». 

\subsection{Новоогарёвский процесс. «Договор о Союзе Суверенных государств»}

Подписание его назначено на 20 августа 1991 года. По новому договору сохранялось единое государство, которое должно было приобрести федеративную устройство, с ликвидацией правящей партии. Республики входящие в союз получали больше прав, а основное государство носило координирующий характер. Когда этот проект был опубликован, те же 6 независимых республик отказались его подписывать. 

\subsection{Августовский путч 19 августа 1991 года}

Против федерализации государства выступили консервативные силы во главе с руководством СССР. Чтобы сохранить позиции, часть руководства попыталась захватить власть и создают ГКЧП (государственный комитет по чрезвычайному положению) в составе: Янаев (вице-президент СССР), Крючков (Председатель КГБ), Павлов (премьер"=министр и министр финансов), Пуго (Министр внутренних дел), Язов (Министр обороны). Их главной задачей являлось сохранение СССР и восстановление тех порядков, что существовали до 1985 года. 
Начинается попытка переворорта (августовский путч) 19 августа 1991 года. 

В стране объявляется чрезвычайное положение, прекращается вещание всех каналов кроме первого, на котором диктор зачитывает сообщение. На улицы всех городов выводятся войска, закрываются все газеты кроме «Правды» и «Вестей», перестают вещать все радиостанции. В объявлении говорится, что президент СССР Горбачев, по состоянию здоровья не может выполнять свои обязанности, поэтому его обязанности переходят к Янаеву. В этом же объявлении говорится, что деятельность всех партий, кроме КПСС приостанавливается.

Сам Горбачёв был в одной из резиденций в Крыму. Туда приехали несколько путчистов, которые пытались получить от него согласие на захват власти. Горбачёв, насколько мы знаем, долго колебался, но согласия прямого не дал.

Основным политическим соперником центрального руководства СССР являлось руководство РСФСР во главе с Ельциным. Поэтому против руководства РСФСР были направлены основные действия путчистов. Вокруг белого дома в Москве были собраны войска, чтобы разогнать парламент и арестовать его участников, но план этот не был реализован. Сам переворот был организован и подготовлен очень плохо. Уже 20 августа вокруг белого дома стали выстраиваться баррикады, на которых находились десятки тысяч человек: не только москвичи, но и приезжие. Часть воинских поразделений переходило на сторону обороняющихся. В конце концов 20--21 августа погибло 3 человека, но войска отказались подчиняться ГКЧП и 22 августа члены ГКЧП были арестованы. 

\subsection{Манифестации с лозунгами «Против КПСС»}
Сразу же за поражением путча прошли массовые манифестации с лозунгами «Против КПСС», что послужило удобным поводом приостановки партии в стране. По указанию Ельцина были опечатаны центры обкомов, райкомов и др. КПСС перестаёт существовать, как правящая политическая структура. Но их экономическое влияние не пропало, просто перешло в другие подразделения страны. Закрываются партийные газеты «Труд», «Россия» (но вскоре заново открываются из-за протеста).

\subsection{Распад СССР. Последствия}
Уже в сентябре 1991 года все союзные республики, которые раньше не заявляли о независимости, сделали это. Высшим органом республик оставались верховные советы, но власть все больше концентрировалась в руках республиканских президентов. Значение верховного совета СССР и съезда народного депутатов СССР сошли на нет. Пятый съезд стал последним, на котором те заявили о самороспуске. 

Горбачёвым в 1991 году предпринимались попытки предотвратить распад страны, но они ни к чему не привели. Продолжался экономический кризис, пытаясь его смягчить, республики начали вводить запрет на вывоз продукции, что нарушало прежние цепочки поставок. В ноябре 1991 года на уровне президентов между РСФСР и среднеазиатскими странами проводились переговоры о создании конфедерации.
Везде отстраняются советы и исполкомы, появляются мэрии и появляется должность мэра.

После провозглашения независимости между разными республиками, начинаются пересматриваться границы. Наиболее ярко это проявилось в возникновении Чеченской республики. События в Чечне и неутихающая война в Южной Осетии, оставили Кавказ на грани всеобъемлющей гражданской войны.

Расли темпы инфляции, сокращалось промышленное и хозяйственное производство. Правительство для расплаты по счетам начинает неограниченную эмиссию, но это приводит к тому, что начинается дефицит товаров. Для многих категорий населения встала реальная проблема выживания. Во многом ситуация была вызвана действиями Ельцинского правительства. Ряд республик начинает вводить свою валюту, это же заявляет РСФСР (это вызывает панику и массовую попытку избавиться от советских рублей).

Надежды на выход из кризиса возлагались на финансовую помощь Запада. Кредиты запада приводили к высокому росту гос.долга до 70 млрд. долларов к концу 1991 года.

Фактически СССР перестал существовать к сентябрю 1991 года. Повторилась ситуация 1917 года.

С крушением КПСС пропал последний институт, объединявший республики "--- Съезд народных депутатов, формировавшийся на основании той же номенклатуры. Не мог выполнить объединяющую роль и президент СССР. Оба этих властных института оказались дискредитированы в ходе перестройки множеством провалов в экономической, социальной сферах, потеряли авторитет.

После ГКЧП распад союза имел лавинообразный характер.

Окончательно распад СССР происходит в конце 1991 года. 1 декабря 1991 года проводится референдум в Украине (объявила о независимости 24 августа). Если на референдуме 17 марта большинство жителей Украины высказались ЗА сохранение СССР, то 1 декабря большинство высказалось ЗА независимость. Вопрос был сформулирован: «Вы за независимость Украины?». Первый президент независимой Украины Кравчук заявил, что будет заключать договоры с союзными государствами, но не войдет в состав, если над союзными государствами будет что-то ещё. 

\section{Учреждение Содружества независимых государств}

8 декабря 1991 года заключаются беловежские соглашения. Россия, Украина и Белоруссия собрались в беловежской пуще. Ельцин, Кравчук и Шушкевич собрались заявить о том, что советский союз прекращает своё существование, но вместо него создаётся Содружество Независимых Государств (СНГ). А президент Горбачёв и другие советские органы власти должны были признать текущие реалии. 
21 декабря произошло собрание лидеров государств в Алма-Ате, провозгласило, что 11 бывших республик СССР входят в СНГ, кроме Эстонии, Латвии, Литвы и Грузии. 25 декабря Горбачёв последний раз выступает с обращением к населению в качестве президента несуществующего государства, заявляет о прекращении полномочий. 31 декабря некому было поздравлять население с Новым Годом. Выступает с обращением сатирик Задорнов. На этом закончилась история СССР, как единого государства, а правопреемником СССР стала Россия.
